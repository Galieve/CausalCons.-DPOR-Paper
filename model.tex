%!TEX root = main.tex
\section{Prefix-Closed and Causally-Extensible Isolation Levels}

TODO THE INTUITION BEHIND THIS CONDITION (USE THE TEXT THAT FOLLOWS)

Besides models presented in figure \ref{fig:consistency_defs}, others isolation levels exists in literature and real life applications \textcolor{red}{cite Constantin's papers + Twitter, shoppingcart...}. However, our algorithm can not be analyzed under an arbitrary model. We characterize in this section the ones that can be employed by our algorithm.
%not all of them can verified with the algorithm

\begin{figure}[H]
	
	\centering
	\begin{subfigure}[b]{.25\textwidth}
		\resizebox{\textwidth}{!}{
			\begin{tikzpicture}[->,>=stealth',shorten >=1pt,auto,node distance=3cm,
				semithick, transform shape]
				\node[draw, rounded corners=2mm,outer sep=0] (t1) at (-3.25, 0) {\begin{tabular}{l} $\init$ \end{tabular}};
				\node[draw, rounded corners=2mm,outer sep=0] (t2) at (-3.25, -2) {\begin{tabular}{l} 
						$\rd{x}$ \pgfsetfillopacity{0.3}\\ $\rd{y}$
				\end{tabular}};
				\node[draw, rounded corners=2mm,outer sep=0] (t3) at (0, 0) {\begin{tabular}{l} 
						$\wrt{x}{2}$%\\
						%\pgfsetfillopacity{0.3}$\wrt{y}{2}$
				\end{tabular}};			
				
				\path (t1.south west) -- (t1.south) coordinate[pos=0.67] (t1x);
				\path (t2.north west) -- (t2.north) coordinate[pos=0.67] (t2x);
				\path (t3.north west) -- (t3.west) coordinate[pos=0.67] (t3x);
				
				\path (t3.west) edge [below] node[right] {$\wro_x$} (t2.north east);
				\path (t1.south) edge [above] node[left] {$\so$} (t2.north);
				\path (t1.east) edge [above] node[above] {$\so$} (t3.west);
				%\path (t1x) edge [right] node {$\wro_y$} (t2x);
			\end{tikzpicture}  
			
		}
		\caption{Extensible history.}
		\label{fig:maxclosed:a}
	\end{subfigure}
	\hspace{.75cm}
	\centering
	\begin{subfigure}[b]{.25\textwidth}
		\resizebox{\textwidth}{!}{
			\begin{tikzpicture}[->,>=stealth',shorten >=1pt,auto,node distance=3cm,
				semithick, transform shape]
				\node[draw, rounded corners=2mm,outer sep=0] (t1) at (-3.25, 0) {\begin{tabular}{l} $\init$ \end{tabular}};
				\node[draw, rounded corners=2mm,outer sep=0] (t2) at (-3.25, -2) {\begin{tabular}{l} 
						$\rd{x}$ \\ $\rd{y}$
				\end{tabular}};
				\node[draw, rounded corners=2mm,outer sep=0] (t3) at (0, 0) {\begin{tabular}{l} 
						$\wrt{x}{2}$\pgfsetfillopacity{0.3}\\
						$\wrt{y}{2}$
				\end{tabular}};			
				
				\path (t1.south west) -- (t1.south) coordinate[pos=0.67] (t1x);
				\path (t2.north west) -- (t2.north) coordinate[pos=0.67] (t2x);
				\path (t3.north west) -- (t3.west) coordinate[pos=0.67] (t3x);
				
				\path (t3.west) edge [below] node[right] {$\wro_x$} (t2.north east);
				\path (t1.south) edge [above] node[left] {$\so \cap \wro_y$} (t2.north);
				\path (t1.east) edge [above] node[above] {$\so$} (t3.west);
			\end{tikzpicture}
			
		}
		\caption{Non-extensible history.}
		\label{fig:maxclosed:b}
	\end{subfigure}
	\hspace{.175cm}
	\centering
	\caption{Example of a dead-lock after swapping two events. We write $\so \cap \wro_x$ when there is a $\so$ and a $\wro_x$ edge sharing both source and target.}
	\label{fig:maxclosed}
	%\vspace{-3mm}
\end{figure}

Let's analyze the histories $h_1$ and $h_2$ described in Figures~\ref{fig:maxclosed:a} and \ref{fig:maxclosed:b} respectively under RA; isolation level under which both are consistent. $h_1$ can be extended adding the event $r_1 = \rd{y}$ and the $\wro$-edge $w_1 \ [\wro] \ r_1$, where $w_1 = \wrt{x}{0}$. However, this is not the case of $h_2$: the only event that could be added in it is $w_2 = \wrt{y}{2}$. If $w_2$ would be added in $h_2$, any relation extending $\so \cup \wro$ and satisfying RA would be cyclic, so it wouldn't be a commit order. The essential difference between these two histories is the following: in $h_1$, $\tr(r)$ is $\so \cup \wro$-maximal while in $h_2$ $\tr(w)$ is not. As real database executions forbid transactions reading from non-committed ones, it is reasonable to allow those transactions $\so \cup \wro$-maximal to be executed completed without hindering the previous committed transactions. 

For a relation $R\subseteq A\times A$, the restriction of $R$ to $A'\times A'$, denoted by $R\downarrow A'\times A'$, is defined by $\{(a,b): (a,b)\in R, a,b\in A'\}$. Also, a set $A'$ is called $R$-downward closed when it contains $a\in A$ every time it contains some $b\in A$ with $(a,b)\in R$.

A \emph{prefix} of a transaction log $\tup{t,E, \po_t}$ is a transaction log $\tup{t,E', \po_t \downarrow E'\times E'}$ such that $E'$ is $\po_t$-downward closed. 
A \emph{prefix} of a history $\hist=\tup{T, \so, \wro}$ is a history $\hist'=\tup{T',\so\downarrow T'\times T',\wro\downarrow T'\times T'}$ such that every transaction log in $T'$ is a prefix of a different transaction log in $T$ but carrying the same id, and $\events{\hist'}\subseteq\events{\hist}$ is $(\po\cup \so \cup \wro)^*$-downward closed.

\begin{figure}[H]
	\centering
	\begin{subfigure}[b]{.25\textwidth}
		\resizebox{\textwidth}{!}{
			\begin{tikzpicture}[->,>=stealth',shorten >=1pt,auto,node distance=3cm,
				semithick, transform shape]
				\node[draw, rounded corners=2mm,outer sep=0] (t1) at (-3.25, 0) {\begin{tabular}{l} $\init$ \end{tabular}};
				\node[draw, rounded corners=2mm,outer sep=0] (t2) at (-3.25, -2) {\begin{tabular}{l} 
						$\rd{x}$ \\ $\rd{y}$
				\end{tabular}};
				\node[draw, rounded corners=2mm,outer sep=0] (t3) at (0, 0) {\begin{tabular}{l} 
						$\wrt{x}{2}$
				\end{tabular}};	
				\node[draw, rounded corners=2mm,outer sep=0] (t4) at (0, -2)
				{\begin{tabular}{l} 
						$\rd{x}$
				\end{tabular}};			
				
				\path (t1.south west) -- (t1.south) coordinate[pos=0.67] (t1x);
				\path (t2.north west) -- (t2.north) coordinate[pos=0.67] (t2x);
				\path (t3.north west) -- (t3.west) coordinate[pos=0.67] (t3x);
				
				\path (t3.west) edge [below] node[right] {$\wro_x$} (t2.north east);
				\path (t1.south) edge [above] node[left] {$\so \cap \wro_y$} (t2.north);
				\path (t1.east) edge [above] node[above] {$\so$} (t3.west);
				\path (t3.south) edge node[right] {$\wro_x$} (t4.north);
				\path (t1.south east) edge node[below] {$\so$} (t4.north west);
			\end{tikzpicture}
			
		}
		\caption{A history.}
		\label{fig:prefix:a}
	\end{subfigure}
	\hspace{.75cm}
	\centering
	\begin{subfigure}[b]{.25\textwidth}
		\resizebox{\textwidth}{!}{
			\begin{tikzpicture}[->,>=stealth',shorten >=1pt,auto,node distance=3cm,
				semithick, transform shape]
				\node[draw, rounded corners=2mm,outer sep=0] (t1) at (-3.25, 0) {\begin{tabular}{l} $\init$ \end{tabular}};
				\node[draw, rounded corners=2mm,outer sep=0] (t2) at (-3.25, -2) {\begin{tabular}{l} 
						$\rd{x}$ \\ $\rd{y}$
				\end{tabular}};
				\node[draw, rounded corners=2mm,outer sep=0] (t3) at (0, 0) {\begin{tabular}{l} 
						$\wrt{x}{2}$
				\end{tabular}};			
				
				\path (t1.south west) -- (t1.south) coordinate[pos=0.67] (t1x);
				\path (t2.north west) -- (t2.north) coordinate[pos=0.67] (t2x);
				\path (t3.north west) -- (t3.west) coordinate[pos=0.67] (t3x);
				
				\path (t3.west) edge [below] node[right] {$\wro_x$} (t2.north east);
				\path (t1.south) edge [above] node[left] {$\so \cap \wro_y$} (t2.north);
				\path (t1.east) edge [above] node[above] {$\so$} (t3.west);

			\end{tikzpicture}
		}
		\caption{A prefix of that history.}
		\label{fig:prefix:b}
	\end{subfigure}
	\hspace{.75cm}
	\centering
	\begin{subfigure}[b]{.25\textwidth}
		\resizebox{\textwidth}{!}{
			\begin{tikzpicture}[->,>=stealth',shorten >=1pt,auto,node distance=3cm,
				semithick, transform shape]
				\node[draw, rounded corners=2mm,outer sep=0] (t1) at (-3.25, 0) {\begin{tabular}{l} $\init$ \end{tabular}};
				\node[draw, rounded corners=2mm,outer sep=0, ] (t2) at (-3.25, -2) {\begin{tabular}{l} 
						$\rd{x}$ \\ $\rd{y}$
				\end{tabular}};
				\node[draw, rounded corners=2mm,outer sep=0, white] (t3) at (0, 0) {\begin{tabular}{l} 
					$\wrt{x}{2}$
				\end{tabular}};	
				\node[draw, rounded corners=2mm,outer sep=0] (t4) at (0, -2)
				{\begin{tabular}{l} 
						$\rd{x}$
				\end{tabular}};			
				
				\path (t1.south west) -- (t1.south) coordinate[pos=0.67] (t1x);
				\path (t2.north west) -- (t2.north) coordinate[pos=0.67] (t2x);
				\path (t3.north west) -- (t3.west) coordinate[pos=0.67] (t3x);
				
				\path (t1.south east) edge node[below] {$\so$} (t4.north west);
				\path (t3.west) edge [below] node[right] {$\wro_x$} (t2.north east);
				\path (t1.south) edge [above] node[left] {$\so \cap \wro_y$} (t2.north);
				\path (t3.south) edge node[right] {$\wro_x$} (t4.north);
			\end{tikzpicture}
			
		}
		\caption{A non-prefix.}
		\label{fig:prefix:c}
	\end{subfigure}
	
	\caption{A history and one of its prefixes. The third picture does not correspond to a prefix as some transactions read from the missing one.}
\end{figure}
%where $T'$ contains prefixes of a subset of transaction logs $T''\subseteq T$ such that $T''$ is $(\so \cup \wro)^*$-downward closed.  

\begin{definition}
An isolation level $I$ is called \emph{prefix-closed} when every prefix of an $I$-consistent history is also $I$-consistent.
\end{definition}

%For be denominated as \callout{STMC-model}, $\mathcal{M}$ has to satisfies the following three axioms for every program $\mathcal{P}$:

%Comparing to the model requirements described in \textcolor{red}{cite viktor's algorithm}, it is causal-extensibility property, a slightly stricter version of \textcolor{red}{Viktor}'s maximal-extensibility, the biggest difference. However, this weak formulation still forbids some axiomatic models such as Serializability \textcolor{red}{cite constantin's paper (SER)} \textcolor{red}{Appendix cite}.
%Indeed, some isolation levels cannot guarantee that reading from the last added $\iwrite$ event will maintain its consistency status.

\begin{theorem}
Every isolation level depicted in figure \ref{fig:consistency_defs} is prefix closed.
\end{theorem}
\begin{proof}
Let $h$ be a consistent history. As any $\so \cup \wro$-prefix-closed sub-history $h'$ of $h$ is a sub-graph of it, and there is a commit order $\co$ for $h$, it suffices to restrict $\co$ to $h'$ for obtaining a commit order for $h'$.
\end{proof}

Let  $\hist=\tup{T, \so, \wro}$ be a history. A transaction $t$ is called $(\so \cup \wro)^*$-maximal in $h$ if $h$ does not contain any transaction $t'$ such that $(t,t')\in (\so \cup \wro)^*$. We define a \emph{causal extension} of a pending transaction $t$ in $h$ with an event $e$ as follows:
\begin{itemize}
\item $e$ is added to $t$ as a maximal element of $\po_t$,
\item if $e$ is a read event, then $\wro$ is extended with some tuple $(t',e)$ such that $t'\ [\so \cup \wro]^*\ t$
\item the other elements of $\hist$ remain unchanged.
\end{itemize}
\textcolor{red}{TODO: MAXIMAL EXTENSIBILITY FOR PREFIX?}

TODO NOTE THAT A HISTORY MAY HAVE MULTIPLE CAUSAL EXTENSIONS WITH A READ. NOT FOR THE OTHER TYPES OF EVENTS. GIVE EXAMPLES.


\begin{definition}
An isolation level $I$ is called \emph{causally-extensible} if for every $I$-consistent history $h$, every $(\so \cup \wro)^*$-maximal pending transaction $t$ in $h$, and every event $e$, there exists a causal extension of $t$ with $e$ that is $I$-consistent.
%	can be extended with an event from a $\so \cup \wro$-maximal pending transaction $T$ in $h$. Moreover, if that event $r$ is a $\iread$, it can always read from a $\iwrite$ event $w$ s.t. in $h$ $\tr(w) \ [\so \cup \wro]^* \ \tr(r)$.
\end{definition}

\begin{theorem}
\label{theorem:causalExtensibleModels-CC-RA-RC}
Causal Consistency ($\CC$), Read Atomic ($\RA$) and Read Committed ($\RC$) are causally-extensible.
\end{theorem}
\begin{proof}
	%$\co$ a commit order that witness this property and $e$ a $\so \cup \wro$-maximal event.
Let $\mathcal{I}$ an isolation level in $\{\CC, \RA, \RC\}$. Let us show that for every non-total consistent history $h$, $\co$ a commit order that witness this property and $e$ a $\so \cup \wro$-maximal event there exists a causal extension of $h$ and $e$. For doing so, we will show that there exists a commit order $\co'$ for $h$ that is also a commit order for a causal extension of $h$. During the following, let us call $\varphi_{\CC}(h, t, e) = t \ [\so \cup \wro]^+ \ \trans{h}{e}$, $\varphi_{\RA}(h,t, e) = t \ [\so \cup \wro] \ \trans{h}{e}$ and $\varphi_{\RC}(h,t, e) = t \ [\wro \circ \po] \ e$; a sub-formula of the axiom that defines the corresponding isolation level.


On one hand, if $e$ is a $\ibegin$ event, let $\co' = \co \cup \{\langle t, \tr(e) \rangle, t \in h\}$ and if $e$ is either a $\iwrite$ or an $\iend$ event, let simply $\co' = \co$. On the other hand, if $e$ is a read event that reads $x$, let $t$ the $\co$-maximum transaction that writes $x$ and $(t, \trans{h}{e}) \in [\so \cup \wro]^+$; and let's pick $\co' = \co$. Either way, let us call $h' = h \oplus e$ or $h' = h \oplus \wro(t, e)$.


As $\co'\restriction_{h} = \co$, it is clearly a commit order for $h$. Moreover, if it does not witness $\mathcal{I}$-consistency for $h'$, there would exist $t_1, t_2, t_3$ transactions s.t. $t_2$ writes some variable $x$, $(t_1, t_3) \in \wro_x$ and $\varphi_{\mathcal{I}}(h, t_2, e)$ but $(t_1, t_2) \in \co'$. Firstly, as $h'$ is an extension of $h$ and $\mathcal{I}$ is prefix-closed, if $\trans{h}{e}$ is not equal to any of them, $\co$ would not witness $\mathcal{I}$-consistency for $h$. As $\trans{h}{e}$ is $\so \cup \wro$-maximal, $t_1 \neq \trans{h}{e} \neq t_2$. Moreover, if $e$ is not a $\iread$ event, $\trans{h}{e}$ cannot be $t_3$, as $t_1$, $t_2$ and $t_3$ would satisfy the same constraints in $h$, which is impossible. On the contrary, if $e$ is a $\iread$ event, $t$ cannot be $t_1$ as it is the $\co$-maximum history among those that write $x$ and $\trans{h}{e}$ causally depends on. In addition, $\trans{h}{e}$ cannot be $t_2$ as that would mean $\co$ does not witness $\mathcal{I}$-completeness. And analogously, as $\trans{h}{w}$ is neither of them, those transactions cannot exists, otherwise $h$ would not be $\mathcal{I}$-consistent. Succinctly, $h'$ is $\mathcal{I}$-consistent.
\end{proof}

\begin{theorem}
\label{theorem:maximally-extensible-PRE}
Prefix Consistency ($\PRE$) is maximally-extensible.
\end{theorem}
\begin{proof}
Let $h$ a non-total $\PRE$-consistent history, $\co$ a commit order that witness this property and $e$ a $\so \cup \wro$-maximal event. As if $e$ is the begin of a transaction, $\co' = \co \cup \{\langle t, \trans{h}{e} \rangle, t \in h\}$ is a witness of $h \oplus e$'s $\PRE$-consistency; let us show that if $\ibegin(\trans{h}{e}) \in h$, there is a commit order $\co'$ for $h$ where $\trans{h}{e}$ is maximal. This new commit order would allow us to prove that a extension is also $\mathcal{I}$-consistent.%If that would be the case, $h \oplus e$ will be consistent with $\co'$ as witness. \textcolor{red}{CUIDAO!}

Let $\co' = \{\langle t, t' \rangle \ | \ t \times t' \in h^2 \text{ s.t. } t \ [\co] \ t' \land t \neq \trans{h}{e}\} \cup \{\langle t, \trans{h}{e} \ | \ t \in h \rangle\}$. As $\co'$ is a total order between transactions, $\co'$ witness $h$ is $\PRE$-consistent if and only if there is no $t_1, t_2, t_3, t_4$ transactions such that $(t_1, t_3) \in \wro_x$, $t_2$ writes $x$, $(t_2, t_4) \in \co'^*$ and $(t_4, t_3) \in \so \cup \wro$ but $(t_1, t_2) \in \co'$. If $\trans{h}{e}$ is not either $t_1$, $t_2$, $t_3$ or $t_4$, as $\co'$ only modifies the relative order between $\trans{h}{e}$ and every other transaction, this situation cannot happen. On one hand, as $\trans{h}{e}$ is $\so \cup \wro$-maximal, $\trans{h}{e}$ cannot be $t_1$ nor $t_4$. On the other hand, as $\trans{h}{e}$ is $\co'$-maximum, it cannot be also $t_1$. Finally, if $t_3 = \trans{h}{e}$, then the exact situation would happen regarding $\co$, which is impossible. Therefore, $\co'$ witness $h$ is $\PRE$-consistent.

If $e$ is not a read event let $h' = h \oplus e$ and otherwise $h' = h \oplus \wro(t, e)$ for some transaction $t$ defined as in theorem \ref{theorem:causalExtensibleModels-CC-RA-RC}: the $\co$-maximum among those transactions that writes $x$ and that $\trans{h}{e}$ causally depends on. Then, once again by contrapositive, if $\co'$ does not witness $h'$'s $\PRE$-consistency, there would exist four transactions $t_1,t_2, t_3, t_4$ s.t. $(t_1, t_3) \in \wro_x$, $t_2$ writes $x$, $(t_2, t_4) \in \co'^*$ and $(t_4, t_3) \in \so \cup \wro$ but $(t_1, t_2) \in \co'$. Firstly, by prefix-closedness of $\PRE$, if $\trans{h}{e}$ is not equal to any of those four, $h$ would not be $\PRE$-consistent. But by $\co'$-maximality of $\trans{h}{e}$, t can only take the role of $t_4$. If $e$ is not a $\iread$ and $\trans{h}{e} = t_4$, by prefix-closedness we deduce $h$ is not $\PRE$-consistent; and if it is, by the same property, $\trans{h}{w}$ must take the role of one of the three remaining. By definition of $\trans{h}{w}$, it cannot be any other than $t_3$, as it is the $\co$-maximum among the transactions $\trans{h}{e}$ depends on; but that, by prefix-closedness, leads again that $h$ is $\PRE$-inconsistent. Hence, $h'$ must be $\PRE$-consistent.
\end{proof}

\begin{figure}[H]
	\centering
	\begin{subfigure}[b]{.23\textwidth}
		\begin{adjustbox}{max width=\textwidth}
			\begin{tabular}{c||c}
			\begin{lstlisting}[xleftmargin=5mm,basicstyle=\ttfamily\scriptsize,escapeinside={(*}{*)}, tabsize=1]
begin;
write((*$z$*),1);
a = read((*x*));
write((*$y$*),1);
commit
			\end{lstlisting} &
			\begin{lstlisting}[xleftmargin=5mm,basicstyle=\ttfamily\scriptsize,escapeinside={(*}{*)}, tabsize=1]
begin;
write((*$z$*),2);
b = read((*y*));
write((*$x$*),2);
commit
			\end{lstlisting} 
			\end{tabular} 
		\end{adjustbox}
		
		\caption{Program $1$.}
		\label{fig:non-causally-extensible:prog-ser}
	\end{subfigure}
	\hspace{.15cm}
	\centering
	\begin{subfigure}[b]{.23\textwidth}
		\resizebox{\textwidth}{!}{
			\begin{tikzpicture}[->,>=stealth',shorten >=1pt,auto,node distance=3cm,
				semithick, transform shape]
				\node[draw, rounded corners=2mm,outer sep=0] (t1) at (-1.5, -0.25) {\begin{tabular}{l} $\init$ \end{tabular}};
				\node[draw, rounded corners=2mm,outer sep=0] (t2) at (-3, -2) {\begin{tabular}{l} 
						$\wrt{z}{1}$ \\ $\rd{x}$ \\ $\wrt{y}{1}$ 
				\end{tabular}};
				\node[draw, rounded corners=2mm,outer sep=0] (t3) at (0, -2) {\begin{tabular}{l} 
						$\wrt{z}{2}$ \\ $\rd{y}$ \\ 	{\pgfsetfillopacity{0.3}$\wrt{x}{2}$}
				\end{tabular}};		
				
				\path (t1.south west) -- (t1.south) coordinate[pos=0.67] (t1sw);
				\path (t1.south east) -- (t1.south) coordinate[pos=0.67] (t1se);
				\path (t2.north east) -- (t2.north) coordinate[pos=0.67] (t2x);
				\path (t3.north west) -- (t3.north) coordinate[pos=0.67] (t3x);
				
				%\path (t3x) edge [above] node[yshift=8,xshift=0] {$\wro_x$} (t2.north east);
				\path (t1sw) edge [left] node {$\wro_x$} (t2x);
				\path (t1se) edge [right] node {$\wro_y$} (t3x);
			\end{tikzpicture}  
		}
		\caption{History non-maximally extensible under $\SER$ nor $\SI$.}
		\label{fig:non-causally-extensible:ser}
	\end{subfigure}
	\centering
\begin{subfigure}[b]{.23\textwidth}
\begin{adjustbox}{max width=\textwidth}
	\begin{tabular}{c||c}
		\begin{lstlisting}[xleftmargin=5mm,basicstyle=\ttfamily\scriptsize,escapeinside={(*}{*)}, tabsize=1]
begin;
a = read((*x*));
write((*$x$*),1);
commit
		\end{lstlisting} &
		\begin{lstlisting}[xleftmargin=5mm,basicstyle=\ttfamily\scriptsize,escapeinside={(*}{*)}, tabsize=1]
begin;
b = read((*y*));
write((*$y$*),2);
commit
		\end{lstlisting} 
		
		\\
		\multicolumn{1}{c}{} & \multicolumn{1}{c}{}
		\\
		\begin{lstlisting}[xleftmargin=5mm,basicstyle=\ttfamily\scriptsize,escapeinside={(*}{*)}, tabsize=1]
begin;
a = read((*y*));
write((*$y$*),1);
commit
			commit
		\end{lstlisting} &
		\begin{lstlisting}[xleftmargin=5mm,basicstyle=\ttfamily\scriptsize,escapeinside={(*}{*)}, tabsize=1]
begin;
b = read((*x*));
write((*$x$*),2);
commit
		\end{lstlisting}
	\end{tabular} 
\end{adjustbox}


	\caption{Program $2$.}
	\label{fig:non-causally-extensible:prog-pre}
\end{subfigure}
\hspace{.15cm}
\centering
\begin{subfigure}[b]{.23\textwidth}
	\resizebox{\textwidth}{!}{
		\begin{tikzpicture}[->,>=stealth',shorten >=1pt,auto,node distance=3cm,
			semithick, transform shape]
			
			%\\ \multicolumn{1}{c}{ \ldots}
			\node[draw, rounded corners=2mm,outer sep=0] (t0) at (0,0) {\begin{tabular}{l} $\init$ \end{tabular}};
			\node[draw, rounded corners=2mm,outer sep=0, label={[font=\small]50:$t_1$}] (t1) at (0, -2) {\begin{tabular}{l} 
					$\rd{x}$ \\ $\wrt{x}{1}$
			\end{tabular}};
			\node[draw, rounded corners=2mm,outer sep=0, label={[font=\small]50:$t_2$}] (t2) at (0, -4) {\begin{tabular}{l} 
					$\rd{y}$ \\ $\wrt{y}{2}$
			\end{tabular}};
			\node[draw, rounded corners=2mm,outer sep=0,  label={[font=\small]50:$t_3$}] (t3) at (3, -2) {\begin{tabular}{l} 
				$\rd{y}$ \\ $\wrt{y}{1}$
			\end{tabular}};	
		
			\node[draw, rounded corners=2mm,outer sep=0,  label={[font=\small]50:$t_4$}] (t4) at (3, -4) {\begin{tabular}{l} 
					\pgfsetfillopacity{0.3} $\rd{x}$ \\ $\wrt{x}{2}$
			\end{tabular}};		
			
			\path (t0.south west) -- (t0.south) coordinate[pos=0.67] (t0sw);
			\path (t0.south east) -- (t0.south) coordinate[pos=0.67] (t0se);
			\path (t1.north west) -- (t1.north) coordinate[pos=0.67] (t1nw);
			\path (t2.north west) -- (t2.north) coordinate[pos=0.67] (t2nw);
			\path (t2.north east) -- (t2.north) coordinate[pos=0.67] (t2ne);
			\path (t1.south west) -- (t1.south) coordinate[pos=0.67] (t1sw);
			\path (t1.south east) -- (t1.south) coordinate[pos=0.67] (t1se);
			\path (t3.south east) -- (t3.south) coordinate[pos=0.67] (t3se);
			\path (t3.north east) -- (t3.north) coordinate[pos=0.67] (t3ne);
			\path (t3.north west) -- (t3.north) coordinate[pos=0.67] (t3nw);
			\path (t3.south west) -- (t3.south) coordinate[pos=0.67] (t3sw);
			\path (t4.north east) -- (t4.north) coordinate[pos=0.67] (t4ne);
			\path (t4.north west) -- (t4.north) coordinate[pos=0.67] (t4nw);
			
			
			\path (t0.south east) edge [above] node {$\wro_y$} (t3.north);
			%\path (t1.south west) edge [bend right] node[above left] {$\wro_x$} (t4.north west);
			\path (t0.south) edge [left] node {$\wro_x$} (t1.north);
			%\path (t1sw) edge [left] node {$\so$} (t2nw);
			\path (t3.south) edge [left] node {$\so$} (t4.north);
			\path (t1.south) edge node [left] {$\so$} (t2.north);
			%\path (t4nw) edge [left] node {$\wro_x$} (t1se);
			
			\path (t0.south west) edge [bend right] node [above left]{$\wro_y$} (t2.north west);
			%\path (t2) edge [right] node {$\wro_x$} (t3);
		\end{tikzpicture}  
		
	}
	\caption{History non-causally extensible under $\PRE$.}
	\label{fig:non-causally-extensible:pre}
\end{subfigure}
	\caption{A bi-threaded program and a $\SER$/ $\SI$-consistent history joint with a four-threaded program and a $\PRE$-consistent history. Events in grey are not yet added to the history.}
	\label{fig:non-causally-extensible}
	%\vspace{-3mm}
\end{figure}

In Figure~\ref{fig:non-causally-extensible} we exhibit why $\PRE, \SI$ and $\SER$ are not causally extensible. In the first case, the program in Figure~\ref{fig:non-causally-extensible:prog-ser}, the history presented in Figure~\ref{fig:non-causally-extensible:ser} is $\SI$-consistent (and therefore $\SER$-consistent) but whose only extension (adding $\wrt{x}{2}$0 )is not maximally-extensible nor causally-extensible. In the second case, the program in Figure~\ref{fig:non-causally-extensible:prog-pre}, the history $h_2$ described in Figure~\ref{fig:non-causally-extensible:pre} is maximally-extensible ($h_2 \oplus \wro(t_1, \rd{x})$ is consistent) but no causally extensible, as ($h_2 \oplus \wro(\init, \rd{x})$ is inconsistent and $(t_1, t_4) \not\in \so\cup \wro$).



%\input{proofs:causal-extensibility}
