%!TEX root = main.tex
\section{Prefix-Closed and Causally-Extensible Isolation Levels}

TODO THE INTUITION BEHIND THIS CONDITION (USE THE TEXT THAT FOLLOWS)

Besides models presented in figure \ref{fig:consistency_defs}, others isolation levels exists in literature and real life applications \textcolor{red}{cite Constantin's papers + Twitter, shoppingcart...}. However, our algorithm can not be analyzed under an arbitrary model. We characterize in this section the ones that can be employed by our algorithm.
%not all of them can verified with the algorithm

\begin{figure}[H]
	
	\centering
	\begin{subfigure}[b]{.25\textwidth}
		\resizebox{\textwidth}{!}{
			\begin{tikzpicture}[->,>=stealth',shorten >=1pt,auto,node distance=3cm,
				semithick, transform shape]
				\node[draw, rounded corners=2mm,outer sep=0] (t1) at (-3.25, 0) {\begin{tabular}{l} $\wrt{x}{0}$ \\ $\wrt{y}{0}$ \end{tabular}};
				\node[draw, rounded corners=2mm,outer sep=0] (t2) at (-3.25, -2) {\begin{tabular}{l} 
						$a \gets \rd{x}$ \pgfsetfillopacity{0.3}\\ $b \gets \rd{y}$
				\end{tabular}};
				\node[draw, rounded corners=2mm,outer sep=0] (t3) at (0, 0) {\begin{tabular}{l} 
						$\wrt{x}{2}$%\\
						%\pgfsetfillopacity{0.3}$\wrt{y}{2}$
				\end{tabular}};			
				
				\path (t1.south west) -- (t1.south) coordinate[pos=0.67] (t1x);
				\path (t2.north west) -- (t2.north) coordinate[pos=0.67] (t2x);
				\path (t3.north west) -- (t3.west) coordinate[pos=0.67] (t3x);
				
				\path (t3x) edge [above] node[yshift=8,xshift=0] {$\wro_x$} (t2.north east);
				%\path (t1x) edge [right] node {$\wro_y$} (t2x);
			\end{tikzpicture}  
			
		}
		\caption{Extensible history.}
		\label{fig:maxclosed:a}
	\end{subfigure}
	\hspace{.75cm}
	\centering
	\begin{subfigure}[b]{.25\textwidth}
		\resizebox{\textwidth}{!}{
			\begin{tikzpicture}[->,>=stealth',shorten >=1pt,auto,node distance=3cm,
				semithick, transform shape]
				\node[draw, rounded corners=2mm,outer sep=0] (t1) at (-3.25, 0) {\begin{tabular}{l} $\wrt{x}{0}$ \\ $\wrt{y}{0}$ \end{tabular}};
				\node[draw, rounded corners=2mm,outer sep=0] (t2) at (-3.25, -2) {\begin{tabular}{l} 
						$a \gets \rd{x}$ \\ $b \gets \rd{y}$
				\end{tabular}};
				\node[draw, rounded corners=2mm,outer sep=0] (t3) at (0, 0) {\begin{tabular}{l} 
						$\wrt{x}{2}$\pgfsetfillopacity{0.3}\\
						$\wrt{y}{2}$
				\end{tabular}};			
				
				\path (t1.south west) -- (t1.south) coordinate[pos=0.67] (t1x);
				\path (t2.north west) -- (t2.north) coordinate[pos=0.67] (t2x);
				\path (t3.north west) -- (t3.west) coordinate[pos=0.67] (t3x);
				
				\path (t3x) edge [above] node[yshift=8,xshift=0] {$\wro_x$} (t2.north east);
				\path (t1x) edge [right] node {$\wro_y$} (t2x);
			\end{tikzpicture}
			
		}
		\caption{Non-extensible history.}
		\label{fig:maxclosed:b}
	\end{subfigure}
	\hspace{.175cm}
	\centering
	\caption{Example of a dead-lock after swapping two events.}
	\label{fig:maxclosed}
	%\vspace{-3mm}
\end{figure}

Let's analyze the histories $h_1$ and $h_2$ described in figure \ref{fig:maxclosed:a} and \ref{fig:maxclosed:b} respectively under RA; isolation level under which both are consistent. $h_1$ can be extended adding the event $r_1 = b \gets \rd{y}$ and the $\wro$-edge $w_1 \ [\wro] \ r_1$, where $w_1 = \wrt{x}{0}$. However, this is not the case of $h_2$: the only event that could be added in it is $w_2 = \wrt{y}{2}$. If $w_2$ would be added in $h_2$, any relation extending $\so \cup \wro$ and satisfying RA would be cyclic, so it wouldn't be a commit order. The essential difference between these two histories is the following: in $h_1$, $\tr(r)$ is $\so \cup \wro$-maximal while in $h_2$ $\tr(w)$ is not. As real database executions forbid transactions reading from non-committed ones, it is reasonable to allow those transactions $\so \cup \wro$-maximal to be executed completed without hindering the previous committed transactions. 

For a relation $R\subseteq A\times A$, the restriction of $R$ to $A'\times A'$, denoted by $R\downarrow A'\times A'$, is defined by $\{(a,b): (a,b)\in R, a,b\in A'\}$. Also, a set $A'$ is called $R$-downward closed when it contains $a\in A$ every time it contains some $b\in A$ with $a\leq b$.

A \emph{prefix} of $\hist=\tup{T, \so, \wro}$ is a history $\hist'=\tup{T',\so\downarrow T'\times T',\wro\downarrow T'\times T'}$ such that $T'$ is $(\so \cup \wro)$-downward closed.  TODO GIVE AN EXAMPLE OF A PREFIX ON THE FIGURE ABOVE.

\begin{definition}
An isolation level $I$ is called \emph{prefix-closed} when every prefix of an $I$-consistent history is also $I$-consistent.
\end{definition}

%For be denominated as \callout{STMC-model}, $\mathcal{M}$ has to satisfies the following three axioms for every program $\mathcal{P}$:

%Comparing to the model requirements described in \textcolor{red}{cite viktor's algorithm}, it is causal-extensibility property, a slightly stricter version of \textcolor{red}{Viktor}'s maximal-extensibility, the biggest difference. However, this weak formulation still forbids some axiomatic models such as Serializability \textcolor{red}{cite constantin's paper (SER)} \textcolor{red}{Appendix cite}.
%Indeed, some isolation levels cannot guarantee that reading from the last added $\iwrite$ event will maintain its consistency status.

\begin{theorem}
Every model depicted in figure \ref{fig:consistency_defs} is prefix closed.
\end{theorem}
\begin{proof}
Let $h$ be a consistent history. As any $\so \cup \wro$-prefix-closed sub-history $h'$ of $h$ is a sub-graph of it, and there is a commit order $\co$ for $h$, it suffices to restrict $\co$ to $h'$ for obtaining a commit order for $h'$.
\end{proof}

Let  $\hist=\tup{T, \so, \wro}$ be a history. A transaction $t$ is called $(\so \cup \wro)$-maximal in $h$ if $h$ does not contain any transaction $t'$ such that $(t,t')\in \so \cup \wro$. We define a \emph{causal extension} of a pending transaction $t$ in $h$ with an event $e$ as follows:
\begin{itemize}
\item $e$ is added to $t$ as a maximal element of $\po_t$,
\item if $e$ is a read event, then $\wro$ is extended with some tuple $(t',e)$ such that $t'\ [\so \cup \wro]^*\ t$
\item the other elements of $\hist$ remain unchanged.
\end{itemize}

TODO NOTE THAT A HISTORY MAY HAVE MULTIPLE CAUSAL EXTENSIONS WITH A READ. NOT FOR THE OTHER TYPES OF EVENTS. GIVE EXAMPLES.


\begin{definition}
An isolation level $I$ is called \emph{causally-extensible} if for every $I$-consistent history $h$, every $(\so \cup \wro)$-maximal pending transaction $t$ in $h$, and every event $e$, there exists a causal extension of $t$ with $e$ that is $I$-consistent.
%	can be extended with an event from a $\so \cup \wro$-maximal pending transaction $T$ in $h$. Moreover, if that event $r$ is a $\iread$, it can always read from a $\iwrite$ event $w$ s.t. in $h$ $\tr(w) \ [\so \cup \wro]^* \ \tr(r)$.
\end{definition}

\begin{theorem}
\label{theorem:causalExtensibleModels-CC-RA-RC}
Causal Consistency ($\CC$), Read Atomic ($\RA$) and Read Committed ($\RC$) are causally-extensible.
\end{theorem}
\begin{proof}
Let $\mathcal{M}$ a model in $\{\CC, \RA, \RC\}$, $h$ a non-total consistent history and let $e$ a $\so \cup \wro$-maximal event. If $e$ is a $\ibegin$ event, $h \bullet e$ is consistent as if there exists a commit order $\co$ for $h$, the relation $\co' = \co \cup \{\langle T, \tr(e) \rangle, T \in h\}$ is a commit order to $h'$. Moreover, if $e$ is either a $\iwrite$ or an $\iend$ event, $h \bullet e$ is edge-wise identical to $h$, so the commit order for $h$ is also a valid commit order for $h \bullet e$. Therefore, let $e$ a $\so \cup \wro$-maximal $\iread$ event that reads variable $x$ and let us find a $\iwrite$ event $w$ s.t. $\tr(w) \ [\so \cup \wro]^* \ \tr(r)$ and that $h'_w = h \bullet_w r$ is consistent. 

For doing so, we will do an induction on the number of $\co$ cycles $h'_w$ has, for some event $w$ s.t. $\tr(w) \ [\so \cup \wro]^* \ \tr(r)$; where $\co$  Clearly, if $h'_w$ is acyclic, by theorem \textcolor{red}{cite theorem h acyclic => exists a co (another paper, I hope it exists somewhere)}, it is consistent. Hence, let's suppose that if $h'_w$ has at most $n$ cycles, there exists another $\iwrite$ event $w_n$ s.t. $r$ causally depends on and $h'_{w_n} = h \bullet_{w_n} r$ is consistent; and let's analyze if the same property can be deduced for a history $h'_{w_{n+1}} = h \bullet_{w_{n+1}} r$ with $n+1$ cycles. As $h$ is consistent and $\tr(w_{n+1}) \ [\so \cup \wro]^* \ \tr(r)$, if there were a cycle, it would be due to a transaction $T$ such that writes $x$, $\tr(w_{n+1}) \ [\co]^* \ T$ and $\varphi_{\mathcal{M}}(T, e)$, where $\varphi_{\CC}(T, e) = T \ [\so \cup \wro]^+ \ \tr(e)$, $\varphi_{\RA}(T, e) = T \ [\so \cup \wro] \ \tr(e)$ and $\varphi_{\RC}(T, e) = T \ [\wro \circ \po] \ e$. In particular, for any of the three models, $T \ [\so \cup \wro]^* \ \tr(r)$. Let $w_n$ a $\iwrite$ event in $T$ that writes $x$ and $h_{w_n} = h \bullet_w r$: if we prove that the number of $\co$-cycles in $h_{w_n}$ is strictly smaller than in $h_{w_{n+1}}$ by induction hypothesis, we can conclude the result.

Firstly, as $T\ [\wro_x] \ \tr(r)$, the cycle that was between $T$ and $\tr(w_{n+1})$ does not exist in $h_{w_n}$. Moreover, analogously as before, if there is a cycle in $h_{w_n}$, it is due to the existence of a transaction $T'$ s.t. $T \ [\co] \ T'$ and $\varphi_{\mathcal{M}}(T', r)$. Therefore, in $h_{w_{n+1}}$ $\tr(w_{n+1}) \ [\co] \ T \ [\co] \ T'$, so there is a cycle between in $\tr(w_{n+1})$ and $T'$. To sum up, every cycle in $h_{w_n}$ has a counterpart in $h_{w_{n+1}}$ and it has one less cycle; so the inductive step holds.

Finally, as $\init$ contains a $\iwrite$ event $w_\emptyset$ that writes $x$, $\init \ [\so \cup \wro]^* \ \tr(r)$ and every history has a finite number of transactions, we can deduce from the history $h'_{w_\emptyset} = h \bullet_{w_\emptyset} r$ that there is a $\iwrite$ event $w$ that $r$ depends causally on and $h \bullet_w r$ is consistent.
\end{proof}

\begin{theorem}
\label{theorem:causalExtensibleModels-PRE}
Prefix Consistency ($\PRE$) is causally-extensible.
\end{theorem}
\begin{proof}
Let $h$ a non-total consistent history and $e$ a $\so \cup \wro$-maximal event. Analogously as in theorem \ref{theorem:causalExtensibleModels-CC-RA-RC}, if $e$ is a $\ibegin$ or an $\iend$ event, $h \bullet e$ is consistent. If $e$ is a $\iwrite$ event that writes a variable $x$, let us show that for $h' = h \bullet e$ we can produce a commit order for it starting from a commit order $\co$ for $h$. If there is a cycle in $h'$ it is because there exists some transactions $T_0, T_1, T_2$ s.t. $T_1 \ [\wro_x] \ T_2$, $T_0 \ [\so \cup \wro] T_2$, $\tr(e) \ [\co]^* \ T_0$ and $T_1 \ [\co] \ \tr(e)$. As $e$ is $\so \cup \wro$-maximal, for every other transaction $T$, $\lnot(\tr(e) \ [\so \cup \wro]^* \ T)$. Let $\co' = \{\langle T, T' \rangle \ | \ T \times T' \in h'^2 \text{ s.t. } T \ [\co] \ T' \land T \neq \tr(e)\} \cup \{\langle T, \tr(e) \ | \ T \in h' \rangle\}$. This relation is a total order as it is $\co\restriction_{h \setminus \tr(e)}$ juxtaposed with $\tr(e)$ which extends $\so \cup \wro$. Moreover, if $h'$ had a $\co'$-cycle, as $\tr(e)$ is $\so \cup \wro$-maximal, it would have to be due to other four transactions distinct from $\tr(e)$; contradicting $h$ is consistent. Therefore, $h'$ is a $\PRE$-consistent history.

Otherwise, if $e$ is a $\iread$ event that reads variable $x$, let $h_{R|W}$ the history obtained by splitting every transaction $T$ in two, one containing every $\iread$ event ($R_T$) that immediately $\so$-precedes a complementary one with every single $\iwrite$ event ($W_T$); along with corresponding $\so_{R|W}$ and $\wro_{R|W}$ relations \textcolor{red}{Cite Ranadeep's paper}. By theorem \textcolor{red}{Cite again Ranadeep's paper}, for every history $\hat{h}$, $\isConsistent[\PRE]{\hat{h}}$ if and only if $\isConsistent[\SER]{\hat{h}_{R|W}}$. Therefore, as $\isConsistent[\SER]{h_{R|W}}$ holds, there is a commit order $\co'$ such that for every $T$ s.t. if $W_{T} \ [\co'] \ R_{\tr(e)}$ them $T \ [\so \cup \wro]^+ \tr(e)$. Therefore, let $w$ the maximum $\iwrite$ event according to $\co'$ before $R_{\tr(e)}$ that writes $x$ and $h'_{R|W} = h_{R|W} \bullet_w e$. This history is clearly serializable taking $\co'$ as commit order. Hence, $h' = h \bullet_w e$ is $\PRE$-consistent and $\tr(w) \ [\so \cup \wro]^* \tr(r)$.
% and in particular, $\lnot(\tr(e) \ [\so \cup \wro]^* \ T)$
\end{proof}

TODO SHOW THAT SERIALIZABILITY IS NOT CAUSALLY-EXTENSIBLE.

%\input{proofs:causal-extensibility}
