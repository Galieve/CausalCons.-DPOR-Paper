\subsubsection{Oracle-respectful histories} $ $\\

The second step in this proof is characterize all reachable histories with some general invariant that can be generalized to every total history. For doing so, we will show that for reachable histories any history order coincide with its canonical order; so any property based on a history order can be generalized to be based on its canonical order.
\textcolor{red}{I know events may ``disappear'' in an execution and maybe they even have no meaning but I need to reason globally, thinking about the future execution in some way... }

\textcolor{red}{TODO: assume for the moment that every event will appear (no ifs) and then see how can we express this definition for the more general case.}

\begin{definition}
\label{def:oracle-respectful}
An ordered history $(h, \leq)$ is \callout{$\ora$-respectful} with respect to $\leq$ if it has at most one pending transaction log and for every pair of events $e \in \prog, e' \in h$ s.t. $e \leq_{\ora} e'$, either $e \leq e'$ or $\exists e'' \in h, \trans{h}{e''} \leq_{\ora} \trans{h}{e}$ s.t. $\trans{h}{e'} \ [\so \cup \wro]^* \ \trans{h}{e''}$, $e'' \leq e$ and $\swapped{h}{e''}$; where if $e \not\in h$ we state $e' \leq e$ always hold but $e \leq e'$ never does. We will denote it by $\oraRespectful{h}{\leq}$. %\textcolor{olive}{REVISAR EL CAMBIO DEL $e'' < e$ -> $T'' < T$.}
\end{definition}

\textcolor{red}{I definitely think transactions shouldn't have histories as an operator, make things confusing in this proof.}

\begin{lemma}
\label{lemma:reachable-or-respectful}
Let $p$ a computable path. Every ordered history in $p$ is $\ora$-respectful with respect to $\leq_h$.
\end{lemma}
\begin{proof}
	
We will prove this property by induction on the number of histories this path has. The base case, the empty path, trivially holds; so let us prove the inductive case: for every path of at most length $n$ the property holds. Let $p$ a path of length $n+1$ and $h_<$ the last reachable history of this path. As $p \setminus \{h\}$ is a computable path of length $n$, the immediate predecessor of $h$ in $p$, $(h_p,{<_{h_p}})$ is $\ora$-respectful with respect to $<_p$. Let $a = \nextEvent(h_p)$.

Firstly, if $a$ is not a $\iread$ nor a $\ibegin$ event and $h = h_p \oplus a$, as $\leq_h$ is an extension of $\leq_{h_p}$, $a$ belongs to the only pending transaction and $\ora$ orders transactions completely, we can deduce that $h$ is $\ora$-respectful with respect to $\leq$. 

In addition, if $a$ is a $\ibegin$ event and $h = h_p \oplus a$, let $e \in \prog, e' \in h$ s.t. $e <_{\ora} e'$. If $e \in h_p$ or $e' \neq a$, as $\leq_{h}$ is an extension of $\leq_{h_p}$ and $\oraRespectful{h_p}{\leq_{h_p}}$ holds, $\oraRespectful{h}{\leq}$ also does it. Moreover, as $a = \min_{\ora} \prog \setminus h_p$, there is no event $a \in \prog \setminus h_p$ s.t. $e \leq_{\ora} a$; so $\oraRespectful{h}{\leq}$ holds.

Moreover, if $a$ is a $\iread$ event and $h = h_p \oplus \wro(a, t)$ for some transaction log $t$, let us call $e \in \prog, e' \in h$ s.t. $e <_{\ora} e'$. Once again, if $e \in h$ or $e' \neq a$ the property holds; so let's suppose $e \in \prog \setminus h_p$ and $e' = a$. Let $b = \ibegin(\trans{h}{a})$, that also belongs to $h_p$. Firstly, as $\trans{h}{e} \leq_{\ora} \trans{h}{e'} = \trans{h}{b}$ we know that $e \leq_{\ora} b$. Secondly, as $\oraRespectful{h_p}{\leq_{h_p}}$, $e \not\in h_p$ and $e \leq_{\ora} b$; there exists $c \in h_p$, $\trans{h_p}{c} \leq_{\ora} \trans{h_p}{a}$ s.t. $\trans{h_p}{b} \ [\so \cup \wro]^* \ \trans{h_p}{c}$, $c \leq b$ and $\swapped{(h_p,{<_{h_p}})}{c}$. As $\trans{h}{a} = \trans{h}{b}$ and $\swapped{(h_p,{<_{h_p}})}{c}$ implies $\swapped{h_<}{c}$, we conclude $\oraRespectful{h}{\leq}$.

But if no previous case is satisfied, it is because $h = \swap((h_p,{<_{h_p}}), r, t)$ for some $r, t \in h_p$ s.t. $\genericProtocol((h_p,{<_{h_p}}), r, t)$ holds. Let $e, e'$ two events s.t. $e \leq_{\ora} e'$. On one hand, if $e \leq e'$, $\oraRespectful{h}{e}$ holds. On the other hand, if $e' < e$ and $e' \leq_{h_p} e$, as $\oraRespectful{h_p}{\leq_{h_p}}$ holds and no swapped event is deleted by $\genericProtocol((h_p,{<_{h_p}}), r, t)$'s definition, the property is also satisfied. Finally, if $e < e'$ and $e \leq_{h_p} e'$, $e$ has to be a deleted event so $e \in \prog \setminus h$. As $r \leq_{h_p} e$, if $e \leq_{\ora} a$, there would exist a $c \in h_p$, $\trans{h_p}{c} \leq_{\ora} \trans{h_p}{e} \leq_{\ora} \trans{h_p}{r}$ s.t. $\trans{h_p}{r} \ [\so \cup \wro]^* \ \trans{h_p}{c}$ and $\swapped{h_<}{c}$. However, this contradicts $\genericProtocol((h_p,{<_{h_p}}), r, t)$ as $a$ must be in the last transaction of $h$, that is $(\so \cup \wro)^+$-maximal; so $e \leq_{\ora} a$. Taking $e'' = r$ the property is witnessed. 
\end{proof}

\begin{proposition}
	\label{proposition:orders-coincide}
	For any reachable history $h$, $\leq^h \equiv \leq_h$.
\end{proposition}
\begin{proof}
For proving this equivalence, we will show that in any computable path and for any ordered history $h$, if $t \leq_h t'$, then $t \leq^h t'$, as by lemma \ref{lemma:canonincal-total-order} $\leq^h$ is a total order and therefore they have to coincide. We will prove this by induction on the number of histories a path has. The base case, the empty path, trivially holds; so let us prove the inductive case: for every path of at most length $n$ the property holds. Let $p$ a path of length $n+1$ and $h_{<_h}$ the last reachable ordered history of this path. As $p \setminus \{h\}$ is a computable path of length $n$, the immediate predecessor of $h$ in $p$, $\leq^{h_p} \equiv \leq_{h_p}$. Let $e = \nextEvent(h_p)$. Firstly, let's note that if $h$ is an extension of $h_p$, as $\oraRespectful{h_p}{<_{h_p}}$, the property can only fail while comparing a transaction $t$ with $\trans{h}{e}$.
	\begin{itemize}
		\item \underline{$h$ extends $h_p$ and $e$ is a $\ibegin$:} As $\dep(h_p, t, \bot) = \dep(h, t, \bot)$ for every transaction in $h_p$, if $t \leq^{h_p} t'$, then $t \leq^h t'$. Moreover, $\dep(h, \trans{h}{e}, \bot) = \{e\}= \min_{\ora} \prog \setminus h_p$. By lemma \ref{lemma:reachable-or-respectful} $h$ is $\ora$-respectful, so for every $t$, $\min_{\ora} \dep(h, t, \bot) <_{\ora} e$; which implies $t <^h \trans{h}{e}$. By lemma \ref{lemma:canonincal-total-order}, $\leq^h$ is a total order, so it coincides with $\leq_h$.
		
		\item \underline{$h$ extends $h_p$ and $e$ is not a $\ibegin$:} As no transaction depends on $\trans{h}{e}$ and $\trans{h}{e} = \last{h_p}$, if we prove that for every pair of transactions $\minimalDependency(h_p, t', t'', \bot) $ $= \minimalDependency(h, t', t'', \bot)$, the lemma would hold. On one hand, $\dep(h, \trans{h}{e}, \bot) = \dep(h_p, \trans{h}{e}, \bot) = \trans{h}{e}$ and in the other hand, by lemma \ref{lemma:reachable-or-respectful}, $\min_{\ora} \dep(h_p, t, \bot) <_{\ora} \trans{h}{e}$. Finally, as $e \not\in \dep(h, \hat{t}, e')$, for every $\hat{t} \neq \trans{h}{e}, e' \neq \bot$, for every pair of transactions $t', t''$, $\minimalDependency(h_p, t', t'' \bot) = \minimalDependency(h, t', t'', \bot)$. 
		
		\item \underline{$h = \swap(h_p, r, t)$, where $t = \trans{h}{e}$:} As $\genericProtocol(h_p, r, t)$ is satisfied and $h$ is $\ora$-respectful, for every event $e'$ and transaction $t'$, $\min_{\ora} \dep(h_p, t', e') = \min_{\ora} \dep(h, t', e')$, so for every pair of transactions $\minimalDependency(h_p, t', t'', \bot) = \minimalDependency(h, t', t'', \bot)$. In particular, this implies $t' \leq^{h_p} t''$ if and only if $t' \leq^h t''$ for every  pair $t', t'' \in h$ and $t' \leq^h \trans{h}{r}$ for every $t' \in h$; so $\leq^h \equiv \leq_h$. 
	\end{itemize}
\end{proof}


Proposition \ref{proposition:orders-coincide} is a very interesting result as it express the following fact: regardless of the computable path that leads to a history, the final order between events will be the same. Therefore, all possible history orders collapse to one, the canonical one. This result will have a key role during both completeness and optimality, as it restricts the possible histories that precede another while describing the computable path leading to it. In addition, proposition \ref{proposition:orders-coincide} together with lemma \ref{lemma:reachable-or-respectful} justify enlarging definition \ref{def:oracle-respectful} with a general order as for reachable histories, $\oraRespectful{h}{\leq_h}$ is equivalent to $\oraRespectful{h}{\leq^h}$. From what follows, we will simply state $h$ is $\ora$-respectful and we will denote it by $\oraRespectfulCanon{h}$. Moreover, we will assume every history is ordered with the canonical order.

%As $\leq^h \equiv \leq_h$ for any reachable history, we will extend $R^{\ora}(h)$ to any history changing $\leq_h$ to $\leq^h$ in \ref{def:oracle-respectful} whenever it is needed. This property is not something reachable histories satisfy but also, as next lemma shows, total histories with $\leq^h$ order do; which justify it as an useful tool for proving completeness.
\begin{lemma}
	\label{lemma:total-respectful}
	Any total history is $\ora$-respectful.
	\begin{proof}
		Let $h$ be a total history and $t, t'$ a pair of transactions s.t. $t \leq_{\ora} t'$. If $t \leq^h t'$, then the statement is satisfied; so let's assume the contrary: $t' \leq^h t$. If $t' \ [\so \cup \wro]^* \ t$, then for every $e \in t, e' \in t'$ $\exists c \in h$ s.t. $\trans{h}{c} \leq_{\ora} \trans{h}{e}$, $\trans{h}{e'} \ [\so \cup \wro]^* \tr(c)$, $\swapped{h}{c}$ and $c \leq^h e$; so the property is satisfied. Otherwise, by definition of $\minimalDependency$, there exists $r' \in h$ s.t. $t' \ [\so \cup \wro]^* \ \trans{h}{r'}$ and $\trans{h}{r'} \leq_{\ora} t$. Moreover, by \textsc{canonicalOrder}'s definition, $\trans{h}{r} \leq^h t$. Finally $\swapped{h}{r'}$ holds as it is the minimum element according $\ora$. To sum up, $\oraRespectfulCanon{h}$ holds.
	\end{proof}
\end{lemma}