%!TEX root = main.tex
\section{Proof of Theorem~\ref{th:corr}}
\label{sec:proofs-algorithm}

As explained in Section~\ref{ssec:corr}, $I$-soundness, the polynomial space bound, and the part of strong completeness that refers to not engaging in fruitless explorations follow directly from definitions. In the following, we focus on $I$-completeness and then optimality. For the sake of the proof's readability, we will omit all local states of the algorithm's definition during the proof. Therefore, we consider programs where we can describe all their events.

%\textcolor{red}{Introduction}
%
%
%\textcolor{red}{The proofs are written when no local state was added, as otherwise I have to assign to every history a local state (randomness is allowed? in that case not one but a set of local states...)}
%%As we are going to prove the completeness of algorithm \ref{algorithm:stmc}, the first property we want to prove is its equivalence with its recursive counterpart, \ref{algorithm:stmc2}. Here, line \ref{algorithm:stmc:inner_loop} refers to 
%\subsection{Soundness}

\begin{lemma}
\label{lemma:soundness-swap}
For any reachable history $h$, if $h' = \swap(h_p, r, t)$ for some $r, t \in h$, then $\swapped{h'}{r}$ holds.
\end{lemma}
\begin{proof}
To prove this lemma we are simply checking every condition stated in the function. As in $h'$ $(r, t) \in \wro$, $t < r$. \textcolor{red}{Incomplete, seems I have to prove this at the same time as $\ora$-respectfulness... (lemma \ref{lemma:reachable-or-respectful} )}
Finally, as every $\iread$ event $e \in h' \setminus \{r\}$ reads from the same transaction in both $h$ and $h'$, we conclude $\swapped{h'}{r}$ holds.
\end{proof}

\begin{theorem}
Algorithm \ref{algorithm:optimal-instantiated} is sound.
\begin{proof}
We want to prove that any computable path is formed by $\mathcal{I}$-sound histories. For doing so, we will do an induction on the number of steps this path has. The base case, when the path is empty, is due to the initial state. There, the history has to be $\emptyset$, which is $\mathcal{I}$-consistent. Thus, let's prove the inductive step, assuming that in any computable path of length at most $n$ every history is consistent. Let be $p$ a computable path of length $n+1$ and $h$ its last history. 

As $p \setminus \{h\}$ is a computable path of length $n$, every history is consistent; so in particular the immediate predecessor of $h$ in $p$: $h_p$. Let $a = \nextEvent(h_p)$, which is always a $\so \cup \wro$-maximal event. If $a$ is not a $\iread$ event and $h = h_p \oplus e$, by definition \ref{def:causally-extensible} we know $h$ is consistent. Analogously, if $a$ is a $\iread$ event and $h = h_p \oplus \wro(r, t)$ for some $t \in h_p$, by $\validWrites$'s definition we know $h$ is consistent. Finally, if $h = \swap(h_p, r, t)$ for some event $r$ and transaction $t$, as $\protocol(h_p, r, t)$ is satisfied, $h$ is consistent.
\end{proof}
\end{theorem}


\subsection{Completeness}

In our algorithm's context, completeness means being able to compute every total history. However, our algorithm works with an extended version of histories where its events are totally ordered. For proving this property, we will need to furnish every history with a total order that coincides with the algorithm's one. This order is given by the canonical order function presented below.

%\textcolor{red}{Problem of this proofs: I assume optimality before proving it! \sout{ I don't know what wording we shall apply} (I tried to modify it, let's see if it is noticeable).}

\begin{algorithm}[H]
	\caption{\textsc{Canonical order}}
	\begin{algorithmic}[1]
		
		\Statex
		\Procedure{\textsc{canonicalOrder}}{$h, T, T'$}
		\State \Return $T \ [\so \cup \wro]^* \ T' \lor$
		\State \qquad $(\lnot(\tr(T') \ [\so \cup \wro]^* \ T) \land \minimalDependency(h, T, T', \bot)$
		\EndProcedure
		
		\Statex
		\Procedure{\minimalDependency}{$h, T, T', e$}
		\Let $a = \min_{<_{\ora}} \dep(h, T, e)$; $a' = \min_{<_{\ora}} \dep(h, T', e)$
		\If{$a \neq a'$}
		\State \Return $a <_{\ora} a'$
		\Else
		\State \Return $\minimalDependency(h, T, T', a)$
		\EndIf
		\EndProcedure
		
		\Statex
		\Procedure{\dep}{$h, T, e$}
		\State \Return $\{r \ | \exists w \text{ s.t. } T \ [\so \cup \wro]^* \ \tr(w) \land \ w \ [\wro] \ r \land \tr(r) \ [\so \cup \wro]^+ \tr(e) \} \cup T$ 
		\EndProcedure	
	\end{algorithmic}
	\label{algorithm:canonical-order}
	%\caption{Generic method for exploring every possible history $h$ of a program $\mathcal{P}$ running under a database with $\mathcal{M}$ as isolation level.}
\end{algorithm}

The function $\textsc{canonicalOrder}$ produces a relation between transactions in a history, denoted $\leq^h$. In algorithm \ref{algorithm:canonical-order}'s description, we denote $\bot$ as the end of the program, which always exists, and that is $\so$-related with every single transaction.

\begin{lemma}
	\label{lemma:dep_shrinks}
	For every history $h$, event $e$ and transaction $T$, $\dep(h, T, \min_{<_{\ora}} \dep(h, T, e)) \subseteq \dep(h, T, e)$. Moreover, if $\dep(h, T, e) \neq T$, the inclusion is strict.
	\begin{proof}
		Let $r' = \min_{<_{\ora}} \dep(h, T, e)$ and $r \in \dep(h, T,r')$. Then, $\exists w \text{ s.t. } T \ [\so \cup \wro]^* \ \tr(w) \land \ w \ [\wro] \ r \land \tr(r) \ [\so \cup \wro]^+ \tr(r')$ and $\exists w' \text{ s.t. } T \ [\so \cup \wro]^* \ \tr(w') \land \ w' \ [\wro] \ r' \land \tr(r') \ [\so \cup \wro]^+ \tr(e)$; so $\tr(r) \ [\so \cup \wro]^+ \tr(r') \ [\so \cup \wro]^+ \tr(e)$. In other words, $r \in \dep(h, T, e)$. The moreover comes trivially as $r' \not\in \dep(h, T, r')$.
	\end{proof}
\end{lemma}

\begin{lemma}
	\label{lemma:minimalDependency-halts}
	For every distinct $T, T'$, $\minimalDependency(h,T,T',e)$ always halts.
	\begin{proof}
		As $h$ is a finite history, every transaction $T$ belongs to $\dep(h, T,e)$, regardless of the event $e$ and via lemma \ref{lemma:dep_shrinks} the set $\dep$ shrinks in each recursive call; we conclude that if $T \neq T'$, there would be a call of $\minimalDependency$ and an event $e$ associated with it s.t. $\min_{<_{\ora}}  \dep(h, T, e) \neq \min_{<_{\ora}}  \dep(h, T', e)$.
	\end{proof}
\end{lemma}

\begin{lemma}
	\label{lemma:canonincal-total-order}
	The relation $\leq^h$ is a total order.
	\begin{proof}
		$ $\\
		\begin{itemize}
			\item \underline{Reflexivity:} By definition, for every $T$, $T \leq^h T$.
			\item \underline{Transitivity:} Let's suppose $a \leq^h b $ and $b \leq^h c$. First, take into account that if $c \neq a$, $\lnot(c \ [\so \cup \wro]^* \ a)$ Here we distinguish four cases:
			\begin{itemize}
				\item If $a \ [\so \cup \wro]^* \ b$ and $b \ [\so \cup \wro]^* \ c$, then $a \ [\so \cup \wro]^* \ c$, so $a \leq^h c$.
				\item If $a \ [\so \cup \wro]^* \ b$ but $\lnot(b \ [\so \cup \wro]^* \ c)$, then for every $e\in h$, $\min_{<_{\ora}} \dep(a, e) \leq_{\ora} \min_{<_{\ora}} \dep(b, e)$, so $a <^h c$.
				\item If $\lnot(a \ [\so \cup \wro]^* \ b)$ but $b \ [\so \cup \wro]^* \ c$, then for every $e\in h$, $\min_{<_{\ora}} \dep(b, e) \leq_{\ora} \min_{<_{\ora}} \dep(c, e)$, so $a <^h c$.
				\item If $\lnot(a \ [\so \cup \wro]^* \ b)$ and $\lnot(b \ [\so \cup \wro]^* \ c)$, then it can be proven by induction that $a <^h c$. \textcolor{red}{It has to be proven iterating on the call function mininalDependency, a bit boring}
				%there exists and $e\in h$ s.t. $\min_{<_{\ora}} \dep(a, e) <_{\ora} \min_{<_{\ora}} \dep(b, e)$ or $\min_{<_{\ora}} \dep(b, e) <_{\ora} \min_{<_{\ora}} \dep(c, e)$, 
			\end{itemize}
			\item \underline{Antisymmetric} For every $a, b$ s.t. $a \leq^h b$ and $b \leq^h a$. If $a \ [\so \cup \wro]^* b$, then $a = b$. If not, then $\minimalDependency(h,a,b,\bot)$ and $\minimalDependency(h,b,a,\bot)$ cannot be satisfied at the same time. \textcolor{red}{Again an induction on $\minimalDependency$ along with the history's finiteness.}
			\item \underline{Strongly connection} Let $a, b$ s.t. $a \not \leq_{\ora} b$. If $b \ [\so \cup \wro]^* a$, then $b \leq_{\ora} a$. Otherwise, as $\lnot(a \ [\so \cup \wro]^* \ b)$ and $\minimalDependency$ halts (lemma \ref{lemma:minimalDependency-halts}) and $\lnot \minimalDependency(h, a, b, e)$, then $\minimalDependency(h,b, a, e)$; so $b <^h a$.
		\end{itemize}
	\end{proof}
\end{lemma}

\begin{definition}
	\label{def:oracle-respectful}
	A reachable history $h$ is \callout{$\ora$-respectful} if it has at most one pending transaction and for every pair of events $e \in \mathcal{P}, e' \in h$ s.t. $e \leq_{\ora} e'$, either $e \leq_h e'$ or $\exists e'' \in h, \tr(e'') \leq_{\ora} \tr(e)$ s.t. $\tr(e') \ [\so \cup \wro]^* \ \tr(e'')$, $e'' \leq_h e$ and $\swapped{h}{e''}$; where if $e \not\in h$ we state $e' \leq_h e$ always hold but $e \leq_h e'$ never does. We will denote it by $R^{\ora}(h)$. %\textcolor{olive}{REVISAR EL CAMBIO DEL $e'' < e$ -> $T'' < T$.}
\end{definition}

\begin{lemma}
	\label{lemma:reachable-or-respectful}
	Every reachable history is $\ora$-respectful.
	\begin{proof}
		We will prove it by induction on the number of \textcolor{red}{cite algorithm}'s stack calls a computable path that leads to a history $h$ needs, $n$. The base case, $n = 0$, is for the trivial history $h = \emptyset$ where it trivially holds; so let us prove the inductive case; being $e \gets \nextEvent(h)$. On one hand, $e$ is not a $\iread$ nor a $\ibegin$ event and $h' = h \bullet e$, as $\lnot \swapped{h}{e}$ and $h'$ is edge-wise identical to $h$, $R^{\ora}(h')$ holds. 
		
		If $e$ is a $\ibegin$ event, $h' = h \bullet e$. Let $a \in \mathcal{P}, b \in h'$ s.t. $a \leq_{\ora} b$. If $a \in h$ or $b \neq e$, as $\leq_{h'}$ is an extension of $\leq_h$ and $R^{\ora}(h)$, the property holds. Moreover, as $e = \min_{\ora} \mathcal{P} \setminus h$, there is no event $a \in \mathcal{P} \setminus h$ s.t. $a \leq_{\ora} e$; so the property holds.
		
		
		On the other hand, if $e$ is a $\iread$ event and $w$ is a $\iwrite$ one, let us prove that $h' = h \bullet_w e$. Let $a \in \mathcal{P}, b \in h'$ s.t. $a \leq_{\ora} b$. Once again, if $a \in h$ or $b \neq e$ the property holds; so let's suppose $a \in \mathcal{P} \setminus h$ and $b = e$. Let $d = \ibegin(\tr(e))$, $d \in h$. As $R^{\ora}(h)$ and $a \not\in h$, $a \leq_{\ora} d$; so there exists $c \in h$, $\tr(c) \leq_{\ora} \tr(a)$ s.t. $\tr(d) \ [\so \cup \wro]^* \ \tr(c)$, $c \leq_h d$ and $\swapped{h}{c}$. As $\tr(r) = \tr(d)$, we conclude $R^{\ora}(h)$.
		
		%We find two situations: either $\tr(r)$ has no swapped events or it has them. If no, as $h_b = h \setminus \tr(r)$ has no pending transactions and $\nextEvent(h_b) = \ibegin(\tr(r))$, $h_b$ is in the reachable path to $h$ and $e = \min_{\ora}  \mathcal{P} \setminus h$. That implies there is not such pair $a, b$, so $R^{\ora}(h')$ holds. Otherwise, let $c \in \tr(r)$ a swapped element. 
		
		Finally, let $h' = \swap(h \bullet e, r, w)$ for some $r, w \in h$ s.t. $\protocol(h \bullet e, r, w)$ holds. Let $a, b$ two event s.t. $a \leq_{\ora} b $. If $a \leq_{h'} b$ or, as $R^{\ora}(h)$ and $\protocol(h \bullet e, r, w)$ holds, $a \not\leq_h b$, then the property is satisfied; so let's suppose $b <_{h'} a$ and $a \leq_h b$. In this situation, $a$ has to be a deleted event, so $a \in \mathcal{P} \setminus h' \cup \{r\}$. As $r \leq_h a$, if $a \leq_{\ora} r$, there would exist a $c \in h $, $\tr(c) \leq_{\ora} \tr(a) \leq_{\ora} \tr(r)$ s.t. $\tr(r) \ [\so \cup \wro]^* \ \tr(c)$ and $\swapped{h}{c}$. However, this contradicts $\protocol(h \bullet e, r, w)$; so $r \leq_{\ora} a$. Taking $e'' = r$ the property is witnessed. 
		%The only transaction in the history whose relative order has changed is $\tr(r)$, so $b \in \tr(r)$. In that setting, we can take $r$ as a
	\end{proof}
\end{lemma}

\begin{proposition}
\label{proposition:orders-coincide}
For any reachable history $h$, $\leq^h \equiv \leq_h$.
\begin{proof}
	We will prove this lemma by induction on the number of steps a computable path leading to $h$ are required by algorithm \textcolor{red}{cite algorithm}. The base case, $n = 0$, implies $h = \emptyset$, so both relations hold. Let's suppose that for every history $h'$ that requires at most $n$ steps, $\leq^{h'} \equiv \leq_{h'}$; and let's analyze $\leq^h$ for a history computed with $n+1$. In particular, there exists a history $h_p$ in that path which is an immediate predecessor of $h$. We will distinguish cases depending on how from $h_p$ we reach $h$; calling $e = \nextEvent(h)$
	\begin{itemize}
		\item \underline{Adding a $ \iend, \iwrite$:} As $h_p$ and $h$ are edge-wise identical, $\leq^h \equiv \leq_h$.
		
		\item \underline{Adding a $\ibegin$:} As $\dep(h_p, T, \bot) = \dep(h, T, \bot)$ for every transaction in $h_p$, if $T \leq^{h_p} T'$, then $T \leq^h T'$. Moreover, $\dep(h, \tr(e), \bot) = \{e\}= \min_{\ora} \mathcal{P} \setminus h_p$. By \ref{lemma:reachable-or-respectful} $h$ is $\ora$-respectful, so for every $T$, $\min_{\ora} \dep(h, T, \bot) <_{\ora} e$; which implies $T <^h \tr(e)$. By lemma \ref{lemma:canonincal-total-order}, $\leq^h$ is a total order, so it coincides with $\leq^h$.
		
		\item \underline{Adding a $\iread$:} As no transaction depends on $\tr(e)$ and $\tr(e) = \last{h_p}$, if we prove that for every pair of transactions $\minimalDependency(h_p, T, T', \bot) = \minimalDependency(h, T, T', \bot)$, the lemma would hold. On one hand, $\dep(h, \tr(e), \bot) = \dep(h_p, \tr(e), \bot) = \tr(e)$ and in the other hand, by lemma \ref{lemma:reachable-or-respectful}, $\min_{\ora} \dep(h_p, T, \bot) <_{\ora} \tr(e)$. Finally, as $e \not\in \dep(h, T, e')$, for every $T \neq \tr(e), e' \neq \bot$, for every pair of transactions $T, T'$, $\minimalDependency(h_p, T, T' \bot) = \minimalDependency(h, T, T' \bot)$. 
		
		\item \underline{Swapping $r \in h$ and $w \in \tr(e)$:} As $\protocol(h \bullet e, r, w)$ is satisfied and $h$ is $\ora$-respectful, for every event $e'$ and transaction $T$, $\min_{\ora} \dep(h_p, T, e') = \min_{\ora} \dep(h, T, e')$, so for every pair of transactions $\minimalDependency(h_p, T, T', \bot) = \minimalDependency(h, T, T', \bot)$. In particular, this implies $T \leq^{h_p} T'$ if and only if $T \leq^h T'$ for every  pair $T, T'$ and $T \leq^h \tr(r)$; so $\leq^h \equiv \leq_h$. 
	\end{itemize}
\end{proof}
\end{proposition}

Proposition \ref{proposition:orders-coincide} is a very interesting result as it express the following fact: regardless of the computable path that leads to a history, the final order between events will be the same. This result will have a key role during both completeness and optimality, as it restricts the possible histories that precede another while describing the computable path leading to it. In addition, proposition \ref{proposition:orders-coincide} together with lemma \ref{lemma:reachable-or-respectful} justify enlarging definition \ref{def:oracle-respectful} with the canonical order instead the computable order; and it is this new shape the one we will be using during the rest of proof.  

%As $\leq^h \equiv \leq_h$ for any reachable history, we will extend $R^{\ora}(h)$ to any history changing $\leq_h$ to $\leq^h$ in \ref{def:oracle-respectful} whenever it is needed. This property is not something reachable histories satisfy but also, as next lemma shows, total histories with $\leq^h$ order do; which justify it as an useful tool for proving completeness.
\begin{lemma}
	\label{lemma:total-respectful}
	Any total history is $\ora$-respectful.
	\begin{proof}
		Let $h$ be a total history and $T, T'$ a pair of transactions s.t. $T \leq_{\ora} T'$. If $T \leq^h T'$, then the statement is satisfied; so let's assume the contrary: $T' \leq^h T$. If $T' \ [\so \cup \wro]^* \ T$, then for every $e \in T, e' \in T'$ $\exists c \in h$ s.t. $\tr(c) \leq_{\ora} \tr(e)$, $\tr(e') \ [\so \cup \wro]^* \tr(c)$, $\swapped{h}{c}$ and $c \leq^h e$; so the property is satisfied. Otherwise, by definition of $\minimalDependency$, there exists $r' \in h$ s.t. $T' \ [\so \cup \wro]^* \ \tr(r')$ and $\tr(r') \leq_{\ora} T$. Moreover, by \textsc{canonicalOrder}'s definition, $\tr(r) \leq^h T$. Finally $\swapped{h}{r'}$ holds as it is the minimum element according $\ora$. To sum up, $R^{\ora}(h)$ holds.
	\end{proof}
\end{lemma}

\begin{algorithm}[H]
	\caption{\textsc{prev}}
	\begin{algorithmic}[1]
		
		\Statex
		\Procedure{\textsc{prev}}{$h$}
		\If{$h = \emptyset$}
		\State \Return $\emptyset$
		\EndIf
		\State $a \gets \last{h}$
		\If{$\lnot \swapped{h}{a}$}
		\State \Return $h \setminus a$
		\Else
		\State \Return $\maxCompletion(h\setminus a, \{e \ | \ e \not\in (h \setminus a) \land e <_{\ora} h.\wro(a) \})$
		\EndIf
		\EndProcedure
		
		\Statex
		\Procedure{\maxCompletion}{$h, D$}
		\If{$D \neq \emptyset$}
		\State $e \gets \min_{<_{\ora}} D$
		\If{$e.type() \neq \iread$}
		\State \Return $\maxCompletion(h \bullet e, D \setminus \{e\})$
		\Else
		\Let $w$ s.t. $\isMaximallyAdded{h \bullet_w e, e}$
		\State \Return $\maxCompletion(h \bullet_w e, D \setminus \{e\})$
		\EndIf
		
		\Else
		\State \Return $h$
		\EndIf
		\EndProcedure
		
		
	\end{algorithmic}
	\label{algorithm:prev}
	%\caption{Generic method for exploring every possible history $h$ of a program $\mathcal{P}$ running under a database with $\mathcal{M}$ as isolation level.}
\end{algorithm}

Function \ref{algorithm:prev} produce a history that are meant to be the previous step of a reachable history. Thanks to this definition, we will show that every total history has a computable path based on applying $\prev^{-1}$ function iteratively until the objective history is reached.


\textcolor{red}{TODO (before): if $h \to \swap(h \bullet e, r, w)$ ``in one step'', actually from $h$ we go to $h \bullet e$ and from it to the swapped. }

%code for splitting algorithmic environment
%\algstore{myalg}
%\end{algorithmic}
%\end{algorithm}

%gap maybe needed to split algorithms in two parts. 

%\begin{algorithm}[H]                   
%\begin{algorithmic} [1]                   % enter the algorithmic environment
%\algrestore{myalg}	



\begin{lemma}
	\label{lemma:prev-respectful}
	For every $\ora$-respectful history $h$, $\prev(h)$ is also $\ora$-respectful.
	\begin{proof}
		Let suppose $h \neq \emptyset$, $h_p = \prev(h)$, $a = \last{h}$, $e \in \mathcal{P}$ and $ e' \in h_p$ s.t. $e \leq_{\ora} e'$. As $R^{\ora}(h)$ is satisfied, either $e \leq^h e'$ or $\exists e'' \in h, \tr(e'') \leq_{\ora} \tr(e)$, $e'' \leq^h e$, $\tr(e') \ [\so \cup \wro]^* \tr(e'')$ and $\swapped{h}{e''}$. If $\lnot \swapped{h}{a}$, $h_p = h \bullet a$; so if $e \leq^h e'$, $e \leq^{h_p} e'$ and if not, $e'' \in h_p$, so $R^{\ora}(h_p)$ holds. 
		
		Otherwise, $\swapped{h}{a}$ and we distinguish between the sets $e$ and $e'$ belong to. Firstly, for every pair of events $\hat{e} \in h_p \setminus h$, $\hat{e}' \in \dep(h, \tr(\hat{e}, \bot))$, we know that $\tr(\hat{e}) \leq_{\ora} \tr(\hat{e}')$. Therefore, $\min_{<_{\ora}} \dep(h, \tr(\hat{e}, \bot)) = \ibegin(\tr(\hat{e}))$. In addition, by construction of $\prev(h)$ and $\ora$-respectfulness of h, for every $\hat{e} \in h, e'' \in h$, $\min_{<_{\ora}} \dep(h_p, \tr(\hat{e}), e'') = \min_{<_{\ora}} \dep(h, \tr(\hat{e}), e'')$. Combining both results, if $e'$ belong to $h$, either $e \leq^{h_p} e'$ or exists a $e'' \in h$ s.t. $e'' \leq^{h_p} e$ and witness $R^{\ora}(h)$ for $e,e'$ (regardless of $e$'s belonging to $h$, $e'' \leq^{h_p} e$). On the contrary, as $h_p$ has no pending transactions, if $e' \not\in h$, $\lnot (\tr(e') \ [\so \cup \wro]^* \ \tr(e))$, so regardless if $\tr(e) \ [\so \cup \wro]^* \tr(e')$, $e \leq^{h_p} e'$. To sum up, $R^{\ora}(h_p)$ holds.
	\end{proof}
\end{lemma}

\begin{lemma}
\label{lemma:soundness-prev}
For every consistent history $\ora$-respectful $h$, if $\prev(h)$ is reachable, then $h$ is also reachable.
\begin{proof}
	Let suppose $h \neq \emptyset$, $h_p = \prev(h)$ and $a = \last{h}$. If $\lnot \swapped{h}{a}$, let $h_n = h_p \bullet a$ if $a$ is not a read, $h_n = h_p \bullet_{h.\wro(a)} a$ in the other case. Either way, $h_n$ is always reachable and it coincides with $h$. Otherwise, $a$ is a $\iread$ event and it swapped; so let us call $w = h.\wro(a)$. Firstly, as $\swapped{h}{a}$, $a <_{\ora} w$, and by lemma \ref{lemma:reachable-or-respectful}, $R^{\ora}(h_p)$ holds, so $a <_{h_p} w$ does; which let us conclude $\compute(h_p)$ will always return $(a, w)$ as a possible swap pair. In addition, all transactions in $h_p$ are non-pending, so in particular $\last{h_p}$ is an $\iend$ event. If we call $h_s = \swap(h_p, a, w)$, and $h_p \setminus h = h_p \setminus h_s$ would hold, as $h \subseteq h_p, h_s \subseteq h_p$, then $h = h_s$; which would allow us to conclude $h$ is reachable from $h_p$.
	
	On one hand, if $e \in h_p \setminus h$, $e \not\in h$ and $e <_{\ora} w$. In particular, $\lnot (\tr(e) \ [\so \cup \wro]^* \ \tr(w))$. Moreover, if $ e \leq_{\ora} a$, by $R^{\ora}(h)$, either $e \leq^h a$ or $\exists e''\in h, e'' \leq_{\ora} e$ s.t. $\tr(a) \ [\so \cup \wro]^* \tr(e'')$, $e'' \leq^h e$ and $\swapped{h}{e''}$; both impossible situations as $e \not\in h$ and $a = \last{h}$; so $a \leq_{\ora} e$. In other words, $e \in h_p \setminus h_s$.
	
	On the other hand, $e \in h_p \setminus h_s$ if and only if $\lnot (\tr(e) \ [\so \cup \wro]^* \ \tr(w))$ and $a <_{\ora} e <_{\ora} w$. If $e \in h$ then $e \leq^{h} a$, and as $h$ is $\ora$-respectful and $a \leq_{\ora} e$, we deduce there exists a $e'' \in h$ s.t. $\tr(e'') \leq_{\ora} \tr(a)$, $\tr(e) \ [\so \cup \wro]^* \tr(e'')$ and $\swapped{h}{e''}$. Moreover, as $c \in h$, $c \in h_p$; but as $\swapped{h_p}{c}$ and $\protocol(h, a, w)$ hold, $c \in h_s$ and so $e$ does. This result leads to a contradiction, so $e \not\in h$; i.e. $e \in h_p \setminus h$.
\end{proof}
\end{lemma}

\begin{corollary}
\label{corollary:prev-swap-identity}
In a consistent $\ora$-respectful history $h$ whose previous history is reachable, if its last event $a$ is swapped, $h$ coincides with $\swap(\prev(h), a, h.\wro(a))$.
\begin{proof}
It comes straight away from the proof of lemma \ref{lemma:soundness-prev}.
\end{proof}
\end{corollary}

\begin{lemma}
\label{lemma:prev-reduces-one}
For every non-empty consistent $\ora$-respectful history $h$, $h_p = \prev(h)$ and $a = \last{h}$, if $\swapped{h}{a}$ then $\{e \in h_p \ | \ \swapped{h_p}{e}\} = \{e \in h \ | \ \swapped{h}{e}\} \setminus \{a\}$, otherwise $h_p = h \setminus a$.
%Let $h$ a history, $h' = \prev(h)$ and at some state $s$, $h'$ appear at line \ref{algorithm:stmc:outer_loop}, from $s$ the algorithm will compute some state $s'$ such that $h$ will also appear at line \ref{algorithm:stmc:outer_loop}.`
\begin{proof}
		Let $a = \last{h}$ and $h' = h \setminus a$. If $a$ is not swapped, then $h_p = h'$, so the lemma holds immediately. Otherwise, as $h_p =  \maxCompletion(h')$, we will show that every event not belonging to $h_p \setminus h'$ is not swapped by induction on every recursive call to $\maxCompletion$. Let us call $D = \{e \ | \ e \not\in h' \land e <_{\ora} \}$. This set, intuitively, contain all the events that would have been deleted from a reachable history $h$ to produce $h_p$. In this setting, let us call $h_{|D|} = h'$, $D_{|D|} = D$ and $D_k = D_{k+1} \setminus \{\min_{<_{\ora}} D_{k+1}\}, \; e_k = \min_{<_{\ora}}D_k$ for every $k, 0 \leq k < |D|$ (i.e. $D_k = D_{k+1} \setminus \{ e_{k+1}\}$). We will prove the lemma by induction on $n = |D| - k$, constructing a collection of histories $h_k$, $0 \leq k < |D|$, such that each one is an extension of its predecessor with a non-swapped event.
		
		The base case, $h_{|D|}$ is trivial as by its definition it corresponds with $h'$. Let's prove the inductive case: $\{e \ | \ \swapped{h_{k+1}}{e}\} = \{e \ | \ \swapped{h'}{e}\}$. If $e_{k+1}$ is not a $\iread$ event, $h_k = h_{k+1} \bullet e_{k+1}$ and $\{e \ | \ \swapped{h_{k}}{e}\} = \{e \ | \ \swapped{h'}{e}\}$; as only $\iread$ events can be swapped. Otherwise, by the model's causal-extensibility there exists a $\iwrite$ event $f_{k+1}$ s.t. writes the same variable and $\isConsistent{h_{k+1} \bullet_{f_{k+1}} e_{k+1}} \land \tr(f_{k+1}) \ [\so \cup \wro]^* \ \tr(e_{k+1})$ holds. $ \{e \ | \ \swapped{h_{k+1}}{e}\} = \{e \ | \ \swapped{h_{k+1} \bullet_{f_{k+1}} e_{k+1}}{e}\}$ holds. Let $E_{k+1} = \{w \ | \ \isConsistent{h_{k+1} \bullet_{w} e_{k+1}} \land \{e \ | \ \swapped{h_{k+1}}{e}\} = \{e \ | \ \swapped{h_{k+1} \bullet_{w} e_{k+1}}{e}\}\}$ and $w_{k+1} = \max_{\leq^{h_{k+1}}} E_{k+1}$. This element is well defined as $f_{k+1}$ belongs to $E_{k+1}$. Therefore, $h_k = h_{k+1} \bullet_{w_{k+1}} e_{k´+1}$ is consistent and $\{e \ | \ \swapped{h_{k}}{e}\} = \{e \ | \ \swapped{h'}{e}\}$. Moreover, let's remark that as $w_{k+1}$ is the maximum write event according to $\leq_{h_{k+1}}$ s.t. $\isConsistent{h_k}$ and $\{e \ | \ \swapped{h_{k}}{e}\} = \{e \ | \ \swapped{h'}{e}\}$ and $R^{\ora}(h)$, it also satisfies $\isMaximallyAdded{h_{k}}{ e_{k+1}, w_{k+1}}$. Altogether, we obtain $h_p = h_0$; which let us conclude $\{e \in h_p \ | \ \swapped{h_p}{e}\} = \{e \in h' \ | \ \swapped{h'}{e}\} = \{e \in h \ | \ \swapped{h}{e}\}\setminus \{a\}$.
	\end{proof}
\end{lemma}

\begin{lemma}
	\label{lemma:prev-leads-empty}
	For every history $h$ there exists some $k_h \in \mathbb{N}$ such that $\prev^{k_h}(h) = \emptyset$.
	\begin{proof}
		This lemma is immediate consequence of lemma \ref{lemma:prev-reduces-one}. Let us call $\xi(h) = |\{e \in h \ | \ \swapped{h}{e}\}$, the number of swapped events in $h$, and let us prove the lemma by induction on $(\xi(h), |h|)$. The base case, $\xi(h) = |h| = 0$ is trivial as $h$ would be $\emptyset$; so let's assume that for every history $h$ such that $\xi(h) < n$ or $\xi(h) =h \land |h| < m$ there exists such $k_h$. Let $h$ then a history s.t. $\xi(h) = n$ and $|h| = m$. $h_p = \prev(h)$. On one hand, if $h_p = h \setminus a$ then $\xi(x_p) = \xi(h)$ and $|h_p| = |h|-1$. On the other hand, if $h_p \neq h \setminus a$, $\xi(h_p) = \xi(h) - 1$. In any case, by induction hypothesis on $h_p$, there exists an integer $k_{h_p}$ such that $\prev^{k_{h_p}}(h_p) = \emptyset$. Therefore, $k_h = k_{h_p}+ 1$ satisfies $\prev^{k_h}(h) = \emptyset$.
	\end{proof}
\end{lemma}

\begin{proposition}
	\label{proposition:chain-histories-proof}
	For every consistent $\ora$-respectful history $h$ exists $k \in \mathbb{N}$ and some sequence of $\ora$-respectful histories $\{h_n\}_{n = 0}^k$, $h_0 = \emptyset$ and $h_k = h$ such that the algorithm will compute.
	\begin{proof}
		Let $h$ a history, $k$ the minimum integer such that $\prev^k(h) = \emptyset$, which exists thanks to lemma \ref{lemma:prev-leads-empty} and $C = \{\prev^{k-n}(h)\}_{n = 0}^k$ a set of indexed histories. By the collection's definition and lemma \ref{lemma:prev-respectful}, $h_0 = \prev^k(h) = \emptyset$, $h_k = \prev^0(h) = h$ and $R^{\ora}(h_n)$ for every $n \in \mathbb{N}$; so let us prove by induction on $n$ that every history in $C$ is reachable. The base case, $h_0$, is trivially achieved; as it is always reachable. In addition, by lemma \ref{lemma:soundness-prev}, we know that if $h_n$ is reachable, $h_{n+1}$ is it too; which proves the inductive step. %Moreover, so we will focus on the inductive case, assuming $h_n$ is reachable and deducing $h_{n+1}$ is it too. 
	\end{proof}
\end{proposition}

\begin{theorem}
	Algorithm \textcolor{red}{cite algorithm} is complete.
	\begin{proof}
		By lemma \ref{lemma:total-respectful}, any consistent total history is $\ora$-respectful. As a consequence of proposition \ref{proposition:chain-histories-proof}, there exist a sequence of reachable histories which $h$ belongs to; so in particular, $h$ is reachable.
	\end{proof}
\end{theorem}



\subsection{Optimality}

For proving optimality we are gonna exploit two properties already studied for completeness: $\ora$-respectfulness and the canonical order. Then, as algorithm \textsc{explore-ce} is sound and complete, we will prove that any computable path leading to a consistent history is the one computed in the completeness' proof.

%\textcolor{red}{I have this new proof shorter than the initial idea but I do not know if it doesn't have typos. TODO: reread it!}

\begin{comment}
\begin{lemma}
\label{lemma:swapped-elements-remain-1step}
Let $h$ a reachable history and $e$ a swapped event in $h$. If $h'$ is an immediate successor of $h$, $e \in h'$, $\swapped{h'}{e}$ and $h'.\wro(e) = h.\wro(e)$.
\begin{proof}
Firstly, if $a = \nextEvent(h)$, every history $h' = h \bullet a$ (or $h' = h \bullet_w a$ if $a$ is a $\iread$ event for some $\iwrite$ $w$) contains $e$ and $e$ is swapped; $h'$ is an extension of $h$ and $\leq_{h'} \restriction_h \equiv \leq_h$. Let's suppose then $h' = \swap(h, r, w)$ for some $r, w \in h$ s.t. $\protocol(h, r, w)$ holds. By algorithm \textcolor{red}{cite algo}'s definition, we know that if $e \in h \setminus h'$, $\isMaximallyAdded{h}{e}$ holds, so $\lnot(\swapped{h}{e})$. In particular, $e \in h'$. Moreover, if there would exist a transaction $T$ such that $T <_{\ora} \tr(e)$, $T <_h \tr(e)$ and $\tr(h'.\wro(e)) \ [\so \cup \wro]^* \ T$, by the $\ora$-respectfulness of $h$ (lemma \ref{lemma:reachable-or-respectful}) we would obtain that $e$ is no swapped in $h$; which contradicts the hypothesis. Therefore, as in every reachable history $h.\wro(e) \leq_h e$, we deduce $\swapped{h'}{e}$. 
%applying or-resp to T <_or \tr(e) and disctiminating cases depending if e <h r or not. (in case we need to expand this)
\end{proof}
\end{lemma}

\begin{corollary}
\label{corollary:swapped-elements-remain}
For every reachable history $h$ and every event $e$ s.t. $\swapped{h}{e}$, if a history $h'$ is reached from $h$, $\swapped{h'}{e}$
\begin{proof}
By induction on the steps' number on a computable path from $h$ to $h'$, applying lemma \ref{lemma:swapped-elements-remain-1step} in the inductive step.
\end{proof}
\end{corollary}

\begin{lemma}
For every $\iread$ event $r$, $\iwrite$ $w$, $w \leq_{\ora} r$ there is no pair of reachable histories $h, h'$ s.t. $h.\wro(r) \neq w$, $h'.\wro(r) = w$ and $h'$ is computed from $h$.
\begin{proof}
Let's suppose such a pair of histories exist a

\end{proof}
\end{lemma}

\end{comment}

\begin{theorem}
	Algorithm \textup{\textsc{explore-ce}} is strongly optimal.
	\begin{proof}
		As the model is causal-extensible, any algorithm optimal is also strongly optimal. Let us prove that for every reachable history there is only a computable path that leads to it from $\emptyset$.
		Let's suppose there exists a history $h$ that is reached $p_1$, $p_2$ by two computable paths. By lemma \ref{proposition:orders-coincide}, we know that $\leq_h \equiv \leq^h$. However, $\leq^h$ is an order that does not depend on the computable path that leads to $h$; so neither does $\leq_h$.  Therefore, we can assume without loss of generality that $h$ is a history with minimal value of $\xi(h) =|\{e \in h \ | \swapped{h}{e}\}|$ and in case of tie, that is minimal with respect $|h|$; values independent of the computable path that leads to $h$. 
		
		We can also assume without loss of generality that the predecessor of $h$ in $p_1$ is $h_1 = \prev{h}$, and $h_2$ is the predecessor of $h$ in $p_2$. If we prove $h_1$ and $h_2$ are identical, $p_1$ and $p_2$ have to also be identical and therefore, the algorithm would be optimal. Firstly, if $\last{h}$ is not a swapped $\iread$ event, by the definition of $\nextEvent$ function $h_2 = h \setminus \last{h} = h_1$. On the contrary, let's suppose $r = \last{h}$ is a swapped event that reads from a transaction $t$. Because $\swapped{h}{r}$ holds, from $h_2$ to $h$ it has to have happened a swap between $r$ and $w$. But by corollary \ref{corollary:prev-swap-identity}, $h = \swap(h_1, r, w)$, so $h_1 \restriction_{h \setminus r} = h_2 \restriction_{h \setminus r}$. As $h_1, h_2$ are both $\ora$-respectful, $e \in h_1 \setminus h \iff e \in h_2 \setminus h$. Finally, as $\genericProtocol(h_i, r, w)$ holds for $i \in \{1,2\}$, for every $\iread$ event $e$ in $h_1 \cap h_2$ there exists a transaction $t_e$ s.t. $\wro(e, t_e)$ for both histories.
	\end{proof}
\end{theorem}