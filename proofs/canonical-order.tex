%\textcolor{red}{Problem of this proofs: I assume optimality before proving it! \sout{ I don't know what wording we shall apply} (I tried to modify it, let's see if it is noticeable).}

\subsubsection{Canonical order} $ $\\

As mentioned, we need to formally define a total order for every history that coincide on reachable histories with the history order. For achieving it, we analyze how the algorithm orders transaction logs in a history. In particular, we observe that if two transactions $t, t'$ have a $(\so \cup \wro)^*$ dependency, the history order in the algorithm orders them analogously. But if they are $(\so \cup \wro)^*$-incomparable, the algorithm prioritizes the one that is read by a smaller $\iread$ event according $\ora$. Combining both arguments recursively we obtain a \textit{canonical order} for a history, which is formally defined with the function presented below.

\begin{algorithm}[H]
	\caption{\textsc{Canonical order}}
	\begin{algorithmic}[1]
		
		\Statex
		\Procedure{\textsc{canonicalOrder}}{$h, t, t'$}
		\State \Return $t \ [\so \cup \wro]^* \ t' \lor$
		\State \qquad $(\lnot(t' \ [\so \cup \wro]^* \ t) \land \minimalDependency(h, t, t', \bot)$
		\EndProcedure
		
		\Statex
		\Procedure{\minimalDependency}{$h, t, t', e$}
		\Let $a = \min_{<_{\ora}} \dep(h, t, e)$; $a' = \min_{<_{\ora}} \dep(h, t', e)$
		\If{$a \neq a'$}
		\State \Return $a <_{\ora} a'$
		\Else
		\State \Return $\minimalDependency(h, t, t', a)$
		\EndIf
		\EndProcedure
		
		\Statex
		\Procedure{\dep}{$h, t, e$}
		\State \Return $\{r \ | \exists t' \text{ s.t. } t \ [\so \cup \wro]^* \ t' \land \ t' \ [\wro] \ r \land \trans{h}{r} \ [\so \cup \wro]^+ \trans{h}{e} \} \cup t$ 
		\EndProcedure	
	\end{algorithmic}
	\label{algorithm:canonical-order}
	%\caption{Generic method for exploring every possible history $h$ of a program $\mathcal{P}$ running under a database with $\mathcal{M}$ as isolation level.}
\end{algorithm}


The function $\textsc{canonicalOrder}$ produces a relation between transactions in a history, denoted $\leq^h$. In algorithm \ref{algorithm:canonical-order}'s description, we denote $\bot$ to represent the end of the program, which always exists, and that is $\so$-related with every single transaction.

Firstly, we prove our canonical order is well defined for every pair of transactions.

\begin{lemma}
	\label{lemma:dep_shrinks}
	For every history $h$, event $e$ and transaction $t$, $\dep(h, t, \min_{<_{\ora}} \dep(h, t, e)) \subseteq \dep(h, t, e)$. Moreover, if $\dep(h, t, e) \neq t$, the inclusion is strict.
	
\end{lemma}
\begin{proof}
	Let $r' = \min_{<_{\ora}} \dep(h, t, e)$. If $\dep(h, t, r') = t$ the lemma is trivially proved, so let's suppose there exists $r \in \dep(h, t,r') \setminus t$. Then, $\exists t' \text{ s.t. } t \ [\so \cup \wro]^* \ t' \land \ t' \ [\wro] \ r \land \trans{h}{r} \ [\so \cup \wro]^+ \trans{h}{r'}$ and $\exists t'' \text{ s.t. } t \ [\so \cup \wro]^* \ t'' \land \ t'' \ [\wro] \ r' \land \trans{h}{r'} \ [\so \cup \wro]^+ \trans{h}{e}$; so $\trans{h}{r} \ [\so \cup \wro]^+ \trans{h}{r'} \ [\so \cup \wro]^+ \trans{h}{e}$. In other words, $r \in \dep(h, t, e)$. The moreover comes trivially as $r' \not\in \dep(h, t, r')$.
\end{proof}

\begin{lemma}
	\label{lemma:minimalDependency-halts}
	For every pair of distinct transactions $t, t'$, $\minimalDependency(h,t,t',\bot)$ always halts.
\end{lemma}
\begin{proof}
	Let's suppose by contrapositive that $\minimalDependency(h,t, t',\bot)$ does not halt. Therefore, there would exist an infinite chain of events $e_n, n \in \mathbb{N}$ such that $e_0 = \bot, e_{n+1} = \min_{\ora}\dep(h, t, e_{n}) = \min_{\ora}\dep(h, t', e_{n})$. Firstly, as $h$ is finite, so are both $\dep(h, t, e_{n})$ and $\dep(h, t', e_{n})$. Moreover, if $e_n \not \in t$, $\dep(h, t, e_{n+1}) \subsetneq \dep(h, t, e_{n})$ (and analogously for $t'$). Therefore, there exist some indexes $n_0, m_0$ such that $e_{n_0} \in t$ and $e_{m_0} \in t'$. Let $k = \max\{n_0, m_0\}$. Because ; but if $e_n \in t$, $t = \dep(h, t, e_n)$ and $e_{n+1} = e_n$, so $e_k = e_{n_0}$ and $e_k = e_{m_0}$. Therefore $e_k \in t \cap t'$; so $t = t'$ as transaction logs do not share events; which contradict the assumptions.
\end{proof}

\begin{corollary}
	The relation $\leq^h$ is well defined for every pair of transactions.
\end{corollary}
\begin{proof}
	As by lemma \ref{lemma:minimalDependency-halts}, we know that $\minimalDependency(h, t, t', \bot)$ always halts if $t \neq t'$; it is clear that $\textsc{canonicalOrder}(h, t, t')$ also does it. Therefore, the relation is well defined.
\end{proof}

Now that $\leq^h$ has been proved a well defined relation between each pair of transactions, let us prove that it is indeed a total order.


\begin{lemma}
	\label{lemma:canonincal-total-order}
	The relation $\leq^h$ is a total order.
\end{lemma}
\begin{proof}
	$ $
	\begin{itemize}
		\item \underline{Strongly connection} Let $t_1, t_2$ s.t. $t_1 \not \leq^h t_2$. If $t_2 \ [\so \cup \wro]^* t_1$, then $t_2 \leq^h t_1$. Otherwise, as $\lnot(t_1 \ [\so \cup \wro]^* \ t_2)$ and $\minimalDependency$ halts (lemma \ref{lemma:minimalDependency-halts}) either $\minimalDependency(h, t_1, t_2, \bot)$ or $\minimalDependency(h,t_2, t_1, \bot)$ holds. But as $t_1 \not\leq^h t_2$, $t_2 \leq^h t_1$.
		\item \underline{Reflexivity:} By definition, for every $t$, $t \leq^h t$.
		\item \underline{Transitivity:} Let $t_1, t_2, t_3$ three distinct transactions such that $t_1 \leq^h t_2 $ and $t_2 \leq^h t_3$. Clearly, if $t_1 \ [\so \cup \wro]^* \ t_3$, $t_1 \leq^h t_3$. However, if $t_3 \ [\so \cup \wro]^* \ t_1$, we would find one of the following three scenarios:
		\begin{itemize}
			\item $t_1 \ [\so \cup \wro]^* \ t_2$, which is impossible by strong connectivity as that would mean $t_3 \leq^h t_2$.
			\item $t_2 \ [\so \cup \wro]^* \ t_3$, which is also impossible by strong connectivity, as $t_2 \leq^h t_1$.
			\item $\lnot(t_1 \ [\so \cup \wro]^* \ t_2)$ and $\lnot(t_2 \ [\so \cup \wro]^* \ t_3)$. Then, let us call $e^i_0 = \bot$ and $e^i_{n+1} = \min_{<_{\ora}}\dep(h, t_i, e^i_{n})$ for $i \in \{1,2, 3\}$. Let's prove by induction that if for every $k < n$ $e^1_n \not\in t^1$, then $e^1_n = e^{2}_n = e^3_n$. Clearly this hold for $n = 0$ and, assuming it holds for every $k \leq n-1$, as $t_1 \leq^h t_2$, $t_2 \leq^h t_3$, we know $e^1_n \leq_{\ora} e^2_n \leq_{\ora} e^3_n$ and as $t^3 \ [\so \cup \wro]^* \ t^1$, if $e^1_n \not\in t^1$, $ e^3_n \leq_{\ora} e^1_n$. In other words, they coincide. However, by lemma \ref{lemma:minimalDependency-halts}, we know $\minimalDependency(h, t^1, t^3, \bot)$ halts, so there exists some minimal $n_0$ such that $e^1_{n_0} \in t^1$; so $e^2_{n_0} \in t_1$. That implies $t^2 \ [\so \cup \wro]^* \ t_1$; which is impossible as $t_1 \leq^h t_2$.
		\end{itemize}
		
		We deduce then that either $t_1 \ [\so \cup \wro]^* \ t_3$ or $\lnot(t_3 \ [\so \cup \wro]^* \ t_1)$. In the latter case, let's take the sequence $e^i_n$, $i \in \{1,2,3\}$ defined in the last paragraph. Then, as by lemma \ref{lemma:minimalDependency-halts} $\minimalDependency(h, t_1, t_3, \bot)$ halts, there exists a maximum index $n_0$ such that $e^1_{n_0} = e^2_{n_0} = e^3_{n_0}$. Then $e^1_{n_0 + 1} <_{\ora} e^2_{n_0+1}$ or $e^2_{n_0+1} <_{\ora} e^3_{n_0}$; so $t_1 \leq^h t_3$.
		
		\item \underline{Antisymmetric} Let $t_1, t_2$ s.t. $t_1 \leq^h t_2$ and $t_2 \leq^h t_1$. If $t_1 \ [\so \cup \wro]^* t_2$, then $t_1 = t_2$. If not, by the symmetric argument, $\lnot(t_2 \ [\so \cup \wro]^* t_1)$. In that situation, by lemma \ref{lemma:minimalDependency-halts} we know both $\minimalDependency(h,t_1,t_2,\bot)$ and $\minimalDependency(h,t_1,t_2,\bot)$ halt and cannot be satisfied at the same time. This contradicts that both $t_1 \leq^h t_2$ and $t_2 \leq^h t_1$ hold; so $t_1 = t_2$.		
	\end{itemize}
\end{proof}