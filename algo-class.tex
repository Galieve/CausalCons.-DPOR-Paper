\section{The class of swapping based algorithms}

\begin{definition}
Given an algorithm $A$ and a model $\mathcal{M}$, we say:
\begin{itemize}
	%\item $A$ is \callout{blind} if for every program $\mathcal{P}$ and every history $h$, $A$ cannot determine the evaluation of any instruction $i$ not yet executed (i.e. $i \in \mathcal{P} \setminus h$),
	%\item $A$ is \callout{weakly $\mathcal{M}$-sound} if for every program $\mathcal{P}$, every total history $h$ computed by $A$ is $\mathcal{M}$-consistent,
	%\item $A$ is \callout{$\mathcal{M}$-sound} if for every program $\mathcal{P}$, every history $h$, partial or not, computed by $A$ is $\mathcal{M}$-consistent,
	\item $A$ is \callout{$\mathcal{M}$-sound} if for every program $\mathcal{P}$, every total history $h$ computed by $A$ is $\mathcal{M}$-consistent, \textcolor{red}{New version! (problems in last section!)}
	\item $A$ is \callout{complete} if for every program $\mathcal{P}$ it computes every possible execution,
	\item $A$ is \callout{weakly optimal} if for every program $\mathcal{P}$ it computes every execution at most once,
	\item $A$ is \callout{optimal} if for every program $\mathcal{P}$ it is weakly optimal and it does not compute blocking executions (i.e. partial histories that cannot be completed).	
\end{itemize}
\end{definition}


\begin{definition}
The algorithm's class $\mathcal{O}^n_{\mathcal{M}}$ is defined as the minimal collection containing all algorithms $\mathcal{M}$-sound, complete and optimal that employ polynomial memory and allows at most $n$ pending transactions. %We also denote $\mathcal{O}_{\mathcal{M}} = \bigcup\limits_{n \in \mathbb{N}} \mathcal{O}^n_{\mathcal{M}}$.
\end{definition}


\begin{comment}
\begin{definition}
%The algorithm's class $\mathcal{L}^n_{\mathcal{M}}$ is defined as the minimal collection containing all algorithms $\mathcal{M}$-sound, complete and weakly optimal that employ polynomial memory and allows at most $n$ pending transactions. 
\end{definition}

\end{comment}


\begin{definition}
The algorithm's class $\mathcal{W}^n_{\mathcal{M}}$ is defined as the minimal collection containing all algorithms \textcolor{red}{\sout{weakly}} $\mathcal{M}$-sound, complete and weakly optimal that employ polynomial memory and allows at most $n$ pending transactions. 
\end{definition}

%\textcolor{red}{We need the three classes for the following: RA/RC/CC/PRE algorithms belong to $O^1$, our PRE+SER/PRE+SI version is $W^1$, for SER we cannot obtaining it in $O^n$ for any $n$ nor in $L^1$ and in RA/RC/CC/PRE $O^1 = L^1$. Moreover, if we allow the subindex there, we can talk about future work allowing multiple pending transactions (it may be reasonable to prove that there is an algorithm for SER in $L^n$) and indicate Viktor's algorithm is not in $L^n$ for any $n$ as it employs exp memory. If we don't want to be that expressive, we can only talk about weakly optimal and then proof that under RA/RC/CC/PRE with our algorithms there is no blocking branches as an extra feature thanks to the models; but I thought it may be clearer to distribute it in this way.}

The above-mentioned classes are, notwithstanding, quite dense and contain lots of algorithms we are not interested in (as non-DPOR algorithms for example). In particular, our work focuses on the collection of algorithms defined by the schema \ref{algorithm:algo-class} for some functions $\genericNext, \genericEvaluate, \genericCompute, \genericProtocol, \genericSwap$ to be defined. Thanks to this approach we will show the existence of some algorithms and easily prove some of their properties.

\begin{algorithm}[H]
	\caption{\textsc{explore} algorithm}
	\begin{algorithmic}[1]
		\InputAlgorithmic $h$: history
		
		\Statex
		
		\State $e \gets  \textbf{\genericNext}(h)$
		
		
		\If{$\textsc{type}(e) = \bot$}
			\If{$\textbf{\genericEvaluate}(h)$}
				\State \textbf{output} $h$
			\EndIf
		%\State $\textbf{\genericEvaluate}(h)$
		\State \Return
		\ElsIf{$\textsc{type}(e) = \iread$}
		
		\State $x \in \mathcal{V}$ s.t. $\readVar{e}{x}$
		\ForAll{$w \in h \text{ s.t. } \writeVar{w}{x}$}
		
		\State $\textsc{explore}(h \bullet_w e)$
		\label{algorithm:algo-class:exploreRead}
		\EndFor
		\Else
		\State $\textsc{explore}(h \bullet e)$
		\label{algorithm:algo-class:exploreStd}
		\EndIf
		
		\State $l \gets \textbf{\genericCompute}(h)$
		
		\ForAll{$(\alpha, \beta) \in l$}
		\If{$\textbf{\genericProtocol}(h, \alpha, \beta)$}
		\State $\textsc{explore}(\textbf{\genericSwap}(h, \alpha, \beta))$	
		
		\EndIf
		\EndFor
		
		
	\end{algorithmic}
	\label{algorithm:algo-class}
	%\caption{Generic method for exploring every possible history $h$ of a program $\mathcal{P}$ running under a database with $\mathcal{M}$ as isolation level.}
\end{algorithm}

Algorithm \ref{algorithm:algo-class} explores systematically the space of histories, selecting a new event $e$ to be added to some history $h$ if possible, thanks to the function $\genericNext$ called \textit{next}. If it wasn't possible, it is due to $h$ being either a total execution or an undesirable one; both behaviors discriminated via \textit{evaluating} $h$ with the $\genericEvaluate$ function. Otherwise, $e$ will added $h$ along with an eventual $\wro$-edge. Moreover, at some point during the search traversal determined by $\genericCompute$ the algorithm will \textit{compute} some collection of events $\alpha, \beta$ that may needed to be reordered. Every events' rescheduling possibly lead to a different execution, so for controlling which reorderings we are shall explore, we enforce a \textit{reordering protocol} (function $\genericProtocol$). In the affirmative case, the new histories would be generated via $\genericSwap$, \textit{swapping} $\beta$ and all their dependencies before $\alpha$. However, the reader shall take into account that this high-level description of algorithm \ref{algorithm:algo-class} may not be satisfied for some $\textsc{explore}$'s instance.
%may not be satisfied for some $\textsc{explore}$'s instances.

%determines so, we will also explore the history generated via $\genericSwap$; where $\alpha$ and $\beta$ have different relative order. It is assumed that \ref{algorithm:algo-class} maintains a total order between events in the history that doesn't contradict $\so$. Finally, let's take into account that this high-level description of the algorithm may not be satisfied for some $\textsc{explore}$'s instances. %\textcolor{red}{Here I wanted to guide the reader of what does this algorithm mean; but remarking that of course, if you add a weird function, then all what I said is a lie (for example if the swap history does not swap at all).}

\begin{definition}
The \callout{swapping-based algorithm's class} for the memory model $\mathcal{M}$, $\mathcal{S}_{\mathcal{M}}$, is the minimal collection containing all algorithms $A$ that can be described as algorithm \ref{algorithm:algo-class}'s instances: $ \textsc{explore}(\genericNext, \genericEvaluate, \genericCompute, \genericProtocol, \genericSwap)$; where $\genericNext, \genericEvaluate, \genericCompute, \genericProtocol, \genericSwap$ are functions defined as follows: 
\begin{itemize}
	\item $\genericNext: \mathcal{H}_\mathcal{P} \to \mathcal{E}_\mathcal{P}$, where for every $h \in \mathcal{H}_{\mathcal{P}}$, $\genericNext(h) \not\in h$ and for every event $e$ s.t. $e \ [\so] \ \genericNext(h)$, $e \in h$, %\textcolor{red}{redundant with weakly optimal; it may be helpful to the reader}
	\item $\genericEvaluate: \mathcal{H}_\mathcal{P} \to \{0, 1\}$, that determines if an execution is total or not, %\textcolor{red}{this is for determine when computing a total inconsistent history shall be taken into account}
	\item $\genericCompute: \mathcal{H}^<_\mathcal{P} \to \mathcal{E}_\mathcal{P}^* \times \mathcal{E}_\mathcal{P}^*$, where for every history $h$, $\genericCompute(h) = (\alpha, \beta)$, $\alpha \cap \beta = \emptyset$ and for every $(e, e') \in \alpha \times \beta$, $e <_h e'$.
	\item $\genericProtocol: \mathcal{H}^<_\mathcal{P} \times \mathcal{E}_\mathcal{P}^* \times \mathcal{E}_\mathcal{P}^* \to \{0,1\}$, %\textcolor{red}{shall we impose the protocol to only be true in only one history for each class? not needed though...}
	\item $\genericSwap: \mathcal{H}^<_\mathcal{P} \times \mathcal{E}_\mathcal{P}^* \times \mathcal{E}_\mathcal{P}^* \to \mathcal{H}^<_\mathcal{P}$, where for every $h, \alpha, \beta$, we have $\genericSwap(h, \alpha, \beta) = h'$, $\alpha, \beta \in h'$, for every $(e, e') \in \alpha \times \beta \implies e >_{h'} e'$ and there exists some $(e, e') \in (\alpha, \beta)$ s.t. $\tr(e') \ [\wro] \ \tr(e)$.
\end{itemize}
\end{definition}

The swapping-based algorithms have been already studied in the literature as for example \textcolor{red}{cite Viktor's algorithm}, which belongs to $\mathcal{S}_{SC}$; where SC in this case is the axiomatic representation of sequential consistency memory model. \textcolor{red}{cite SC.}