\section{The particular case of Prefix Consistency}

\textit{Prefix Consistency} is an special isolation level as it is not causal consistent \textcolor{red}{Example} but every partial history $\PRE$-consistent can be extended into a consistent history \ref{theorem:maximally-extensible-PRE}. Therefore, neither algorithm \textcolor{red}{cite algorithm} is guaranteed to be sound, complete and strongly optimal for $\PRE$ nor the non-existence of such algorithm can be deduced in the same way as in theorem \textcolor{red}{cite SI and SER dont work}. In this section we present one program for which algorithm \textcolor{red}{cite algorithm} may or may not be consistent depending on the oracle order employed.

\begin{figure}[H]
	
	\centering
\begin{subfigure}[b]{.25\textwidth}
\begin{minipage}{2cm}
\begin{lstlisting}[xleftmargin=5mm,basicstyle=\ttfamily\scriptsize,escapeinside={(*}{*)}]
begin;
a=read((*$x$*));
write((*$x$*),1);
commit
begin;
b=read((*$y$*));
write((*$y$*),1);
commit
\end{lstlisting}
\end{minipage}
\begin{minipage}{1mm}
||
\end{minipage}
\hspace{-5mm}
\begin{minipage}{1.3cm}
\begin{lstlisting}[xleftmargin=5mm,basicstyle=\ttfamily\scriptsize,escapeinside={(*}{*)}]
begin;
a=read((*$y$*));
write((*$y$*),2);
commit
begin;
b=read((*$x$*));
write((*$x$*),2);
commit
\end{lstlisting}
\end{minipage}
\caption{Program.}
\label{fig:pref-impossibility:prog}
\end{subfigure}
	\hspace{.5cm}
	\centering
	\begin{subfigure}[b]{.15\textwidth}
		\resizebox{\textwidth}{!}{
			\begin{tikzpicture}[->,>=stealth',shorten >=1pt,auto,node distance=3cm,
				semithick, transform shape]
				
				%\\ \multicolumn{1}{c}{ \ldots}
				\node[draw, rounded corners=2mm,outer sep=0] (t0) at (0,0) {\begin{tabular}{l} $\init$ \end{tabular}};
				\node[draw, rounded corners=2mm,outer sep=0, label={[font=\small]50:$t_1$}] (t1) at (0, -2) {\begin{tabular}{l} 
						$a \gets \rd{x}$ \\ $\wrt{x}{1}$
				\end{tabular}};
				\node[draw, rounded corners=2mm,outer sep=0, label={[font=\small]50:$t_2$}] (t2) at (0, -8) {\begin{tabular}{l} 
						$a \gets \rd{y}$ \\ $\wrt{y}{2}$
				\end{tabular}};
				\node[draw, rounded corners=2mm,outer sep=0,  label={[font=\small]50:$t_3$}] (t3) at (0, -4) {\begin{tabular}{l} 
						$b \gets \rd{y}$ \\ $\wrt{y}{1}$
				\end{tabular}};	
				
				\node[draw, rounded corners=2mm,outer sep=0,  label={[font=\small]50:$t_4$}] (t4) at (0, -6) {\begin{tabular}{l} 
						$b \gets \rd{x}$ \\ $\wrt{x}{2}$
				\end{tabular}};		
				
				\path (t0.south west) -- (t0.south) coordinate[pos=0.67] (t0sw);
				\path (t0.south east) -- (t0.south) coordinate[pos=0.67] (t0se);
				\path (t1.north west) -- (t1.north) coordinate[pos=0.67] (t1nw);
				\path (t2.north west) -- (t2.north) coordinate[pos=0.67] (t2nw);
				\path (t2.north east) -- (t2.north) coordinate[pos=0.67] (t2ne);
				\path (t1.south west) -- (t1.south) coordinate[pos=0.67] (t1sw);
				\path (t1.south east) -- (t1.south) coordinate[pos=0.67] (t1se);
				\path (t3.south east) -- (t3.south) coordinate[pos=0.67] (t3se);
				\path (t3.north east) -- (t3.north) coordinate[pos=0.67] (t3ne);
				\path (t3.north west) -- (t3.north) coordinate[pos=0.67] (t3nw);
				\path (t3.south west) -- (t3.south) coordinate[pos=0.67] (t3sw);
				\path (t4.north east) -- (t4.north) coordinate[pos=0.67] (t4ne);
				\path (t4.north west) -- (t4.north) coordinate[pos=0.67] (t4nw);
				
				
				\path (t0.south east) edge [bend left] node {$\wro_y$} (t3.north east);
				\path (t1.south west) edge [bend right] node[above left] {$\wro_x$} (t4.north west);
				\path (t0.south) edge [left] node {$\wro_x$} (t1.north);
				%\path (t1sw) edge [left] node {$\so$} (t2nw);
				\path (t3.south) edge [left] node {$\so$} (t4.north);
				\path (t1.south west) edge [bend right] node [left] {$\so$} (t2.north west);
				%\path (t4nw) edge [left] node {$\wro_x$} (t1se);
		
				\path (t0.south east) edge [bend left] node [below left]{$\wro_y$} (t2.north east);
				%\path (t2) edge [right] node {$\wro_x$} (t3);
			\end{tikzpicture}  
			
		}
		\caption{History $1$.}
		\label{fig:pref-impossibility:a}
	\end{subfigure}
	\hspace{.3cm}
	\centering
	\begin{subfigure}[b]{.15\textwidth}
		\resizebox{\textwidth}{!}{
			\begin{tikzpicture}[->,>=stealth',shorten >=1pt,auto,node distance=3cm,
				semithick, transform shape]
				
				%\\ \multicolumn{1}{c}{ \ldots}
				\node[draw, rounded corners=2mm,outer sep=0] (t0) at (0,0) {\begin{tabular}{l} $\init$ \end{tabular}};
				\node[draw, rounded corners=2mm,outer sep=0, label={[font=\small]50:$t_1$}] (t1) at (0, -2) {\begin{tabular}{l} 
						$a \gets \rd{x}$ \\ $\wrt{x}{1}$
				\end{tabular}};
				\node[draw, rounded corners=2mm,outer sep=0, label={[font=\small]50:$t_2$}] (t2) at (0, -8) {\begin{tabular}{l} 
						$a \gets \rd{y}$ \\ $\wrt{y}{2}$
				\end{tabular}};
				\node[draw, rounded corners=2mm,outer sep=0,  label={[font=\small]50:$t_3$}] (t3) at (0, -4) {\begin{tabular}{l} 
						$b \gets \rd{y}$ \\ $\wrt{y}{1}$
				\end{tabular}};	
				
				\node[draw, rounded corners=2mm,outer sep=0,  label={[font=\small]50:$t_4$}] (t4) at (0, -6) {\begin{tabular}{l} 
						$b \gets \rd{x}$ \\ $\wrt{x}{2}$
				\end{tabular}};		
				
				\path (t0.south west) -- (t0.south) coordinate[pos=0.67] (t0sw);
				\path (t0.south east) -- (t0.south) coordinate[pos=0.67] (t0se);
				\path (t1.north west) -- (t1.north) coordinate[pos=0.67] (t1nw);
				\path (t2.north west) -- (t2.north) coordinate[pos=0.67] (t2nw);
				\path (t2.north east) -- (t2.north) coordinate[pos=0.67] (t2ne);
				\path (t1.south west) -- (t1.south) coordinate[pos=0.67] (t1sw);
				\path (t1.south east) -- (t1.south) coordinate[pos=0.67] (t1se);
				\path (t3.south east) -- (t3.south) coordinate[pos=0.67] (t3se);
				\path (t3.north east) -- (t3.north) coordinate[pos=0.67] (t3ne);
				\path (t3.north west) -- (t3.north) coordinate[pos=0.67] (t3nw);
				\path (t3.south west) -- (t3.south) coordinate[pos=0.67] (t3sw);
				\path (t4.north east) -- (t4.north) coordinate[pos=0.67] (t4ne);
				\path (t4.north west) -- (t4.north) coordinate[pos=0.67] (t4nw);
				
				
				\path (t0.south east) edge [bend left] node {$\wro_y$} (t3.north east);
				\path (t0.south west) edge [bend right] node[above left] {$\wro_x$} (t4.north west);
				\path (t0.south) edge [left] node {$\wro_x$} (t1.north);
				%\path (t1sw) edge [left] node {$\so$} (t2nw);
				\path (t3.south) edge [left] node {$\so$} (t4.north);
				\path (t1.south west) edge [bend right] node [left] {$\so$} (t2.north west);
				%\path (t4nw) edge [left] node {$\wro_x$} (t1se);
				
				\path (t3.south east) edge [bend left] node [right]{$\wro_y$} (t2.north east);
				%\path (t2) edge [right] node {$\wro_x$} (t3);
			\end{tikzpicture}  
			
		}
		\caption{History $2$.}
		\label{fig:pref-impossibility:b}
	\end{subfigure}
	\hspace{.3cm}
	\centering
	\begin{subfigure}[b]{.15\textwidth}
		\resizebox{\textwidth}{!}{
			\begin{tikzpicture}[->,>=stealth',shorten >=1pt,auto,node distance=3cm,
				semithick, transform shape]
				
				%\\ \multicolumn{1}{c}{ \ldots}
				\node[draw, rounded corners=2mm,outer sep=0] (t0) at (0,0) {\begin{tabular}{l} $\init$ \end{tabular}};
				\node[draw, rounded corners=2mm,outer sep=0, label={[font=\small]50:$t_1$}] (t1) at (0, -2) {\begin{tabular}{l} 
						$a \gets \rd{x}$ \\ $\wrt{x}{1}$
				\end{tabular}};
				\node[draw, rounded corners=2mm,outer sep=0, label={[font=\small]50:$t_2$}] (t2) at (0, -8) {\begin{tabular}{l} 
						$a \gets \rd{y}$ \\ $\wrt{y}{2}$
				\end{tabular}};
				\node[draw, rounded corners=2mm,outer sep=0,  label={[font=\small]50:$t_3$}] (t3) at (0, -4) {\begin{tabular}{l} 
						$b \gets \rd{y}$ \\ $\wrt{y}{1}$
				\end{tabular}};	
				
				\node[draw, rounded corners=2mm,outer sep=0,  label={[font=\small]50:$t_4$}] (t4) at (0, -6) {\begin{tabular}{l} 
						$b \gets \rd{x}$ \\ $\wrt{x}{2}$
				\end{tabular}};		
				
				\path (t0.south west) -- (t0.south) coordinate[pos=0.67] (t0sw);
				\path (t0.south east) -- (t0.south) coordinate[pos=0.67] (t0se);
				\path (t1.north west) -- (t1.north) coordinate[pos=0.67] (t1nw);
				\path (t2.north west) -- (t2.north) coordinate[pos=0.67] (t2nw);
				\path (t2.north east) -- (t2.north) coordinate[pos=0.67] (t2ne);
				\path (t1.south west) -- (t1.south) coordinate[pos=0.67] (t1sw);
				\path (t1.south east) -- (t1.south) coordinate[pos=0.67] (t1se);
				\path (t3.south east) -- (t3.south) coordinate[pos=0.67] (t3se);
				\path (t3.north east) -- (t3.north) coordinate[pos=0.67] (t3ne);
				\path (t3.north west) -- (t3.north) coordinate[pos=0.67] (t3nw);
				\path (t3.south west) -- (t3.south) coordinate[pos=0.67] (t3sw);
				\path (t4.north east) -- (t4.north) coordinate[pos=0.67] (t4ne);
				\path (t4.north west) -- (t4.north) coordinate[pos=0.67] (t4nw);
				
				
				\path (t0.south east) edge [bend left] node {$\wro_y$} (t3.north east);
				\path (t1.south west) edge [bend right] node[above left] {$\wro_x$} (t4.north west);
				\path (t0.south) edge [left] node {$\wro_x$} (t1.north);
				%\path (t1sw) edge [left] node {$\so$} (t2nw);
				\path (t3.south) edge [left] node {$\so$} (t4.north);
				\path (t1.south west) edge [bend right] node [left] {$\so$} (t2.north west);
				%\path (t4nw) edge [left] node {$\wro_x$} (t1se);
				
				\path (t3.south east) edge [bend left] node [right]{$\wro_y$} (t2.north east);
				%\path (t2) edge [right] node {$\wro_x$} (t3);
			\end{tikzpicture}  
			
		}
		\caption{History $3$.}
		\label{fig:pref-impossibility:c}
	\end{subfigure}
	
	\hspace{.3cm}
	\centering
	\begin{subfigure}[b]{.15\textwidth}
		\resizebox{\textwidth}{!}{
			\begin{tikzpicture}[->,>=stealth',shorten >=1pt,auto,node distance=3cm,
				semithick, transform shape]
				
				%\\ \multicolumn{1}{c}{ \ldots}
				\node[draw, rounded corners=2mm,outer sep=0] (t0) at (0,0) {\begin{tabular}{l} $\init$ \end{tabular}};
				\node[draw, rounded corners=2mm,outer sep=0, label={[font=\small]50:$t_1$}] (t1) at (0, -2) {\begin{tabular}{l} 
						$a \gets \rd{x}$ \\ $\wrt{x}{1}$
				\end{tabular}};
				\node[draw, rounded corners=2mm,outer sep=0, label={[font=\small]50:$t_2$}] (t2) at (0, -4) {\begin{tabular}{l} 
						$a \gets \rd{y}$ \\ $\wrt{y}{2}$
				\end{tabular}};
				\node[draw, rounded corners=2mm,outer sep=0,  label={[font=\small]50:$t_3$}] (t3) at (0, -6) {\begin{tabular}{l} 
						$b \gets \rd{y}$ \\ $\wrt{y}{1}$
				\end{tabular}};	
				
				\node[draw, rounded corners=2mm,outer sep=0,  label={[font=\small]50:$t_4$}] (t4) at (0, -8) {\begin{tabular}{l} 
						$b \gets \rd{x}$ \\ $\wrt{x}{2}$
				\end{tabular}};		
				
				\path (t0.south west) -- (t0.south) coordinate[pos=0.67] (t0sw);
				\path (t0.south east) -- (t0.south) coordinate[pos=0.67] (t0se);
				\path (t1.north west) -- (t1.north) coordinate[pos=0.67] (t1nw);
				\path (t2.north west) -- (t2.north) coordinate[pos=0.67] (t2nw);
				\path (t2.north east) -- (t2.north) coordinate[pos=0.67] (t2ne);
				\path (t1.south west) -- (t1.south) coordinate[pos=0.67] (t1sw);
				\path (t1.south east) -- (t1.south) coordinate[pos=0.67] (t1se);
				\path (t3.south east) -- (t3.south) coordinate[pos=0.67] (t3se);
				\path (t3.north east) -- (t3.north) coordinate[pos=0.67] (t3ne);
				\path (t3.north west) -- (t3.north) coordinate[pos=0.67] (t3nw);
				\path (t3.south west) -- (t3.south) coordinate[pos=0.67] (t3sw);
				\path (t4.north east) -- (t4.north) coordinate[pos=0.67] (t4ne);
				\path (t4.north west) -- (t4.north) coordinate[pos=0.67] (t4nw);
				
				
				\path (t2.south) edge [left] node {$\wro_y$} (t3.north);
				\path (t1.south west) edge [bend right] node[above right] {$\wro_x$} (t4.north west);
				\path (t0.south) edge [left] node {$\wro_x$} (t1.north);
				%\path (t1sw) edge [left] node {$\so$} (t2nw);
				\path (t3.south) edge [left] node {$\so$} (t4.north);
				\path (t1.south) edge [right] node [left] {$\so$} (t2.north);
				%\path (t4nw) edge [left] node {$\wro_x$} (t1se);
				
				\path (t0.south east) edge [bend left] node [right]{$\wro_y$} (t2.north east);
				%\path (t2) edge [right] node {$\wro_x$} (t3);
			\end{tikzpicture}  
			
		}
		\caption{History $4$.}
		\label{fig:pref-impossibility:d}
	\end{subfigure}\hspace{.5cm}
	\hspace{.3cm}
	\centering
	\begin{subfigure}[b]{.15\textwidth}
	\resizebox{\textwidth}{!}{
		\begin{tikzpicture}[->,>=stealth',shorten >=1pt,auto,node distance=3cm,
			semithick, transform shape]
			
			%\\ \multicolumn{1}{c}{ \ldots}
			\node[draw, rounded corners=2mm,outer sep=0] (t0) at (0,0) {\begin{tabular}{l} $\init$ \end{tabular}};
			\node[draw, rounded corners=2mm,outer sep=0, label={[font=\small]50:$t_1$}] (t1) at (0, -6) {\begin{tabular}{l} 
					$a \gets \rd{x}$ \\ $\wrt{x}{1}$
			\end{tabular}};
			\node[draw, rounded corners=2mm,outer sep=0, label={[font=\small]50:$t_2$}] (t2) at (0, -8) {\begin{tabular}{l} 
					$a \gets \rd{y}$ \\ $\wrt{y}{2}$
			\end{tabular}};
			\node[draw, rounded corners=2mm,outer sep=0,  label={[font=\small]50:$t_3$}] (t3) at (0, -2) {\begin{tabular}{l} 
					$b \gets \rd{y}$ \\ $\wrt{y}{1}$
			\end{tabular}};	
			
			\node[draw, rounded corners=2mm,outer sep=0,  label={[font=\small]50:$t_4$}] (t4) at (0, -4) {\begin{tabular}{l} 
					$b \gets \rd{x}$ \\ $\wrt{x}{2}$
			\end{tabular}};		
			
			\path (t0.south west) -- (t0.south) coordinate[pos=0.67] (t0sw);
			\path (t0.south east) -- (t0.south) coordinate[pos=0.67] (t0se);
			\path (t1.north west) -- (t1.north) coordinate[pos=0.67] (t1nw);
			\path (t2.north west) -- (t2.north) coordinate[pos=0.67] (t2nw);
			\path (t2.north east) -- (t2.north) coordinate[pos=0.67] (t2ne);
			\path (t1.south west) -- (t1.south) coordinate[pos=0.67] (t1sw);
			\path (t1.south east) -- (t1.south) coordinate[pos=0.67] (t1se);
			\path (t3.south east) -- (t3.south) coordinate[pos=0.67] (t3se);
			\path (t3.north east) -- (t3.north) coordinate[pos=0.67] (t3ne);
			\path (t3.north west) -- (t3.north) coordinate[pos=0.67] (t3nw);
			\path (t3.south west) -- (t3.south) coordinate[pos=0.67] (t3sw);
			\path (t4.north east) -- (t4.north) coordinate[pos=0.67] (t4ne);
			\path (t4.north west) -- (t4.north) coordinate[pos=0.67] (t4nw);
			
			
			\path (t0.south) edge [left] node {$\wro_y$} (t3.north);
			\path (t0.south west) edge [bend right] node[above left] {$\wro_x$} (t4.north west);
			\path (t4.south) edge [left] node [left] {$\wro_x$} (t1.north);
			%\path (t1sw) edge [left] node {$\so$} (t2nw);
			\path (t3.south) edge [left] node {$\so$} (t4.north);
			\path (t1.south) edge [right] node [left] {$\so$} (t2.north);
			%\path (t4nw) edge [left] node {$\wro_x$} (t1se);
			
			\path (t3.south east) edge [bend left] node [left]{$\wro_y$} (t2.north east);
			%\path (t2) edge [right] node {$\wro_x$} (t3);
		\end{tikzpicture}  
		
	}
	\caption{History $5$.}
	\label{fig:pref-impossibility:e}
	\end{subfigure}
	\caption{A program along its $\PRE$-consistent histories.}
	\label{fig:pref-impossibility}
	%\vspace{-3mm}
\end{figure}

Let's analyze how algorithm \textcolor{red}{cite algorithm} would behave under program depicted in figure \ref{fig:pref-impossibility:prog}; where $<_{\ora}$ is defined as $t_1 <_{\ora} t_2 <_{\ora} t_3 <_{\ora} t_4$. In this situation, histories $h_1$ and $h_4$ (\ref{fig:pref-impossibility:a} and \ref{fig:pref-impossibility:d}) are computed without any call to $\swap$ function. It is clear that under $h_4$'s branch no swaps can be produced as transactions are totally ordered via $(\so \cup \wro)^*$. However, a swap between $t_2$ and $t_3$ can be produced in the history $h_1' = h_1 \setminus t_4$ (which precedes $h_1$) leading to histories $h_2$ and $h_3$ (figure \ref{fig:pref-impossibility:b} and \ref{fig:pref-impossibility:c}). However, as the $\iread$ event in $t_2$ in both $h_2$ and $h_3$ is swapped, we cannot swap $t_4$ and $t_1$ by $\protocol$'s definition. Therefore, $h_5$ (\ref{fig:pref-impossibility:e}) is never reached.

On the other hand, let $\ora'$ the oracle order defined as $t_1 <_{\ora} t_3 <_{\ora} t_4 <_{\ora} t_2$. In this case, histories $h_1$, $h_2$ and $h_3$ can be obtained without any call to $\swap$. Let $h_3' = h_3 \setminus t_2$; the maximum common prefix of both $h_1$ and $h_3$. Then, in $h_3$ (respectively $h_3'$), every event is maximally added, so $t_2$ and $t_3$ can be swapped to obtain $h_4$ (respectively $t_1$ and $t_4$ to obtain $h_5$). To conclude, with $\ora'$ as oracle order, the algorithm is complete.
