%!TEX root = main.tex
\subsection{Completeness}

By definition, $\textsc{explore-ce}$ is $I$-complete if for any given program $\prog$, it outputs every history in $\histOf[I]{\prog}$. Since $\textsc{explore-ce}$ tracks \emph{ordered} histories, it is useful 

In our algorithm's context, completeness means being able to compute every total consistent history. However, our algorithm works with an extended version of histories where their events are totally ordered. For proving this property, we will need to furnish every history with a total order that coincides with the algorithm's one.
%\textcolor{red}{Problem of this proofs: I assume optimality before proving it! \sout{ I don't know what wording we shall apply} (I tried to modify it, let's see if it is noticeable).}

\subsubsection{Canonical order} $ $\\

As mentioned, we need to formally define a total order for every history that coincide on reachable histories with the history order. For achieving it, we analyze how the algorithm orders transaction logs in a history. In particular, we observe that if two transactions $t, t'$ have a $(\so \cup \wro)^*$ dependency, the history order in the algorithm orders them analogously. But if they are $(\so \cup \wro)^*$-incomparable, the algorithm prioritizes the one that is read by a smaller $\iread$ event according $\ora$. Combining both arguments recursively we obtain a \textit{canonical order} for a history, which is formally defined with the function presented below.

\begin{algorithm}[H]
	\caption{\textsc{Canonical order}}
	\begin{algorithmic}[1]
		
		\Statex
		\Procedure{\textsc{canonicalOrder}}{$h, t, t'$}
		\State \Return $t \ [\so \cup \wro]^* \ t' \lor$
		\State \qquad $(\lnot(t' \ [\so \cup \wro]^* \ t) \land \minimalDependency(h, t, t', \bot)$
		\EndProcedure
		
		\Statex
		\Procedure{\minimalDependency}{$h, t, t', e$}
		\Let $a = \min_{<_{\ora}} \dep(h, t, e)$; $a' = \min_{<_{\ora}} \dep(h, t', e)$
		\If{$a \neq a'$}
		\State \Return $a <_{\ora} a'$
		\Else
		\State \Return $\minimalDependency(h, t, t', a)$
		\EndIf
		\EndProcedure
		
		\Statex
		\Procedure{\dep}{$h, t, e$}
		\State \Return $\{r \ | \exists t' \text{ s.t. } t \ [\so \cup \wro]^* \ t' \land \ t' \ [\wro] \ r \land \trans{h}{r} \ [\so \cup \wro]^+ \trans{h}{e} \} \cup t$ 
		\EndProcedure	
	\end{algorithmic}
	\label{algorithm:canonical-order}
	%\caption{Generic method for exploring every possible history $h$ of a program $\mathcal{P}$ running under a database with $\mathcal{M}$ as isolation level.}
\end{algorithm}


The function $\textsc{canonicalOrder}$ produces a relation between transactions in a history, denoted $\leq^h$. In algorithm \ref{algorithm:canonical-order}'s description, we denote $\bot$ to represent the end of the program, which always exists, and that is $\so$-related with every single transaction.

Firstly, we prove our canonical order is well defined for every pair of transactions.

\begin{lemma}
	\label{lemma:dep_shrinks}
	For every history $h$, event $e$ and transaction $t$, $\dep(h, t, \min_{<_{\ora}} \dep(h, t, e)) \subseteq \dep(h, t, e)$. Moreover, if $\dep(h, t, e) \neq t$, the inclusion is strict.
	
\end{lemma}
\begin{proof}
	Let $r' = \min_{<_{\ora}} \dep(h, t, e)$. If $\dep(h, t, r') = t$ the lemma is trivially proved, so let's suppose there exists $r \in \dep(h, t,r') \setminus t$. Then, $\exists t' \text{ s.t. } t \ [\so \cup \wro]^* \ t' \land \ t' \ [\wro] \ r \land \trans{h}{r} \ [\so \cup \wro]^+ \trans{h}{r'}$ and $\exists t'' \text{ s.t. } t \ [\so \cup \wro]^* \ t'' \land \ t'' \ [\wro] \ r' \land \trans{h}{r'} \ [\so \cup \wro]^+ \trans{h}{e}$; so $\trans{h}{r} \ [\so \cup \wro]^+ \trans{h}{r'} \ [\so \cup \wro]^+ \trans{h}{e}$. In other words, $r \in \dep(h, t, e)$. The moreover comes trivially as $r' \not\in \dep(h, t, r')$.
\end{proof}

\begin{lemma}
	\label{lemma:minimalDependency-halts}
	For every pair of distinct transactions $t, t'$, $\minimalDependency(h,t,t',\bot)$ always halts.
\end{lemma}
\begin{proof}
	Let's suppose by contrapositive that $\minimalDependency(h,t, t',\bot)$ does not halt. Therefore, there would exist an infinite chain of events $e_n, n \in \mathbb{N}$ such that $e_0 = \bot, e_{n+1} = \min_{\ora}\dep(h, t, e_{n}) = \min_{\ora}\dep(h, t', e_{n})$. Firstly, as $h$ is finite, so are both $\dep(h, t, e_{n})$ and $\dep(h, t', e_{n})$. Moreover, if $e_n \not \in t$, $\dep(h, t, e_{n+1}) \subsetneq \dep(h, t, e_{n})$ (and analogously for $t'$). Therefore, there exist some indexes $n_0, m_0$ such that $e_{n_0} \in t$ and $e_{m_0} \in t'$. Let $k = \max\{n_0, m_0\}$. Because ; but if $e_n \in t$, $t = \dep(h, t, e_n)$ and $e_{n+1} = e_n$, so $e_k = e_{n_0}$ and $e_k = e_{m_0}$. Therefore $e_k \in t \cap t'$; so $t = t'$ as transaction logs do not share events; which contradict the assumptions.
\end{proof}

\begin{corollary}
	The relation $\leq^h$ is well defined for every pair of transactions.
\end{corollary}
\begin{proof}
	As by lemma \ref{lemma:minimalDependency-halts}, we know that $\minimalDependency(h, t, t', \bot)$ always halts if $t \neq t'$; it is clear that $\textsc{canonicalOrder}(h, t, t')$ also does it. Therefore, the relation is well defined.
\end{proof}

Now that $\leq^h$ has been proved a well defined relation between each pair of transactions, let us prove that it is indeed a total order.


\begin{lemma}
	\label{lemma:canonincal-total-order}
	The relation $\leq^h$ is a total order.
\end{lemma}
\begin{proof}
	$ $
	\begin{itemize}
		\item \underline{Strongly connection} Let $t_1, t_2$ s.t. $t_1 \not \leq^h t_2$. If $t_2 \ [\so \cup \wro]^* t_1$, then $t_2 \leq^h t_1$. Otherwise, as $\lnot(t_1 \ [\so \cup \wro]^* \ t_2)$ and $\minimalDependency$ halts (lemma \ref{lemma:minimalDependency-halts}) either $\minimalDependency(h, t_1, t_2, \bot)$ or $\minimalDependency(h,t_2, t_1, \bot)$ holds. But as $t_1 \not\leq^h t_2$, $t_2 \leq^h t_1$.
		\item \underline{Reflexivity:} By definition, for every $t$, $t \leq^h t$.
		\item \underline{Transitivity:} Let $t_1, t_2, t_3$ three distinct transactions such that $t_1 \leq^h t_2 $ and $t_2 \leq^h t_3$. Clearly, if $t_1 \ [\so \cup \wro]^* \ t_3$, $t_1 \leq^h t_3$. However, if $t_3 \ [\so \cup \wro]^* \ t_1$, we would find one of the following three scenarios:
		\begin{itemize}
			\item $t_1 \ [\so \cup \wro]^* \ t_2$, which is impossible by strong connectivity as that would mean $t_3 \leq^h t_2$.
			\item $t_2 \ [\so \cup \wro]^* \ t_3$, which is also impossible by strong connectivity, as $t_2 \leq^h t_1$.
			\item $\lnot(t_1 \ [\so \cup \wro]^* \ t_2)$ and $\lnot(t_2 \ [\so \cup \wro]^* \ t_3)$. Then, let us call $e^i_0 = \bot$ and $e^i_{n+1} = \min_{<_{\ora}}\dep(h, t_i, e^i_{n})$ for $i \in \{1,2, 3\}$. Let's prove by induction that if for every $k < n$ $e^1_n \not\in t^1$, then $e^1_n = e^{2}_n = e^3_n$. Clearly this hold for $n = 0$ and, assuming it holds for every $k \leq n-1$, as $t_1 \leq^h t_2$, $t_2 \leq^h t_3$, we know $e^1_n \leq_{\ora} e^2_n \leq_{\ora} e^3_n$ and as $t^3 \ [\so \cup \wro]^* \ t^1$, if $e^1_n \not\in t^1$, $ e^3_n \leq_{\ora} e^1_n$. In other words, they coincide. However, by lemma \ref{lemma:minimalDependency-halts}, we know $\minimalDependency(h, t^1, t^3, \bot)$ halts, so there exists some minimal $n_0$ such that $e^1_{n_0} \in t^1$; so $e^2_{n_0} \in t_1$. That implies $t^2 \ [\so \cup \wro]^* \ t_1$; which is impossible as $t_1 \leq^h t_2$.
		\end{itemize}
		
		We deduce then that either $t_1 \ [\so \cup \wro]^* \ t_3$ or $\lnot(t_3 \ [\so \cup \wro]^* \ t_1)$. In the latter case, let's take the sequence $e^i_n$, $i \in \{1,2,3\}$ defined in the last paragraph. Then, as by lemma \ref{lemma:minimalDependency-halts} $\minimalDependency(h, t_1, t_3, \bot)$ halts, there exists a maximum index $n_0$ such that $e^1_{n_0} = e^2_{n_0} = e^3_{n_0}$. Then $e^1_{n_0 + 1} <_{\ora} e^2_{n_0+1}$ or $e^2_{n_0+1} <_{\ora} e^3_{n_0}$; so $t_1 \leq^h t_3$.
		
		\item \underline{Antisymmetric} Let $t_1, t_2$ s.t. $t_1 \leq^h t_2$ and $t_2 \leq^h t_1$. If $t_1 \ [\so \cup \wro]^* t_2$, then $t_1 = t_2$. If not, by the symmetric argument, $\lnot(t_2 \ [\so \cup \wro]^* t_1)$. In that situation, by lemma \ref{lemma:minimalDependency-halts} we know both $\minimalDependency(h,t_1,t_2,\bot)$ and $\minimalDependency(h,t_1,t_2,\bot)$ halt and cannot be satisfied at the same time. This contradicts that both $t_1 \leq^h t_2$ and $t_2 \leq^h t_1$ hold; so $t_1 = t_2$.		
	\end{itemize}
\end{proof}

\subsubsection{Oracle-respectful histories} $ $\\

The second step in this proof is characterize all reachable histories with some general property that can be generalized to every total history. For doing so, we will show that for reachable histories any history order coincide with its canonical order; so any property based on a history order can be generalized to be based on its canonical order.
\textcolor{red}{I know events may ``disappear'' in an execution and maybe they even have no meaning but I need to reason globally, thinking about the future execution in some way... }

\textcolor{red}{TODO: assume for the moment that every event will appear (no ifs) and then see how can we express this definition for the more general case.}

\begin{definition}
	\label{def:oracle-respectful}
	A reachable history $h$ is \callout{$\ora$-respectful} if it has at most one pending transaction log and for every pair of events $e \in \prog, e' \in h$ s.t. $e \leq_{\ora} e'$, either $e \leq_h e'$ or $\exists e'' \in h, \trans{h}{e''} \leq_{\ora} \trans{h}{e}$ s.t. $\trans{h}{e'} \ [\so \cup \wro]^* \ \trans{h}{e''}$, $e'' \leq_h e$ and $\swapped{h}{e''}$; where if $e \not\in h$ we state $e' \leq_h e$ always hold but $e \leq_h e'$ never does. We will denote it by $R^{\ora}(h)$. %\textcolor{olive}{REVISAR EL CAMBIO DEL $e'' < e$ -> $T'' < T$.}
\end{definition}

\textcolor{red}{I know that in histories the history order does not have subindexes, but I think the proofs remain clearer with them.}

\textcolor{red}{I definitely think transactions shouldn't have histories as an operator, make things confusing in this proof.}

\textcolor{red}{TODO: Add soundness of swappable}
\begin{lemma}
\label{lemma:reachable-or-respectful}
	Every reachable history is $\ora$-respectful.
\end{lemma}
\begin{proof}
	
For proving this property, we will show that in any computable path every history is $\ora$-respectful; and we will prove it by induction on the number of histories this path has. The base case, the empty path, trivially holds; so let us prove the inductive case: for every path of at most length $n$ the property holds. Let $p$ a path of length $n+1$ and $h$ the last reachable history of this path. As $p \setminus \{h\}$ is a computable path of length $n$, the immediate predecessor of $h$ in $p$, $h_p$ is $\ora$-respectful. Let $e = \nextEvent(h_p)$.

Firstly, if $e$ is not a $\iread$ nor a $\ibegin$ event and $h = h_p \oplus e$, as $\leq_h$ is an extension of $\leq_{h_p}$, $e$ belongs to the only pending transactions and oracle order orders transactions completely, we can deduce that $h$ is $\ora$-respectful. In addition, if $e$ is a $\ibegin$ event and $h = h_p \oplus e$, let $a \in \prog, b \in h$ s.t. $a <_{\ora} b$. If $a \in h_p$ or $b \neq e$, as $\leq_{h}$ is an extension of $\leq_{h_p}$ and $R^{\ora}(h_p)$ holds, $R^{\ora}(h)$ also does it. Moreover, as $e = \min_{\ora} \prog \setminus h_p$, there is no event $a \in \prog \setminus h_p$ s.t. $a \leq_{\ora} e$; so $h$ is $\ora$-respectful.

Moreover, if $e$ is a $\iread$ event and $h = h_p \oplus \wro(e, t)$ for some transaction log $t$, let us call $a \in \prog, b \in h$ s.t. $a <_{\ora} b$. Once again, if $a \in h$ or $b \neq e$ the property holds; so let's suppose $a \in \prog \setminus h_p$ and $b = e$. Let $d = \ibegin(\trans{h}{e})$, that also belongs to $h_p$. As $R^{\ora}(h_p)$ and $a \not\in h_p$, $a \leq_{\ora} d$; so there exists $c \in h_p$, $\trans{h}{c} \leq_{\ora} \trans{h}{a}$ s.t. $\trans{h}{d} \ [\so \cup \wro]^* \ \trans{h}{c}$, $c \leq_h d$ and $\swapped{h}{c}$. As $\trans{h}{r} = \trans{h}{d}$, we conclude $R^{\ora}(h)$.

But if any previous case holds, it is because $h = \swap(h_p \oplus e, r, t)$ for some $r, t \in h_p$ s.t. $\protocol(h_p \oplus e, r, t)$ holds. Let $a, b$ two events s.t. $a \leq_{\ora} b $. On one hand, if $a \leq_{h} b$ or $a \not\leq_{h_p} b$, as $R^{\ora}(h_p)$ and $\protocol(h_p \oplus e, r, t)$ holds, the property is satisfied. On the other hand, if $b <_{h} a$ and $a \leq_{h_p} b$, $a$ has to be a deleted event so $a \in \prog \setminus h \cup \{r\}$. As $r \leq_{h_p} a$, if $a \leq_{\ora} r$, there would exist a $c \in h $, $\trans{h}{c} \leq_{\ora} \trans{h}{a} \leq_{\ora} \trans{h}{r}$ s.t. $\trans{h}{r} \ [\so \cup \wro]^* \ \trans{h}{c}$ and $\swapped{h}{c}$. However, this contradicts $\protocol(h_p \oplus e, r, t)$; so $r \leq_{\ora} a$. Taking $e'' = r$ the property is witnessed. 
	%The only transaction in the history whose relative order has changed is $\trans{h}{r}$, so $b \in \trans{h}{r}$. In that setting, we can take $r$ as a
\end{proof}

%For proving this property, we will show that in any computable path every history is $\ora$-respectful; and we will prove it by induction on the number of histories this path has. The base case, the empty path, trivially holds; so let us prove the inductive case: for every path of at most length $n$ the property holds. Let $p$ a path of length $n+1$ and $h$ the last reachable history of this path. As $p \setminus \{h\}$ is a computable path of length $n$, the immediate predecessor of $h$ in $p$, $h_p$ is $\ora$-respectful. Let $e = \nextEvent(h_p)$.

\begin{proposition}
	\label{proposition:orders-coincide}
	For any reachable history $h$, $\leq^h \equiv \leq_h$.
\end{proposition}
\begin{proof}
For proving this equivalence, we will show that in any computable path $t \leq_h t'$, then $t \leq^h t'$, as by lemma \ref{lemma:canonincal-total-order} $\leq^h$ is a total order and therefore they have to coincide; and we will prove it by induction on the number of histories this path has. The base case, the empty path, trivially holds; so let us prove the inductive case: for every path of at most length $n$ the property holds. Let $p$ a path of length $n+1$ and $h$ the last reachable history of this path. As $p \setminus \{h\}$ is a computable path of length $n$, the immediate predecessor of $h$ in $p$, $\leq^{h_p} \equiv \leq_{h_p}$. Let $e = \nextEvent(h_p)$.
	\begin{itemize}
		\item \underline{$h = h_p \oplus e$ and $e$ is a $ \iend, \iwrite$:} As $h_p$ and $h$ are edge-wise identical, $\leq^h \equiv \leq_h$.
		
		\item \underline{$h = h_p \oplus e$ and $e$ is a $\ibegin$:} As $\dep(h_p, t, \bot) = \dep(h, t, \bot)$ for every transaction in $h_p$, if $t \leq^{h_p} t'$, then $t \leq^h t'$. Moreover, $\dep(h, \trans{h}{e}, \bot) = \{e\}= \min_{\ora} \prog \setminus h_p$. By lemma \ref{lemma:reachable-or-respectful} $h$ is $\ora$-respectful, so for every $t$, $\min_{\ora} \dep(h, t, \bot) <_{\ora} e$; which implies $t <^h \trans{h}{e}$. By lemma \ref{lemma:canonincal-total-order}, $\leq^h$ is a total order, so it coincides with $\leq_h$.
		
		\item \underline{$h = h_p \oplus \wro(e, t)$ for some $t \in h_p$ and $e$ is a $\iread$:} As no transaction depends on $\trans{h}{e}$ and $\trans{h}{e} = \last{h_p}$, if we prove that for every pair of transactions $\minimalDependency(h_p, t', t'', \bot) $ $= \minimalDependency(h, t', t'', \bot)$, the lemma would hold. On one hand, $\dep(h, \trans{h}{e}, \bot) = \dep(h_p, \trans{h}{e}, \bot) = \trans{h}{e}$ and in the other hand, by lemma \ref{lemma:reachable-or-respectful}, $\min_{\ora} \dep(h_p, t, \bot) <_{\ora} \trans{h}{e}$. Finally, as $e \not\in \dep(h, \hat{t}, e')$, for every $\hat{t} \neq \trans{h}{e}, e' \neq \bot$, for every pair of transactions $t', t''$, $\minimalDependency(h_p, t', t'' \bot) = \minimalDependency(h, t', t'', \bot)$. 
		
		\item \underline{$h = \swap(h_p, r, t)$, where $t = \trans{h}{e}$:} As $\protocol(h \oplus e, r, t)$ is satisfied and $h$ is $\ora$-respectful, for every event $e'$ and transaction $t'$, $\min_{\ora} \dep(h_p, t', e') = \min_{\ora} \dep(h, t', e')$, so for every pair of transactions $\minimalDependency(h_p, t', t'', \bot) = \minimalDependency(h, t', t'', \bot)$. In particular, this implies $t' \leq^{h_p} t''$ if and only if $t' \leq^h t''$ for every  pair $t', t''$ and $t' \leq^h \trans{h}{r}$; so $\leq^h \equiv \leq_h$. 
	\end{itemize}
\end{proof}


Proposition \ref{proposition:orders-coincide} is a very interesting result as it express the following fact: regardless of the computable path that leads to a history, the final order between events will be the same. This result will have a key role during both completeness and optimality, as it restricts the possible histories that precede another while describing the computable path leading to it. In addition, proposition \ref{proposition:orders-coincide} together with lemma \ref{lemma:reachable-or-respectful} justify enlarging definition \ref{def:oracle-respectful} with the canonical order instead the computable order; and it is this new shape the one we will be using during the rest of proof.  

%As $\leq^h \equiv \leq_h$ for any reachable history, we will extend $R^{\ora}(h)$ to any history changing $\leq_h$ to $\leq^h$ in \ref{def:oracle-respectful} whenever it is needed. This property is not something reachable histories satisfy but also, as next lemma shows, total histories with $\leq^h$ order do; which justify it as an useful tool for proving completeness.
\begin{lemma}
	\label{lemma:total-respectful}
	Any total history is $\ora$-respectful.
	\begin{proof}
		Let $h$ be a total history and $t, t'$ a pair of transactions s.t. $t \leq_{\ora} t'$. If $t \leq^h t'$, then the statement is satisfied; so let's assume the contrary: $t' \leq^h t$. If $t' \ [\so \cup \wro]^* \ t$, then for every $e \in t, e' \in t'$ $\exists c \in h$ s.t. $\trans{h}{c} \leq_{\ora} \trans{h}{e}$, $\trans{h}{e'} \ [\so \cup \wro]^* \tr(c)$, $\swapped{h}{c}$ and $c \leq^h e$; so the property is satisfied. Otherwise, by definition of $\minimalDependency$, there exists $r' \in h$ s.t. $t' \ [\so \cup \wro]^* \ \trans{h}{r'}$ and $\trans{h}{r'} \leq_{\ora} T$. Moreover, by \textsc{canonicalOrder}'s definition, $\trans{h}{r} \leq^h T$. Finally $\swapped{h}{r'}$ holds as it is the minimum element according $\ora$. To sum up, $R^{\ora}(h)$ holds.
	\end{proof}
\end{lemma}


\begin{algorithm}[H]
	\caption{\textsc{prev}}
	\begin{algorithmic}[1]
		
		\Statex
		\Procedure{\textsc{prev}}{$h$}
		\If{$h = \emptyset$}
		\State \Return $\emptyset$
		\EndIf
		\State $a \gets \last{h}$
		\If{$\lnot \swapped{h}{a}$}
		\State \Return $h \setminus a$
		\Else
		\State \Return $\maxCompletion(h\setminus a, \{e \ | \ e \not\in (h \setminus a) \land e <_{\ora} h.\wro(a) \})$
		\EndIf
		\EndProcedure
		
		\Statex
		\Procedure{\maxCompletion}{$h, D$}
		\If{$D \neq \emptyset$}
		\State $e \gets \min_{<_{\ora}} D$
		\If{$e.type() \neq \iread$}
		\State \Return $\maxCompletion(h \bullet e, D \setminus \{e\})$
		\Else
		\Let $w$ s.t. $\isMaximallyAdded{h \bullet_w e, e}$
		\State \Return $\maxCompletion(h \bullet_w e, D \setminus \{e\})$
		\EndIf
		
		\Else
		\State \Return $h$
		\EndIf
		\EndProcedure
		
		
	\end{algorithmic}
	\label{algorithm:prev}
	%\caption{Generic method for exploring every possible history $h$ of a program $\mathcal{P}$ running under a database with $\mathcal{M}$ as isolation level.}
\end{algorithm}

Function \ref{algorithm:prev} produce a history that are meant to be the previous step of a reachable history. Thanks to this definition, we will show that every total history has a computable path based on applying $\prev^{-1}$ function iteratively until the objective history is reached.


\textcolor{red}{TODO (somewhere before): if $h \to \swap(h \bullet e, r, w)$ ``in one step'', actually from $h$ we go to $h \bullet e$ and from it to the swapped. }

%code for splitting algorithmic environment
%\algstore{myalg}
%\end{algorithmic}
%\end{algorithm}

%gap maybe needed to split algorithms in two parts. 

%\begin{algorithm}[H]                   
%\begin{algorithmic} [1]                   % enter the algorithmic environment
%\algrestore{myalg}	



\begin{lemma}
	\label{lemma:prev-respectful}
	For every $\ora$-respectful history $h$, $\prev(h)$ is also $\ora$-respectful.
	\begin{proof}
		Let suppose $h \neq \emptyset$, $h_p = \prev(h)$, $a = \last{h}$, $e \in \mathcal{P}$ and $ e' \in h_p$ s.t. $e \leq_{\ora} e'$. As $R^{\ora}(h)$ is satisfied, either $e \leq^h e'$ or $\exists e'' \in h, \tr(e'') \leq_{\ora} \tr(e)$, $e'' \leq^h e$, $\tr(e') \ [\so \cup \wro]^* \tr(e'')$ and $\swapped{h}{e''}$. If $\lnot \swapped{h}{a}$, $h_p = h \bullet a$; so if $e \leq^h e'$, $e \leq^{h_p} e'$ and if not, $e'' \in h_p$, so $R^{\ora}(h_p)$ holds. 
		
		Otherwise, $\swapped{h}{a}$ and we distinguish between the sets $e$ and $e'$ belong to. Firstly, for every pair of events $\hat{e} \in h_p \setminus h$, $\hat{e}' \in \dep(h, \tr(\hat{e}, \bot))$, we know that $\tr(\hat{e}) \leq_{\ora} \tr(\hat{e}')$. Therefore, $\min_{<_{\ora}} \dep(h, \tr(\hat{e}, \bot)) = \ibegin(\tr(\hat{e}))$. In addition, by construction of $\prev(h)$ and $\ora$-respectfulness of h, for every $\hat{e} \in h, e'' \in h$, $\min_{<_{\ora}} \dep(h_p, \tr(\hat{e}), e'') = \min_{<_{\ora}} \dep(h, \tr(\hat{e}), e'')$. Combining both results, if $e'$ belong to $h$, either $e \leq^{h_p} e'$ or exists a $e'' \in h$ s.t. $e'' \leq^{h_p} e$ and witness $R^{\ora}(h)$ for $e,e'$ (regardless of $e$'s belonging to $h$, $e'' \leq^{h_p} e$). On the contrary, as $h_p$ has no pending transactions, if $e' \not\in h$, $\lnot (\tr(e') \ [\so \cup \wro]^* \ \tr(e))$, so regardless if $\tr(e) \ [\so \cup \wro]^* \tr(e')$, $e \leq^{h_p} e'$. To sum up, $R^{\ora}(h_p)$ holds.
	\end{proof}
\end{lemma}

\begin{lemma}
\label{lemma:soundness-prev}
For every consistent history $\ora$-respectful $h$, if $\prev(h)$ is reachable, then $h$ is also reachable.
\begin{proof}
	Let suppose $h \neq \emptyset$, $h_p = \prev(h)$ and $a = \last{h}$. If $\lnot \swapped{h}{a}$, let $h_n = h_p \bullet a$ if $a$ is not a read, $h_n = h_p \bullet_{h.\wro(a)} a$ in the other case. Either way, $h_n$ is always reachable and it coincides with $h$. Otherwise, $a$ is a $\iread$ event and it swapped; so let us call $w = h.\wro(a)$. Firstly, as $\swapped{h}{a}$, $a <_{\ora} w$, and by lemma \ref{lemma:reachable-or-respectful}, $R^{\ora}(h_p)$ holds, so $a <_{h_p} w$ does; which let us conclude $\compute(h_p)$ will always return $(a, w)$ as a possible swap pair. In addition, all transactions in $h_p$ are non-pending, so in particular $\last{h_p}$ is an $\iend$ event. If we call $h_s = \swap(h_p, a, w)$, and $h_p \setminus h = h_p \setminus h_s$ would hold, as $h \subseteq h_p, h_s \subseteq h_p$, then $h = h_s$; which would allow us to conclude $h$ is reachable from $h_p$.
	
	On one hand, if $e \in h_p \setminus h$, $e \not\in h$ and $e <_{\ora} w$. In particular, $\lnot (\tr(e) \ [\so \cup \wro]^* \ \tr(w))$. Moreover, if $ e \leq_{\ora} a$, by $R^{\ora}(h)$, either $e \leq^h a$ or $\exists e''\in h, e'' \leq_{\ora} e$ s.t. $\tr(a) \ [\so \cup \wro]^* \tr(e'')$, $e'' \leq^h e$ and $\swapped{h}{e''}$; both impossible situations as $e \not\in h$ and $a = \last{h}$; so $a \leq_{\ora} e$. In other words, $e \in h_p \setminus h_s$.
	
	On the other hand, $e \in h_p \setminus h_s$ if and only if $\lnot (\tr(e) \ [\so \cup \wro]^* \ \tr(w))$ and $a <_{\ora} e <_{\ora} w$. If $e \in h$ then $e \leq^{h} a$, and as $h$ is $\ora$-respectful and $a \leq_{\ora} e$, we deduce there exists a $e'' \in h$ s.t. $\tr(e'') \leq_{\ora} \tr(a)$, $\tr(e) \ [\so \cup \wro]^* \tr(e'')$ and $\swapped{h}{e''}$. Moreover, as $c \in h$, $c \in h_p$; but as $\swapped{h_p}{c}$ and $\protocol(h, a, w)$ hold, $c \in h_s$ and so $e$ does. This result leads to a contradiction, so $e \not\in h$; i.e. $e \in h_p \setminus h$.
\end{proof}
\end{lemma}

\begin{corollary}
\label{corollary:prev-swap-identity}
In a consistent $\ora$-respectful history $h$ whose previous history is reachable, if its last event $a$ is swapped, $h$ coincides with $\swap(\prev(h), a, h.\wro(a))$.
\begin{proof}
It comes straight away from the proof of lemma \ref{lemma:soundness-prev}.
\end{proof}
\end{corollary}

\begin{lemma}
\label{lemma:prev-reduces-one}
For every non-empty consistent $\ora$-respectful history $h$, $h_p = \prev(h)$ and $a = \last{h}$, if $\swapped{h}{a}$ then $\{e \in h_p \ | \ \swapped{h_p}{e}\} = \{e \in h \ | \ \swapped{h}{e}\} \setminus \{a\}$, otherwise $h_p = h \setminus a$.
%Let $h$ a history, $h' = \prev(h)$ and at some state $s$, $h'$ appear at line \ref{algorithm:stmc:outer_loop}, from $s$ the algorithm will compute some state $s'$ such that $h$ will also appear at line \ref{algorithm:stmc:outer_loop}.`
\begin{proof}
		Let $a = \last{h}$ and $h' = h \setminus a$. If $a$ is not swapped, then $h_p = h'$, so the lemma holds immediately. Otherwise, as $h_p =  \maxCompletion(h')$, we will show that every event not belonging to $h_p \setminus h'$ is not swapped by induction on every recursive call to $\maxCompletion$. Let us call $D = \{e \ | \ e \not\in h' \land e <_{\ora} \}$. This set, intuitively, contain all the events that would have been deleted from a reachable history $h$ to produce $h_p$. In this setting, let us call $h_{|D|} = h'$, $D_{|D|} = D$ and $D_k = D_{k+1} \setminus \{\min_{<_{\ora}} D_{k+1}\}, \; e_k = \min_{<_{\ora}}D_k$ for every $k, 0 \leq k < |D|$ (i.e. $D_k = D_{k+1} \setminus \{ e_{k+1}\}$). We will prove the lemma by induction on $n = |D| - k$, constructing a collection of histories $h_k$, $0 \leq k < |D|$, such that each one is an extension of its predecessor with a non-swapped event.
		
		The base case, $h_{|D|}$ is trivial as by its definition it corresponds with $h'$. Let's prove the inductive case: $\{e \ | \ \swapped{h_{k+1}}{e}\} = \{e \ | \ \swapped{h'}{e}\}$. If $e_{k+1}$ is not a $\iread$ event, $h_k = h_{k+1} \bullet e_{k+1}$ and $\{e \ | \ \swapped{h_{k}}{e}\} = \{e \ | \ \swapped{h'}{e}\}$; as only $\iread$ events can be swapped. Otherwise, by the model's causal-extensibility there exists a $\iwrite$ event $f_{k+1}$ s.t. writes the same variable and $\isConsistent{h_{k+1} \bullet_{f_{k+1}} e_{k+1}} \land \tr(f_{k+1}) \ [\so \cup \wro]^* \ \tr(e_{k+1})$ holds. $ \{e \ | \ \swapped{h_{k+1}}{e}\} = \{e \ | \ \swapped{h_{k+1} \bullet_{f_{k+1}} e_{k+1}}{e}\}$ holds. Let $E_{k+1} = \{w \ | \ \isConsistent{h_{k+1} \bullet_{w} e_{k+1}} \land \{e \ | \ \swapped{h_{k+1}}{e}\} = \{e \ | \ \swapped{h_{k+1} \bullet_{w} e_{k+1}}{e}\}\}$ and $w_{k+1} = \max_{\leq^{h_{k+1}}} E_{k+1}$. This element is well defined as $f_{k+1}$ belongs to $E_{k+1}$. Therefore, $h_k = h_{k+1} \bullet_{w_{k+1}} e_{k´+1}$ is consistent and $\{e \ | \ \swapped{h_{k}}{e}\} = \{e \ | \ \swapped{h'}{e}\}$. Moreover, let's remark that as $w_{k+1}$ is the maximum write event according to $\leq_{h_{k+1}}$ s.t. $\isConsistent{h_k}$ and $\{e \ | \ \swapped{h_{k}}{e}\} = \{e \ | \ \swapped{h'}{e}\}$ and $R^{\ora}(h)$, it also satisfies $\isMaximallyAdded{h_{k}}{ e_{k+1}, w_{k+1}}$. Altogether, we obtain $h_p = h_0$; which let us conclude $\{e \in h_p \ | \ \swapped{h_p}{e}\} = \{e \in h' \ | \ \swapped{h'}{e}\} = \{e \in h \ | \ \swapped{h}{e}\}\setminus \{a\}$.
	\end{proof}
\end{lemma}

\begin{lemma}
	\label{lemma:prev-leads-empty}
	For every history $h$ there exists some $k_h \in \mathbb{N}$ such that $\prev^{k_h}(h) = \emptyset$.
	\begin{proof}
		This lemma is immediate consequence of lemma \ref{lemma:prev-reduces-one}. Let us call $\xi(h) = |\{e \in h \ | \ \swapped{h}{e}\}$, the number of swapped events in $h$, and let us prove the lemma by induction on $(\xi(h), |h|)$. The base case, $\xi(h) = |h| = 0$ is trivial as $h$ would be $\emptyset$; so let's assume that for every history $h$ such that $\xi(h) < n$ or $\xi(h) =h \land |h| < m$ there exists such $k_h$. Let $h$ then a history s.t. $\xi(h) = n$ and $|h| = m$. $h_p = \prev(h)$. On one hand, if $h_p = h \setminus a$ then $\xi(x_p) = \xi(h)$ and $|h_p| = |h|-1$. On the other hand, if $h_p \neq h \setminus a$, $\xi(h_p) = \xi(h) - 1$. In any case, by induction hypothesis on $h_p$, there exists an integer $k_{h_p}$ such that $\prev^{k_{h_p}}(h_p) = \emptyset$. Therefore, $k_h = k_{h_p}+ 1$ satisfies $\prev^{k_h}(h) = \emptyset$.
	\end{proof}
\end{lemma}

\begin{proposition}
	\label{proposition:chain-histories-proof}
	For every consistent $\ora$-respectful history $h$ exists $k \in \mathbb{N}$ and some sequence of $\ora$-respectful histories $\{h_n\}_{n = 0}^k$, $h_0 = \emptyset$ and $h_k = h$ such that the algorithm will compute.
	\begin{proof}
		Let $h$ a history, $k$ the minimum integer such that $\prev^k(h) = \emptyset$, which exists thanks to lemma \ref{lemma:prev-leads-empty} and $C = \{\prev^{k-n}(h)\}_{n = 0}^k$ a set of indexed histories. By the collection's definition and lemma \ref{lemma:prev-respectful}, $h_0 = \prev^k(h) = \emptyset$, $h_k = \prev^0(h) = h$ and $R^{\ora}(h_n)$ for every $n \in \mathbb{N}$; so let us prove by induction on $n$ that every history in $C$ is reachable. The base case, $h_0$, is trivially achieved; as it is always reachable. In addition, by lemma \ref{lemma:soundness-prev}, we know that if $h_n$ is reachable, $h_{n+1}$ is it too; which proves the inductive step. %Moreover, so we will focus on the inductive case, assuming $h_n$ is reachable and deducing $h_{n+1}$ is it too. 
	\end{proof}
\end{proposition}

\begin{theorem}
\label{theorem:completeness}
	Algorithm \ref{algorithm:optimal-instantiated} is complete.
	\begin{proof}
		By lemma \ref{lemma:total-respectful}, any consistent total history is $\ora$-respectful. As a consequence of proposition \ref{proposition:chain-histories-proof}, there exist a sequence of reachable histories which $h$ belongs to; so in particular, $h$ is reachable.
	\end{proof}
\end{theorem}

