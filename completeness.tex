%!TEX root = main.tex
\subsection{Completeness}

By definition, $\textsc{explore-ce}$ is $I$-complete if for any given program $\prog$, it outputs every history in $\histOf[I]{\prog}$. Let $h \in \histOf[I]{\prog}$. Our objective is to produce a computable path of ordered histories that lead to $h$ (i.e. a (finite) ordered collection of ordered histories such that $h_0 = \emptyset$ and for every $n$, if $e= \nextEvent(h_n)$ $h_{n+1} = h_n \oplus e$, $h_{n+1} = h_n \oplus \wro(e, t)$ for some $t \in h_n$ or $h_{n+1} = \swap(h_n, r, t)$ for some $r, t \in h_n$).

However, algorithm works with ordered histories. Therefore, we first have to furnish $h$ with a total order called \textit{canonical order} that, if $h$ were reachable, it would coincide with its history order.  Secondly, we describe a function $\prev$ defined over the set of all partial histories that, if $h$ is reachable, $\prev(h)$ returns the previous history of $h$ computed by $\textsc{explore-ce}$. Then, we prove that there exists a collection of histories $h_n = h$, $h_0 = \emptyset$ and $h_n = \prev(h_{n+1})$. As it ends in the initial state, we can therefore prove that this collection conforms an actual computable path; which allow us to conclude that $h$ is reachable. Nevertheless, for proving both the equivalence between history order and canonical order and the soundness of function $\prev$ we will define the notion of \textit{$or$-respectfulness}, an invariant satisfied by every reachable history based on the events' relative positions in the oracle order.

%\textcolor{red}{Problem of this proofs: I assume optimality before proving it! \sout{ I don't know what wording we shall apply} (I tried to modify it, let's see if it is noticeable).}

\subsubsection{Canonical order} $ $\\

As mentioned, we need to formally define a total order for every history that coincide on reachable histories with the history order. For achieving it, we analyze how the algorithm orders transaction logs in a history. In particular, we observe that if two transactions $t, t'$ have a $(\so \cup \wro)^*$ dependency, the history order in the algorithm orders them analogously. But if they are $(\so \cup \wro)^*$-incomparable, the algorithm prioritizes the one that is read by a smaller $\iread$ event according $\ora$. Combining both arguments recursively we obtain a \textit{canonical order} for a history, which is formally defined with the function presented below.

\begin{algorithm}[H]
	\caption{\textsc{Canonical order}}
	\begin{algorithmic}[1]
		
		\Statex
		\Procedure{\textsc{canonicalOrder}}{$h, t, t'$}
		\State \Return $t \ [\so \cup \wro]^* \ t' \lor$
		\State \qquad $(\lnot(t' \ [\so \cup \wro]^* \ t) \land \minimalDependency(h, t, t', \bot)$
		\EndProcedure
		
		\Statex
		\Procedure{\minimalDependency}{$h, t, t', e$}
		\Let $a = \min_{<_{\ora}} \dep(h, t, e)$; $a' = \min_{<_{\ora}} \dep(h, t', e)$
		\If{$a \neq a'$}
		\State \Return $a <_{\ora} a'$
		\Else
		\State \Return $\minimalDependency(h, t, t', a)$
		\EndIf
		\EndProcedure
		
		\Statex
		\Procedure{\dep}{$h, t, e$}
		\State \Return $\{r \ | \exists t' \text{ s.t. } t \ [\so \cup \wro]^* \ t' \land \ t' \ [\wro] \ r \land \trans{h}{r} \ [\so \cup \wro]^+ \trans{h}{e} \} \cup t$ 
		\EndProcedure	
	\end{algorithmic}
	\label{algorithm:canonical-order}
	%\caption{Generic method for exploring every possible history $h$ of a program $\mathcal{P}$ running under a database with $\mathcal{M}$ as isolation level.}
\end{algorithm}


The function $\textsc{canonicalOrder}$ produces a relation between transactions in a history, denoted $\leq^h$. In algorithm \ref{algorithm:canonical-order}'s description, we denote $\bot$ to represent the end of the program, which always exists, and that is $\so$-related with every single transaction.

Firstly, we prove our canonical order is well defined for every pair of transactions.

\begin{lemma}
	\label{lemma:dep_shrinks}
	For every history $h$, event $e$ and transaction $t$, $\dep(h, t, \min_{<_{\ora}} \dep(h, t, e)) \subseteq \dep(h, t, e)$. Moreover, if $\dep(h, t, e) \neq t$, the inclusion is strict.
	
\end{lemma}
\begin{proof}
	Let $r' = \min_{<_{\ora}} \dep(h, t, e)$. If $\dep(h, t, r') = t$ the lemma is trivially proved, so let's suppose there exists $r \in \dep(h, t,r') \setminus t$. Then, $\exists t' \text{ s.t. } t \ [\so \cup \wro]^* \ t' \land \ t' \ [\wro] \ r \land \trans{h}{r} \ [\so \cup \wro]^+ \trans{h}{r'}$ and $\exists t'' \text{ s.t. } t \ [\so \cup \wro]^* \ t'' \land \ t'' \ [\wro] \ r' \land \trans{h}{r'} \ [\so \cup \wro]^+ \trans{h}{e}$; so $\trans{h}{r} \ [\so \cup \wro]^+ \trans{h}{r'} \ [\so \cup \wro]^+ \trans{h}{e}$. In other words, $r \in \dep(h, t, e)$. The moreover comes trivially as $r' \not\in \dep(h, t, r')$.
\end{proof}

\begin{lemma}
	\label{lemma:minimalDependency-halts}
	For every pair of distinct transactions $t, t'$, $\minimalDependency(h,t,t',\bot)$ always halts.
\end{lemma}
\begin{proof}
	Let's suppose by contrapositive that $\minimalDependency(h,t, t',\bot)$ does not halt. Therefore, there would exist an infinite chain of events $e_n, n \in \mathbb{N}$ such that $e_0 = \bot, e_{n+1} = \min_{\ora}\dep(h, t, e_{n}) = \min_{\ora}\dep(h, t', e_{n})$. Firstly, as $h$ is finite, so are both $\dep(h, t, e_{n})$ and $\dep(h, t', e_{n})$. Moreover, if $e_n \not \in t$, $\dep(h, t, e_{n+1}) \subsetneq \dep(h, t, e_{n})$ (and analogously for $t'$). Therefore, there exist some indexes $n_0, m_0$ such that $e_{n_0} \in t$ and $e_{m_0} \in t'$. Let $k = \max\{n_0, m_0\}$. Because ; but if $e_n \in t$, $t = \dep(h, t, e_n)$ and $e_{n+1} = e_n$, so $e_k = e_{n_0}$ and $e_k = e_{m_0}$. Therefore $e_k \in t \cap t'$; so $t = t'$ as transaction logs do not share events; which contradict the assumptions.
\end{proof}

\begin{corollary}
	The relation $\leq^h$ is well defined for every pair of transactions.
\end{corollary}
\begin{proof}
	As by lemma \ref{lemma:minimalDependency-halts}, we know that $\minimalDependency(h, t, t', \bot)$ always halts if $t \neq t'$; it is clear that $\textsc{canonicalOrder}(h, t, t')$ also does it. Therefore, the relation is well defined.
\end{proof}

Now that $\leq^h$ has been proved a well defined relation between each pair of transactions, let us prove that it is indeed a total order.


\begin{lemma}
	\label{lemma:canonincal-total-order}
	The relation $\leq^h$ is a total order.
\end{lemma}
\begin{proof}
	$ $
	\begin{itemize}
		\item \underline{Strongly connection} Let $t_1, t_2$ s.t. $t_1 \not \leq^h t_2$. If $t_2 \ [\so \cup \wro]^* t_1$, then $t_2 \leq^h t_1$. Otherwise, as $\lnot(t_1 \ [\so \cup \wro]^* \ t_2)$ and $\minimalDependency$ halts (lemma \ref{lemma:minimalDependency-halts}) either $\minimalDependency(h, t_1, t_2, \bot)$ or $\minimalDependency(h,t_2, t_1, \bot)$ holds. But as $t_1 \not\leq^h t_2$, $t_2 \leq^h t_1$.
		\item \underline{Reflexivity:} By definition, for every $t$, $t \leq^h t$.
		\item \underline{Transitivity:} Let $t_1, t_2, t_3$ three distinct transactions such that $t_1 \leq^h t_2 $ and $t_2 \leq^h t_3$. Clearly, if $t_1 \ [\so \cup \wro]^* \ t_3$, $t_1 \leq^h t_3$. However, if $t_3 \ [\so \cup \wro]^* \ t_1$, we would find one of the following three scenarios:
		\begin{itemize}
			\item $t_1 \ [\so \cup \wro]^* \ t_2$, which is impossible by strong connectivity as that would mean $t_3 \leq^h t_2$.
			\item $t_2 \ [\so \cup \wro]^* \ t_3$, which is also impossible by strong connectivity, as $t_2 \leq^h t_1$.
			\item $\lnot(t_1 \ [\so \cup \wro]^* \ t_2)$ and $\lnot(t_2 \ [\so \cup \wro]^* \ t_3)$. Then, let us call $e^i_0 = \bot$ and $e^i_{n+1} = \min_{<_{\ora}}\dep(h, t_i, e^i_{n})$ for $i \in \{1,2, 3\}$. Let's prove by induction that if for every $k < n$ $e^1_n \not\in t^1$, then $e^1_n = e^{2}_n = e^3_n$. Clearly this hold for $n = 0$ and, assuming it holds for every $k \leq n-1$, as $t_1 \leq^h t_2$, $t_2 \leq^h t_3$, we know $e^1_n \leq_{\ora} e^2_n \leq_{\ora} e^3_n$ and as $t^3 \ [\so \cup \wro]^* \ t^1$, if $e^1_n \not\in t^1$, $ e^3_n \leq_{\ora} e^1_n$. In other words, they coincide. However, by lemma \ref{lemma:minimalDependency-halts}, we know $\minimalDependency(h, t^1, t^3, \bot)$ halts, so there exists some minimal $n_0$ such that $e^1_{n_0} \in t^1$; so $e^2_{n_0} \in t_1$. That implies $t^2 \ [\so \cup \wro]^* \ t_1$; which is impossible as $t_1 \leq^h t_2$.
		\end{itemize}
		
		We deduce then that either $t_1 \ [\so \cup \wro]^* \ t_3$ or $\lnot(t_3 \ [\so \cup \wro]^* \ t_1)$. In the latter case, let's take the sequence $e^i_n$, $i \in \{1,2,3\}$ defined in the last paragraph. Then, as by lemma \ref{lemma:minimalDependency-halts} $\minimalDependency(h, t_1, t_3, \bot)$ halts, there exists a maximum index $n_0$ such that $e^1_{n_0} = e^2_{n_0} = e^3_{n_0}$. Then $e^1_{n_0 + 1} <_{\ora} e^2_{n_0+1}$ or $e^2_{n_0+1} <_{\ora} e^3_{n_0}$; so $t_1 \leq^h t_3$.
		
		\item \underline{Antisymmetric} Let $t_1, t_2$ s.t. $t_1 \leq^h t_2$ and $t_2 \leq^h t_1$. If $t_1 \ [\so \cup \wro]^* t_2$, then $t_1 = t_2$. If not, by the symmetric argument, $\lnot(t_2 \ [\so \cup \wro]^* t_1)$. In that situation, by lemma \ref{lemma:minimalDependency-halts} we know both $\minimalDependency(h,t_1,t_2,\bot)$ and $\minimalDependency(h,t_1,t_2,\bot)$ halt and cannot be satisfied at the same time. This contradicts that both $t_1 \leq^h t_2$ and $t_2 \leq^h t_1$ hold; so $t_1 = t_2$.		
	\end{itemize}
\end{proof}

\subsubsection{Oracle-respectful histories} $ $\\

The second step in this proof is characterize all reachable histories with some general property that can be generalized to every total history. For doing so, we will show that for reachable histories any history order coincide with its canonical order; so any property based on a history order can be generalized to be based on its canonical order.
\textcolor{red}{I know events may ``disappear'' in an execution and maybe they even have no meaning but I need to reason globally, thinking about the future execution in some way... }

\textcolor{red}{TODO: assume for the moment that every event will appear (no ifs) and then see how can we express this definition for the more general case.}

\begin{definition}
	\label{def:oracle-respectful}
	A reachable history $h$ is \callout{$\ora$-respectful} if it has at most one pending transaction log and for every pair of events $e \in \prog, e' \in h$ s.t. $e \leq_{\ora} e'$, either $e \leq_h e'$ or $\exists e'' \in h, \trans{h}{e''} \leq_{\ora} \trans{h}{e}$ s.t. $\trans{h}{e'} \ [\so \cup \wro]^* \ \trans{h}{e''}$, $e'' \leq_h e$ and $\swapped{h}{e''}$; where if $e \not\in h$ we state $e' \leq_h e$ always hold but $e \leq_h e'$ never does. We will denote it by $R^{\ora}(h)$. %\textcolor{olive}{REVISAR EL CAMBIO DEL $e'' < e$ -> $T'' < T$.}
\end{definition}

\textcolor{red}{I know that in histories the history order does not have subindexes, but I think the proofs remain clearer with them.}

\textcolor{red}{I definitely think transactions shouldn't have histories as an operator, make things confusing in this proof.}

\textcolor{red}{TODO: Add soundness of swappable}
\begin{lemma}
\label{lemma:reachable-or-respectful}
	Every reachable history is $\ora$-respectful.
\end{lemma}
\begin{proof}
	
For proving this property, we will show that in any computable path every history is $\ora$-respectful; and we will prove it by induction on the number of histories this path has. The base case, the empty path, trivially holds; so let us prove the inductive case: for every path of at most length $n$ the property holds. Let $p$ a path of length $n+1$ and $h$ the last reachable history of this path. As $p \setminus \{h\}$ is a computable path of length $n$, the immediate predecessor of $h$ in $p$, $h_p$ is $\ora$-respectful. Let $e = \nextEvent(h_p)$.

Firstly, if $e$ is not a $\iread$ nor a $\ibegin$ event and $h = h_p \oplus e$, as $\leq_h$ is an extension of $\leq_{h_p}$, $e$ belongs to the only pending transactions and oracle order orders transactions completely, we can deduce that $h$ is $\ora$-respectful. In addition, if $e$ is a $\ibegin$ event and $h = h_p \oplus e$, let $a \in \prog, b \in h$ s.t. $a <_{\ora} b$. If $a \in h_p$ or $b \neq e$, as $\leq_{h}$ is an extension of $\leq_{h_p}$ and $R^{\ora}(h_p)$ holds, $R^{\ora}(h)$ also does it. Moreover, as $e = \min_{\ora} \prog \setminus h_p$, there is no event $a \in \prog \setminus h_p$ s.t. $a \leq_{\ora} e$; so $h$ is $\ora$-respectful.

Moreover, if $e$ is a $\iread$ event and $h = h_p \oplus \wro(e, t)$ for some transaction log $t$, let us call $a \in \prog, b \in h$ s.t. $a <_{\ora} b$. Once again, if $a \in h$ or $b \neq e$ the property holds; so let's suppose $a \in \prog \setminus h_p$ and $b = e$. Let $d = \ibegin(\trans{h}{e})$, that also belongs to $h_p$. As $R^{\ora}(h_p)$ and $a \not\in h_p$, $a \leq_{\ora} d$; so there exists $c \in h_p$, $\trans{h}{c} \leq_{\ora} \trans{h}{a}$ s.t. $\trans{h}{d} \ [\so \cup \wro]^* \ \trans{h}{c}$, $c \leq_h d$ and $\swapped{h}{c}$. As $\trans{h}{r} = \trans{h}{d}$, we conclude $R^{\ora}(h)$.

But if any previous case holds, it is because $h = \swap(h_p \oplus e, r, t)$ for some $r, t \in h_p$ s.t. $\protocol(h_p \oplus e, r, t)$ holds. Let $a, b$ two events s.t. $a \leq_{\ora} b $. On one hand, if $a \leq_{h} b$ or $a \not\leq_{h_p} b$, as $R^{\ora}(h_p)$ and $\protocol(h_p \oplus e, r, t)$ holds, the property is satisfied. On the other hand, if $b <_{h} a$ and $a \leq_{h_p} b$, $a$ has to be a deleted event so $a \in \prog \setminus h \cup \{r\}$. As $r \leq_{h_p} a$, if $a \leq_{\ora} r$, there would exist a $c \in h $, $\trans{h}{c} \leq_{\ora} \trans{h}{a} \leq_{\ora} \trans{h}{r}$ s.t. $\trans{h}{r} \ [\so \cup \wro]^* \ \trans{h}{c}$ and $\swapped{h}{c}$. However, this contradicts $\protocol(h_p \oplus e, r, t)$; so $r \leq_{\ora} a$. Taking $e'' = r$ the property is witnessed. 
	%The only transaction in the history whose relative order has changed is $\trans{h}{r}$, so $b \in \trans{h}{r}$. In that setting, we can take $r$ as a
\end{proof}

%For proving this property, we will show that in any computable path every history is $\ora$-respectful; and we will prove it by induction on the number of histories this path has. The base case, the empty path, trivially holds; so let us prove the inductive case: for every path of at most length $n$ the property holds. Let $p$ a path of length $n+1$ and $h$ the last reachable history of this path. As $p \setminus \{h\}$ is a computable path of length $n$, the immediate predecessor of $h$ in $p$, $h_p$ is $\ora$-respectful. Let $e = \nextEvent(h_p)$.

\begin{proposition}
	\label{proposition:orders-coincide}
	For any reachable history $h$, $\leq^h \equiv \leq_h$.
\end{proposition}
\begin{proof}
For proving this equivalence, we will show that in any computable path $t \leq_h t'$, then $t \leq^h t'$, as by lemma \ref{lemma:canonincal-total-order} $\leq^h$ is a total order and therefore they have to coincide; and we will prove it by induction on the number of histories this path has. The base case, the empty path, trivially holds; so let us prove the inductive case: for every path of at most length $n$ the property holds. Let $p$ a path of length $n+1$ and $h$ the last reachable history of this path. As $p \setminus \{h\}$ is a computable path of length $n$, the immediate predecessor of $h$ in $p$, $\leq^{h_p} \equiv \leq_{h_p}$. Let $e = \nextEvent(h_p)$.
	\begin{itemize}
		\item \underline{$h = h_p \oplus e$ and $e$ is a $ \iend, \iwrite$:} As $h_p$ and $h$ are edge-wise identical, $\leq^h \equiv \leq_h$.
		
		\item \underline{$h = h_p \oplus e$ and $e$ is a $\ibegin$:} As $\dep(h_p, t, \bot) = \dep(h, t, \bot)$ for every transaction in $h_p$, if $t \leq^{h_p} t'$, then $t \leq^h t'$. Moreover, $\dep(h, \trans{h}{e}, \bot) = \{e\}= \min_{\ora} \prog \setminus h_p$. By lemma \ref{lemma:reachable-or-respectful} $h$ is $\ora$-respectful, so for every $t$, $\min_{\ora} \dep(h, t, \bot) <_{\ora} e$; which implies $t <^h \trans{h}{e}$. By lemma \ref{lemma:canonincal-total-order}, $\leq^h$ is a total order, so it coincides with $\leq_h$.
		
		\item \underline{$h = h_p \oplus \wro(e, t)$ for some $t \in h_p$ and $e$ is a $\iread$:} As no transaction depends on $\trans{h}{e}$ and $\trans{h}{e} = \last{h_p}$, if we prove that for every pair of transactions $\minimalDependency(h_p, t', t'', \bot) $ $= \minimalDependency(h, t', t'', \bot)$, the lemma would hold. On one hand, $\dep(h, \trans{h}{e}, \bot) = \dep(h_p, \trans{h}{e}, \bot) = \trans{h}{e}$ and in the other hand, by lemma \ref{lemma:reachable-or-respectful}, $\min_{\ora} \dep(h_p, t, \bot) <_{\ora} \trans{h}{e}$. Finally, as $e \not\in \dep(h, \hat{t}, e')$, for every $\hat{t} \neq \trans{h}{e}, e' \neq \bot$, for every pair of transactions $t', t''$, $\minimalDependency(h_p, t', t'' \bot) = \minimalDependency(h, t', t'', \bot)$. 
		
		\item \underline{$h = \swap(h_p, r, t)$, where $t = \trans{h}{e}$:} As $\protocol(h \oplus e, r, t)$ is satisfied and $h$ is $\ora$-respectful, for every event $e'$ and transaction $t'$, $\min_{\ora} \dep(h_p, t', e') = \min_{\ora} \dep(h, t', e')$, so for every pair of transactions $\minimalDependency(h_p, t', t'', \bot) = \minimalDependency(h, t', t'', \bot)$. In particular, this implies $t' \leq^{h_p} t''$ if and only if $t' \leq^h t''$ for every  pair $t', t''$ and $t' \leq^h \trans{h}{r}$; so $\leq^h \equiv \leq_h$. 
	\end{itemize}
\end{proof}


Proposition \ref{proposition:orders-coincide} is a very interesting result as it express the following fact: regardless of the computable path that leads to a history, the final order between events will be the same. This result will have a key role during both completeness and optimality, as it restricts the possible histories that precede another while describing the computable path leading to it. In addition, proposition \ref{proposition:orders-coincide} together with lemma \ref{lemma:reachable-or-respectful} justify enlarging definition \ref{def:oracle-respectful} with the canonical order instead the computable order; and it is this new shape the one we will be using during the rest of proof.  

%As $\leq^h \equiv \leq_h$ for any reachable history, we will extend $R^{\ora}(h)$ to any history changing $\leq_h$ to $\leq^h$ in \ref{def:oracle-respectful} whenever it is needed. This property is not something reachable histories satisfy but also, as next lemma shows, total histories with $\leq^h$ order do; which justify it as an useful tool for proving completeness.
\begin{lemma}
	\label{lemma:total-respectful}
	Any total history is $\ora$-respectful.
	\begin{proof}
		Let $h$ be a total history and $t, t'$ a pair of transactions s.t. $t \leq_{\ora} t'$. If $t \leq^h t'$, then the statement is satisfied; so let's assume the contrary: $t' \leq^h t$. If $t' \ [\so \cup \wro]^* \ t$, then for every $e \in t, e' \in t'$ $\exists c \in h$ s.t. $\trans{h}{c} \leq_{\ora} \trans{h}{e}$, $\trans{h}{e'} \ [\so \cup \wro]^* \tr(c)$, $\swapped{h}{c}$ and $c \leq^h e$; so the property is satisfied. Otherwise, by definition of $\minimalDependency$, there exists $r' \in h$ s.t. $t' \ [\so \cup \wro]^* \ \trans{h}{r'}$ and $\trans{h}{r'} \leq_{\ora} T$. Moreover, by \textsc{canonicalOrder}'s definition, $\trans{h}{r} \leq^h T$. Finally $\swapped{h}{r'}$ holds as it is the minimum element according $\ora$. To sum up, $R^{\ora}(h)$ holds.
	\end{proof}
\end{lemma}

\subsubsection{Previous of a history} 
$ $\\

As a third and final step in our proof, we define the function \textit{previous} that, for a every history $h$, if $\prev(h)$ is reachable, then $h$ is also reachable. Moreover, $\prev(h)$ will belong to the same computable path.

\begin{algorithm}[H]
		\label{algorithm:prev}
	\caption{\textsc{prev}}
	\begin{algorithmic}[1]
		
		\Statex
		\Procedure{\textsc{prev}}{$h$}
		\If{$h = \emptyset$}
		\State \Return $\emptyset$
		\EndIf
		\State $a \gets \last{h}$
		\If{$\lnot \swapped{h}{a}$}
		\State \Return $h \setminus a$
		\Else
		\Let $t$ s.t. $(t,r) \in \wro$.
		\State \Return $\maxCompletion(h\setminus a, \{e \ | \ e \not\in (h \setminus a) \land e <_{\ora} t \})$
		\EndIf
		\EndProcedure
\algstore{myalg}
\end{algorithmic}
\end{algorithm}

%gap maybe needed to split algorithms in two parts. 

\begin{algorithm}[H]                   
\begin{algorithmic} [1]                   % enter the algorithmic environment
\algrestore{myalg}			
		\Statex
		\Procedure{\maxCompletion}{$h, D$}
		\If{$D \neq \emptyset$}
		\State $e \gets \min_{<_{\ora}} D$
		\If{$\mathit{type}(e) \neq \iread$}
		\State \Return $\maxCompletion(h \oplus e, D \setminus \{e\})$
		\Else
		\Let $t$ s.t. $\isMaximallyAdded{h \oplus \wro(t, e)}{e}$ holds
		\State \Return $\maxCompletion(h \oplus \wro(t, e), D \setminus \{e\})$
		\EndIf
		
		\Else
		\State \Return $h$
		\EndIf
		\EndProcedure
		
		
	\end{algorithmic}

	%\caption{Generic method for exploring every possible history $h$ of a program $\prog$ running under a database with $\mathcal{M}$ as isolation level.}
\end{algorithm}

%code for splitting algorithmic environment
%\algstore{myalg}
%\end{algorithmic}
%\end{algorithm}

%gap maybe needed to split algorithms in two parts. 

%\begin{algorithm}[H]                   
%\begin{algorithmic} [1]                   % enter the algorithmic environment
%\algrestore{myalg}	

First, we show that the invariant of our algorithm is preserved via $\prev$.

\begin{lemma}
	\label{lemma:prev-respectful}
	For every $\ora$-respectful history $h$, $\prev(h)$ is also $\ora$-respectful.
\end{lemma}
\begin{proof}
	Let suppose $h \neq \emptyset$, $h_p = \prev(h)$, $a = \last{h}$, $e \in \prog$ and $ e' \in h_p$ s.t. $e \leq_{\ora} e'$. We explore different cases depending if $e, e'$ belong to $h$ or not. If $e' \in h_p \setminus h$, $\lnot(\swapped{h_p}{e})$ and $ \lnot(\swapped{h_p}{e'})$ holds. As $\min_{<_{\ora}} \dep(h, \trans{h}{e'}, \bot) = \ibegin(\trans{h}{e'})$, we obtain that $\min_{<_{\ora}}\dep(h,\trans{h}{e'}) \leq_{\ora} e' \leq_{\ora} \ibegin(\trans{h}{e'})$. Therefore, as $e' \in h_p \in h$, $\lnot(\trans{h}{e'} \ [\so \cup \wro]^+ \ \trans{h}{e})$, so $e \leq^h e'$. And if $e' \in h$, either $e \leq^h e'$ or $e' \leq^h e$. In the former case, both are in $h$ and therefore, in $h_p$. As it cannot happen that $e' \in \trans{h}{a}$ and $e \leq^{h_p} a$ because $\swapped{h}{a}$ and $e \leq_{\ora} e'$, we conclude that $e \leq^h e'$ ($\leq_{h_p}$ keeps the relative orders between transactions different from $\trans{h}{a}$ and by lemma \ref{reachable-or-respectful} they coincide). In the latter case,  by $\oraRespectfulCanon{h}$, there exists $e''$ that witness it. In particular, $\swapped{h}{e''}$ holds, so $e'' \in h_p$. $e''$ witness $\oraRespectfulCanon{h_p}$ holds. In the three cases we deduce that $\oraRespectfulCanon{h_p}$.
	%As $\oraRespectfulCanon{h}$ is satisfied, either $e \leq^h e'$ or $\exists e'' \in h, \trans{h}{e''} \leq_{\ora} \trans{h}{e}$, $e'' \leq^h e$, $\trans{h}{e'} \ [\so \cup \wro]^* \trans{h}{e''}$ and $\swapped{h}{e''}$. If $\lnot \swapped{h}{a}$, $h = h_p \oplus a$. Therefore, if $e \leq^h e'$, we have that $e \leq^{h_p} e'$ holds, but if not, $e'' \in h_p$ and witness the property. In both case $\oraRespectfulCanon{h_p}$ holds.
	
	%and every other event $\hat{e}'$ s.t. $\hat{e}' \in \dep(h, \trans{h}{\hat{e}}, \bot)$, we know that $\trans{h}{\hat{e}} \leq_{\ora} \trans{h}{\hat{e}'}$ as neither $\hat{e}$ nor any event after it in $h$ is swapped in $h$.
	  
	
	
	 %as $e' \leq^{h_p} e$, there exists a $e'' \in h$ s.t. $e'' \leq^{h} e$ that witness $\oraRespectfulCanon{h}$ for $e,e'$. Thus, as $e \in h_p$, $e'' \in h_p$; and therefore, $\oraRespectful{h_p}{e'}$ holds.
	
	%Combining both results, if $e'$ belong to $h$, either $e \leq^{h_p} e'$, so $\oraRespectful{h_p}{e'}$ holds or exists a $e'' \in h$ s.t. $e'' \leq^{h_p} e$ and witness $\oraRespectfulCanon{h}$ for $e,e'$. Thus, as $e \in h_p$, $e'' \in h_p$; and therefore, $\oraRespectful{h_p}{e'}$ holds. But if not, $e' \in h_p \setminus h$. As $h_p$ has no pending transactions, $\lnot (\trans{h}{e'} \ [\so \cup \wro]^* \ \trans{h}{e})$, so regardless if $\trans{h}{e} \ [\so \cup \wro]^* \trans{h}{e'}$, $e \leq^{h_p} e'$. To sum up, $\oraRespectfulCanon{h_p}$ holds.
\end{proof}

Next, we have to prove that previous is a sound function, i.e. the composition between $\textsc{explore-ce}$ and $\prev$ give us the identity. For doing so, in the case a history is a swap, we deduce that both histories should not contain the same elements and they read the same; so they have to coincide.

\begin{lemma}
	\label{lemma:soundness-prev}
	For every consistent history $\ora$-respectful $h$, if $\prev(h)$ is reachable, then $h$ is also reachable.
\end{lemma}
\begin{proof}
	Let suppose $h \neq \emptyset$, $h_p = \prev(h)$ and $a = \last{h}$. If $\lnot \swapped{h}{a}$, let $h_n = h_p \oplus a$ if $a$ is not a read and $h_n = h_p \oplus \wro(t, a)$, where $t$ is the transaction s.t. $(t, r) \in \wro$, otherwise. Either way, $h_n$ is always reachable and it coincides with $h$. On the contrary, if $\swapped{h}{a}$, $a$ is a $\iread$ event and it swapped; so let us call $t$ to the transaction s.t. $(t, r) \in \wro$. Firstly, as $\swapped{h}{a}$, $a <_{\ora} t$, and by lemma \ref{lemma:reachable-or-respectful}, $\oraRespectfulCanon{h_p}$ holds, so $a <_{h_p} t$ does; which let us conclude $\compute(h_p)$ will always return $(a, t)$ as a possible swap pair. In addition, all transactions in $h_p$ are non-pending, so in particular $\last{h_p}$ is an $\iend$ event. If we call $h_s = \swap(h_p, a, t)$, and we prove that $h_p \setminus h = h_p \setminus h_s$ holds, then we would deduce $h = h_s$ as $\wro(t, a)$ in both $h_p, h_s$ and $h \subseteq h_p, h_s \subseteq h_p$; which would allow us to conclude $h$ is reachable from $h_p$.
	
	On one hand, if $e \in h_p \setminus h$, we deduce that $e \not\in h$ and $e <_{\ora} t$. In particular, $\lnot (\trans{h}{e} \ [\so \cup \wro]^* \ t)$. Moreover, if $ e \leq_{\ora} a$, by $\oraRespectfulCanon{h}$, either $e \leq^h a$ or $\exists e''\in h, e'' \leq_{\ora} e$ s.t. $\tr(a) \ [\so \cup \wro]^* \trans{h}{e''}$, $e'' \leq^h e$ and $\swapped{h}{e''}$; both impossible situations as $e \not\in h$ and $a = \last{h}$; so $a \leq_{\ora} e$. In other words, $e \in h_p \setminus h_s$.
	
	On the other hand, $e \in h_p \setminus h_s$ if and only if $\lnot (\trans{h}{e} \ [\so \cup \wro]^* \ \tr(w))$ and $a <_{\ora} e <_{\ora} w$. If $e$ would belong to $h$ then $e \leq^{h} a$. As $h$ is $\ora$-respectful and $a \leq_{\ora} e$, we deduce there exists a $e'' \in h$ s.t. $\trans{h}{e''} \leq_{\ora} \tr(a)$, $\trans{h}{e} \ [\so \cup \wro]^* \trans{h}{e''}$ and $\swapped{h}{e''}$. Moreover, as $e'' \in h$, $e'' \in h_p$. By corollary \ref{corollary:soundness-swapped} $\swapped{h_p}{e''}$ and $\genericProtocol(h_p, a, t)$ hold,  $e'' \in h_s$ and so $e$ does. This result leads to a contradiction, so $e \not\in h$; i.e. $e \in h_p \setminus h$.
\end{proof}


\begin{corollary}
	\label{corollary:prev-swap-identity}
	In a consistent $\ora$-respectful history $h$ whose previous history is reachable, if $a = \last{h}$, $\swapped{h}{a}$ and $t$ is a transaction such that $(t, a) \in \wro$, $h$ coincides with $\swap(\prev(h), a, t)$.
\end{corollary}
\begin{proof}
	It comes straight away from the proof of lemma \ref{lemma:soundness-prev}.
\end{proof}

Once proven that $\prev$ is sound, let us prove that for every history we can compose $\prev$ a finite number of times obtaining the empty history. We are going to prove it by induction on the number of swapped events, so we prove first the recursive composition finishes in finite time and then we conclude our claim.

\begin{lemma}
	\label{lemma:prev-reduces-one}
	For every non-empty consistent $\ora$-respectful history $h$, $h_p = \prev(h)$ and $a = \last{h}$, if $\swapped{h}{a}$ then $\{e \in h_p \ | \ \swapped{h_p}{e}\} = \{e \in h \ | \ \swapped{h}{e}\} \setminus \{a\}$, otherwise $h_p = h \setminus a$.
	%Let $h$ a history, $h' = \prev(h)$ and at some state $s$, $h'$ appear at line \ref{algorithm:stmc:outer_loop}, from $s$ the algorithm will compute some state $s'$ such that $h$ will also appear at line \ref{algorithm:stmc:outer_loop}.`
	\begin{proof}
		Let $a = \last{h}$ and $h' = h \setminus a$. If $\lnot(\swapped{h}{a})$, then $h_p = h'$ and the lemma holds trivially. Otherwise, as $h_p =  \maxCompletion(h')$, we will show that every event not belonging to $h_p \setminus h'$ is not swapped by induction on every recursive call to $\maxCompletion$. Let us call $D = \{e \ | \ e \not\in h' \land e <_{\ora} \}$. This set, intuitively, contain all the events that would have been deleted from a reachable history $h$ to produce $h_p$. In this setting, let us call $h_{|D|} = h'$, $D_{|D|} = D$ and $D_k = D_{k+1} \setminus \{\min_{<_{\ora}} D_{k+1}\}, \; e_k = \min_{<_{\ora}}D_k$ for every $k, 0 \leq k < |D|$ (i.e. $D_k = D_{k+1} \setminus \{ e_{k+1}\}$). We will prove the lemma by induction on $n = |D| - k$, constructing a collection of histories $h_k$, $0 \leq k < |D|$, such that each one is an extension of its predecessor with a non-swapped event.
		
		The base case, $h_{|D|}$ is trivial as by its definition it corresponds with $h'$. Let's prove the inductive case: $\{e \ | \ \swapped{h_{k+1}}{e}\} = \{e \ | \ \swapped{h'}{e}\}$. If $e_{k+1}$ is not a $\iread$ event, $h_k = h_{k+1} \oplus e_{k+1}$ and $\{e \ | \ \swapped{h_{k}}{e}\} = \{e \ | \ \swapped{h'}{e}\}$; as only $\iread$ events can be swapped. Otherwise, $e_{k+1}$ is a read event. By the isolation level's causal-extensibility there exists a transaction $f_{k+1}$ that writes the same variable as $e_{k+1}$, $f_{k+1} \ [\so \cup \wro]^* \ \trans{h}{e_{k+1}}$ and $h_{k+1} \oplus \wro( f_{k+1}, e_{k+1})$ is consistent. Moreover, by corollary \ref{corollary:soundness-swapped} $ \{e \ | \ \swapped{h_{k+1}}{e}\} = \{e \ | \ \swapped{h_{k+1} \oplus \wro(f_{k+1}, e_{k+1})}{e}\}$ holds. 
		
		Let $E_{k+1} = \{t \ | \ h_{k+1} \oplus \wro(t, e_{k+1}) \text{ is consistent }  \land \{e \ | \ \swapped{h_{k+1}}{e}\} = \{e \ | \ \swapped{h_{k+1} \oplus \wro(t, e_{k+1}) }{e}\}\}$ and let $t_{k+1} = \max_{\leq^{h_{k+1}}} E_{k+1}$. This element is well defined as $f_{k+1}$ belongs to $E_{k+1}$. Therefore, $h_k = h_{k+1} \oplus \wro(t_{k+1}, e_{k+1})$ is consistent and $\{e \ | \ \swapped{h_{k}}{e}\} = \{e \ | \ \swapped{h'}{e}\}$. Moreover, let's remark that as $t_{k+1}$ is the maximum transaction according to $\leq_{h_{k+1}}$ s.t. is consistent and $\{e \ | \ \swapped{h_{k}}{e}\} = \{e \ | \ \swapped{h'}{e}\}$. In addition, as $\oraRespectfulCanon{h}$ holds, it also satisfies $\isMaximallyAdded{h_{k}}{ e_{k+1}, w_{k+1}}$. Altogether, we obtain $h_p = h_0$; which let us conclude $\{e \in h_p \ | \ \swapped{h_p}{e}\} = \{e \in h' \ | \ \swapped{h'}{e}\} = \{e \in h \ | \ \swapped{h}{e}\}\setminus \{a\}$.
	\end{proof}
\end{lemma}


\begin{lemma}
	\label{lemma:prev-leads-empty}
	For every history $h$ there exists some $k_h \in \mathbb{N}$ such that $\prev^{k_h}(h) = \emptyset$.
	\begin{proof}
		This lemma is immediate consequence of lemma \ref{lemma:prev-reduces-one}. Let us call $\xi(h) = |\{e \in h \ | \ \swapped{h}{e}\}|$, the number of swapped events in $h$, and let us prove the lemma by induction on $(\xi(h), |h|)$. The base case, $\xi(h) = |h| = 0$ is trivial as $h$ would be $\emptyset$; so let's assume that for every history $h$ such that $\xi(h) < n$ or $\xi(h) =h \land |h| < m$ there exists such $k_h$. Let $h$ then a history s.t. $\xi(h) = n$ and $|h| = m$. $h_p = \prev(h)$. On one hand, if $h_p = h \setminus a$ then $\xi(x_p) = \xi(h)$ and $|h_p| = |h|-1$. On the other hand, if $h_p \neq h \setminus a$, $\xi(h_p) = \xi(h) - 1$. In any case, by induction hypothesis on $h_p$, there exists an integer $k_{h_p}$ such that $\prev^{k_{h_p}}(h_p) = \emptyset$. Therefore, $k_h = k_{h_p}+ 1$ satisfies $\prev^{k_h}(h) = \emptyset$.
	\end{proof}
\end{lemma}

\begin{proposition}
	\label{proposition:chain-histories-proof}
	For every consistent $\ora$-respectful history $h$ exists $k \in \mathbb{N}$ and some sequence of $\ora$-respectful histories $\{h_n\}_{n = 0}^k$, $h_0 = \emptyset$ and $h_k = h$ such that the algorithm will compute.
	\begin{proof}
		Let $h$ a history, $k$ the minimum integer such that $\prev^k(h) = \emptyset$, which exists thanks to lemma \ref{lemma:prev-leads-empty} and $C = \{\prev^{k-n}(h)\}_{n = 0}^k$ a set of indexed histories. By the collection's definition and lemma \ref{lemma:prev-respectful}, $h_0 = \prev^k(h) = \emptyset$, $h_k = \prev^0(h) = h$ and $R^{\ora}(h_n)$ for every $n \in \mathbb{N}$; so let us prove by induction on $n$ that every history in $C$ is reachable. The base case, $h_0$, is trivially achieved; as it is always reachable. In addition, by lemma \ref{lemma:soundness-prev}, we know that if $h_n$ is reachable, $h_{n+1}$ is it too; which proves the inductive step. %Moreover, so we will focus on the inductive case, assuming $h_n$ is reachable and deducing $h_{n+1}$ is it too. 
	\end{proof}
\end{proposition}

\begin{theorem}
	\label{theorem:completeness}
	The algorithm \textup{\textsc{explore-ce}} is complete.
	\begin{proof}
		By lemma \ref{lemma:total-respectful}, any consistent total history is $\ora$-respectful. As a consequence of proposition \ref{proposition:chain-histories-proof}, there exist a sequence of reachable histories which $h$ belongs to; so in particular, $h$ is reachable.
	\end{proof}
\end{theorem}

