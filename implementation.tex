\section{Implementation}

We implemented algorithm \ref{algorithm:algo-class} on top of Java PathFinder (JPF) \textcolor{red}{cite JPF}, a well-known verification tool for concurrent programs. We do not rely on any concrete database but store the information as separated state from JPF's state and assume the database is always correct (i.e. we ensure every isolation level by an axiomatic check on the database state along all operations applied on it). Internally, the database state is stored as a collection of String, one per write instruction executed; along with the $\so$, $\wro$-dependencies between the events associated to them. Therefore, we can simulate from simple variables to actual tables provided the client has proper methods to transform from one to the other representative.

Our implementation is nonetheless compatible with JPF's standards, as the main routine is a particular instantiation of a search algorithm. Moreover, we built consistency checkers based on the algorithm's presented in \textcolor{red}{Ranadeep and Enea's paper} for different isolation levels to allow a modular implementation of every algorithm described. In particular, thanks to the inherent JPF modularity, depending on the parameters received by the main JPF routine, we can select both the algorithm and the isolation levels employed; which simplify the client's usage.

However, for performance issues, we developed an iterative version of the algorithm \ref{algorithm:algo-class} for reducing the actual memory consumption (JPF executes programs under a Java Virtual Machine running on top of another Java Virtual Machine, which explain its standard memory consumption). In addition, as JPF is not designed for supporting database operations, we developed an API that simulates every database instruction. The set of instructions provided only allow $\ibegin, \iwrite, \iread$ and $\icommit$ operations, but they have been proved expressive enough both during our experiments and in real time applications. We admit any Java program that (1) can be parsed and executed by JPF and (2) has an equivalent translation into a program written with the syntax defined in Figure~\ref{fig:syntax}.

